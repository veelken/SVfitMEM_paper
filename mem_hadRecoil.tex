\subsection{Treatment of hadronic activity}
\label{sec:mem_hadRecoil}

The phase-space for QCD radiation is large at the centre-of-mass
energies of the LHC
and as a consequence, particles with masses up to a few hundred GeV 
are typically produced in association with a sizeable hadronic activity~\cite{Alwall:2010cq}.
We refer to the vectorial sum of all particles in the event that do not originate from the $\PHiggs$ boson decay
as the ``hadronic recoil'' and denote the corresponding momentum by $\bm{p}^{\rec}$.
CMS, as well as ATLAS, have developed sophisticated techniques to improve the reconstruction 
of the hadronic recoil~\cite{CMS-JME-13-003,ATLAS-CONF-2014-019}.

The effect of the hadronic recoil is to modify the kinematics of the $\PHiggs$ boson decay
by boosting the $\Pgt$ leptons and their decay products,
in the transverse plane and in longitudinal direction.
The longitudinal component of $\bm{p}^{\rec}$ has a small effect on 
the Bjorken scaling variables $x_{a}$ and $x_{b}$ only and can be neglected.
However, the components of $\bm{p}^{\rec}$ in the transverse plane need to be accounted for.
The LO ME that we use to model the production of the $\PHiggs$ boson production 
is adequate for events without hadronic activity,
in which the $\PHiggs$ boson has zero $\pT$.
In order to use the LO ME in analyses at the LHC,
the momenta of the visible $\Pgt$ decay products and of the neutrinos produced in the $\Pgt$ decays
need to be transformed into a frame in which the $\PHiggs$ boson has zero $\pT$.
The transformation is achieved by a Lorentz boost in the transverse plane,
using $(\hat{E}^{\rec} = \pThat^{\rec}, \pXhat^{\rec}, \pYhat^{\rec}, 0)$ as boost
vector.
We denote the true values of energies and momenta by symbols with a
hat if they are given in the laboratory frame, and by symbols a tilde in case they are given in the zero transverse momentum frame of the $\PHiggs$ boson.
Symbols with neither hat nor tilde denote values measured in the
laboratory frame.

The hadronic recoil is reconstructed with a typical resolution of $10$--$20$~\GeV at the LHC~\cite{CMS-JME-13-003,ATLAS-CONF-2014-019}.
It is demonstrated in Ref.~\cite{Alwall:2010cq} that the imperfect reconstruction of the hadronic recoil
in the detector may cause a significant bias on the $\PHiggs$ boson mass reconstructed by the ME method in case one ignores
the experimental resolution on the
hadronic recoil. We account
for the experimental resolution by introduction
of a TF $W_{\rec}( \pX^{\rec}, \pY^{\rec} | \pXhat^{\rec},
\pYhat^{\rec} )$.
We then marginalize Eq.~(\ref{eq:mem}) with respect to $\pXhat^{\rec}$ and $\pYhat^{\rec}$:
\begin{align}
& \mathcal{P}(\bm{y};\pX^{\rec},\pY^{\rec}|m_{\PHiggs}) =
\frac{1}{\sigma(m_{\PHiggs})} \, \int \, dx_{a} \, dx_{b} \, d\Phi_{n} \,
d\pXhat^{\rec} \, d\pYhat^{\rec} \, \frac{f(x_{a}) f(x_{b})}{2 \,
  x_{a} \, x_{b} \, s} \cdot \nonumber \\
& \qquad (2\pi)^{4} \, \delta( x_{a} \, \Ehat_{a} + x_{b} \, \Ehat_{b} -
(\Ehat^{\rec} + \Ehat_{\Pgt(1)} + \Ehat_{\Pgt(2)}) ) \, \delta^{3}( x_{a} \,
\bm{\hat{p}}^{a} + x_{b} \, \bm{\hat{p}}^{b} - (\bm{\hat{p}}^{\rec} + \bm{\hat{p}}^{\Pgt(1)}
+ \bm{\hat{p}}^{\Pgt(2)}) ) \cdot \nonumber \\
& \qquad \vert \mathcal{M}(\bm{\tilde{p}},m_{\PHiggs}) \vert^{2} \, W(\bm{y}|\bm{\hat{p}}) \, W_{\rec}( \pX^{\rec}, \pY^{\rec} | \pXhat^{\rec}, \pYhat^{\rec} ) \, .
\label{eq:mem_mod_hadRecoil}
\end{align}
The cross section $\sigma(m_{\PHiggs})$ needs to be replaced by
\begin{equation}
\sigma'(m_{\PHiggs}) = \sigma(m_{\PHiggs}) \, \int \, d\pXhat^{\rec} \, d\pYhat^{\rec} 
\end{equation}
in order for $\mathcal{P}(\bm{y};\pX^{\rec},\pY^{\rec}|m_{\PHiggs})$ to have the correct dimensions.
The symbol $\bm{y}$ refers to the measured momenta of the visible $\Pgt$ decay products, $\bm{p^{\vis(1)}}$ and $\bm{p^{\vis(2)}}$.
Using the relations $(\hat{E}_{a},\bm{\hat{p}}^{a}) = \frac{\sqrt{s}}{2} \, (1, 0,
0, 1)$ and $(\hat{E}_{b},\bm{\hat{p}}^{b}) = \frac{\sqrt{s}}{2} \, (1,
0, 0, -1)$
for the energies and momenta of the two colliding protons,
the integral over the $\delta$-function that ensures the conservation of energy and momentum can be expressed by:
\begin{align}
& \int \, dx_{a} \, dx_{b} \, d\pXhat^{\rec} \, d\pYhat^{\rec} \, \delta(
x_{a} \, \Ehat_{a} + x_{b} \, \Ehat_{b} - (\Ehat^{\rec} +
\Ehat_{\Pgt(1)} + \Ehat_{\Pgt(2)})) \cdot \nonumber \\
& \qquad \delta^{3}(
x_{a} \, \bm{p}^{a} + x_{b} \, \bm{\hat{p}}^{b} - (\bm{\hat{p}}^{\rec} +
\bm{\hat{p}}^{\Pgt(1)} + \bm{\hat{p}}^{\Pgt(2)})) \nonumber \\
= & \, \frac{2}{s} \, \int \,
d\pXhat^{\rec} \, d\pYhat^{\rec} \, 
\delta( \pXhat^{\rec} + \pXhat^{\Pgt(1)} + \pXhat^{\Pgt(2)} ) \,
\delta( \pYhat^{\rec} + \pYhat^{\Pgt(1)} + \pYhat^{\Pgt(2)} ) \, ,
\label{eq:delta4}
\end{align}
with:
\begin{align}
x_{a} = & \frac{1}{\sqrt{s}} \, \left( \Ehat^{\rec} + \Ehat_{\Pgt(1)} +
\Ehat_{\Pgt(2)} + (\pZhat^{\rec} + \pZhat^{\Pgt(1)} +
\pZhat^{\Pgt(2)}) \right)
\, , \nonumber \\
x_{b} = & \frac{1}{\sqrt{s}} \, \left( \Ehat^{\rec} + \Ehat_{\Pgt(1)}
  + \Ehat_{\Pgt(2)} - (\pZhat^{\rec} + \pZhat^{\Pgt(1)} +
  \pZhat^{\Pgt(2)}) \right) \, .
\label{eq:xa_and_xb}
\end{align}
The integration over $d\pXhat^{\rec}$ and $d\pYhat^{\rec}$ removes the $\delta$-functions 
$\delta( \pXhat^{\rec} + \pXhat^{\Pgt(1)} + \pXhat^{\Pgt(2)} )$ and
$\delta( \pYhat^{\rec} + \pYhat^{\Pgt(1)} + \pYhat^{\Pgt(2)} )$ in Eq.~(\ref{eq:delta4}).
Substituting Eq.~(\ref{eq:meFactorization}) into Eq.~(\ref{eq:mem_mod_hadRecoil}), we obtain:
\begin{align}
&
\mathcal{P}(\bm{p}^{\vis(1)},\bm{p}^{\vis(2)};\pX^{\rec},\pY^{\rec}|m_{\PHiggs})
= \frac{1}{\sigma(m_{\PHiggs})} \, \frac{32\pi^{4}}{s} \, \int \,
 d\Phi_{n} \, \frac{f(x_{a}) f(x_{b})}{2 \, x_{a} \, x_{b} \, s} \, 
 \vert \mathcal{M}_{\Pp\Pp \to \PHiggs \to \Pgt\Pgt}(\bm{\tilde{p}},m_{\PHiggs}) \vert^{2} \cdot \hspace{2cm} \nonumber \\
& \qquad \vert \BW^{(1)}_{\Pgt} \vert^{2} \cdot \vert \mathcal{M}^{(1)}_{\Pgt\to\cdots}(\bm{\tilde{p}}) \vert^{2} 
 \cdot \vert \BW^{(2)}_{\Pgt} \vert^{2} \cdot \vert \mathcal{M}^{(2)}_{\Pgt\to\cdots}(\bm{\tilde{p}}) \vert^{2} \cdot \nonumber \\
& \qquad W(\bm{p}^{\vis(1)}|\bm{\hat{p}}^{\vis(1)}) \, W(\bm{p}^{\vis(2)}|\bm{\hat{p}}^{\vis(2)}) \, W_{\rec}( \pX^{\rec},\pY^{\rec} | \pXhat^{\rec},\pYhat^{\rec} ) \, ,
\label{eq:mem_with_hadRecoil}
\end{align}
with $x_{a}$ and $x_{b}$ given by Eq.~(\ref{eq:xa_and_xb}) and:
\begin{equation}
\pXhat^{\rec} = -(\pXhat^{\Pgt(1)} + \pXhat^{\Pgt(2)}) \, ,
\qquad \pYhat^{\rec} = -(\pYhat^{\Pgt(1)} + \pYhat^{\Pgt(2)}) \, .
\label{eq:xpXhat_and_pYhat}
\end{equation}
The form of the TF for the hadronic recoil, $W_{\rec}( \pX^{\rec},\pY^{\rec} | \pXhat^{\rec},\pYhat^{\rec} )$, is
discussed in Section~\ref{sec:mem_TF_hadRecoil}.

In practice, the experimental resolution on the momenta of electrons, muons and also $\tauh$
is typically negligible compared to the resolution on the hadronic recoil.
In good approximation, the resolution on the hadronic recoil is equivalent
to the resolution on the vectorial sum of the momenta, in the transverse plane,
of all particles reconstructed in the event.
The latter is referred to as ``missing transverse momentum'' and denoted by $\vecMET$.
The following relations hold for the components of the $\vecMET$
vector:
\begin{equation}
\METx = - \left( \pX^{\rec} + \pX^{\vis(1)} + \pX^{\vis(2)} \right)
\, \mbox { and } \,
\METy = - \left( \pY^{\rec} + \pY^{\vis(1)} + \pY^{\vis(2)} \right) \, .
\label{eq:met}
\end{equation}
These relations are valid on reconstruction level as well as for the
true values of the momenta, \ie Eq.~(\ref{eq:met}) is valid also
in case all $\pX$ and $\pY$ are replaced by $\pXhat$ and $\pYhat$.
Approximating the resolution on the hadronic recoil by the resolution
on $\vecMET$ has the advantage that the resolutions on $\vecMET$ have
been studied in detail and published by the ATLAS as well as CMS collaborations~\cite{CMS-JME-13-003,ATLAS-CONF-2014-019}.

