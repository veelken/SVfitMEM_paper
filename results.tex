\section{Results}
\label{sec:results}

The performance of the $m_{\Pgt\Pgt}$ reconstruction is studied with
simulated samples of $\PHiggs \to \Pgt\Pgt$ and $\PZ/\Pggx \to \Pgt\Pgt$ events.
A SM $\PHiggs \to \Pgt\Pgt$ signal sample for a $\PHiggs$ boson of
mass $m_{\PHiggs} = 125$~\GeV is generated with the next-to-leading-order (NLO) program POWHEG v2~\cite{POWHEG1,POWHEG2,POWHEG3}.
We also study the $m_{\Pgt\Pgt}$ reconstruction in events containing pseudoscalar $\PHiggs$ bosons of mass $200$, $300$, $500$, $800$, $1200$, $1800$, and $2600$~\GeV.
The latter samples are generated with the LO generator PYTHIA 8.2~\cite{pythia8}.
In all the cases, the $\PHiggs$ bosons are produced via the gluon fusion process.
The $\PZ/\Pggx \to \Pgt\Pgt$ background sample is generated with the LO MadGraph program, in the version MadGraph\_aMCatNLO 2.2.2~\cite{MadGraph_aMCatNLO}.
All events are generated for proton-proton collisions at $\sqrt{s} = 13$~\TeV centre-of-mass energy.
The samples produced by MadGraph and POWHEG are generated with the NNPDF3.0 set of parton distribution functions,
while the samples produced by PYTHIA use the NNPDF2.3LO set~\cite{NNPDF1,NNPDF2,NNPDF3}.
Parton shower and hadronization processes are modelled using the generator PYTHIA with the tune CUETP8M1~\cite{PYTHIA_CUETP8M1tune_CMS}.
The latter is based on the Monash tune~\cite{PYTHIA_MonashTune}.
The decays of $\Pgt$ leptons, including polarization effects, are modelled by PYTHIA.

The experimental resolutions on the $\pT$ of $\tauh$ and on the $\pXhat^{\rec}$ and $\pYhat^{\rec}$ of the hadronic recoil 
are simulated by sampling from the TF described in
Sections~\ref{sec:mem_TF_tauToHadDecays}
and~\ref{sec:mem_TF_hadRecoil}.
The $\eta$ and $\phi$ of $\tauh$,
as well as the $\pT$, $\eta$, and $\phi$ of electrons and muons are assumed to be reconstructed perfectly.

Distributions in $m_{\Pgt\Pgt}$ are computed separately for events in which 
both $\Pgt$ leptons decay hadronically ($\tauh\tauh$), 
events in which one $\Pgt$ lepton decays hadronically and the other into a muon ($\Pgm\tauh$),
and events in which one $\Pgt$ lepton decays into a muon and the other into an electron ($\Pe\Pgm$).
The visible $\Pgt$ decay products are required to pass selection criteria on $\pT$ and $\eta$,
which are motivated by the SM $\PHiggs \to \Pgt\Pgt$ analysis performed by the CMS collaboration during LHC run $1$~\cite{HIG-13-004}.
Events in the $\tauh\tauh$ decay channel are required to contain
two $\tauh$ of $\pT > 45$~\GeV and $\vert\eta\vert < 2.1$.
Events in the $\Pgm\tauh$ channel
are required to contain a muon of $\pT > 20$~\GeV and $\vert\eta\vert < 2.1$ and a $\tauh$ of $\pT > 30$~\GeV and $\vert\eta\vert < 2.3$.
Events selected in the $\Pe\Pgm$ channel are required to contain a muon with $\vert\eta\vert < 2.1$ and an electron with $\vert\eta\vert < 2.4$.
The lepton of higher $\pT$ (either the electron or the muon) is required to satisfy the condition $\pT > 20$~\GeV,
while the lepton of lower $\pT$ is required to satisfy $\pT > 10$~\GeV.
Similar selection criteria on $\pT$ and $\eta$ of the visible $\Pgt$ decay products were applied in the $\PHiggs \to \Pgt\Pgt$
analyses performed by the ATLAS
collaboration during LHC run $1$~\cite{ATLAS_HiggsTauTau_SM,ATLAS_HiggsTauTau_MSSM}.

The $m_{\Pgt\Pgt}$ distributions reconstructed using the 
improved version of the SVfit algorithm described in this paper
are compared to the distributions in $m_{\Pgt\Pgt}$ reconstructed by the previous version of the
SVfit algorithm, described in Ref.~\cite{SVfit}, and by the ``collinear-approximation'' (CA)
method~\cite{massRecoCollinearApprox}.
The improved version of the SVfit algorithm is denoted by SVfitMEM 
and is used with and without the artificial regularization term described in Section~\ref{sec:mem_logM}.
In the latter case, we use $\kappa = 3$ for the $\Pgm\tauh$ and $\tauh\tauh$ channels and $\kappa = 2$ for the $\Pe\Pgm$ channel.
The value of $\kappa$ has been chosen such that the optimal resolution on $m_{\Pgt\Pgt}$, quantified as described below, is achieved in each channel.
We expect the optimal choice of $\kappa$ to depend on the experimental resolution on the $\pT$ of $\tauh$ and on the hadronic recoil,
with a smaller (larger) value of $\kappa$ to be optimal in case of better (worse) experimental resolution,
and we recommend to perform a reoptimization of $\kappa$ in practical applications of our algorithm.
We refer to the previous version of the SVfit algorithm as ``classic'' SVfit.
The $m_{\Pgt\Pgt}$ distributions reconstructed by the different algorithms
in $\PZ/\Pggx \to \Pgt\Pgt$ background and $\PHiggs \to \Pgt\Pgt$ signal events
of mass $125$, $200$, $300$, $500$, and $800$~\GeV
are shown in Figs.~\ref{fig:massDistributions_tautau} to~\ref{fig:massDistributions_emu}.
The axis of abscissae and the ordinate are drawn in logarithmic scale to better visualize differences in the high mass tails.

\begin{figure}
\setlength{\unitlength}{1mm}
\begin{center}
\ifx\ver\verPAPER
\begin{picture}(160,170)(0,0)
\put(-5.5, 120.0){\mbox{\includegraphics*[height=50mm]
  {plots/makeSVfitMEM_PerformancePlots_DYJets_hadhad_log.pdf}}}
\put(64.0, 120.0){\mbox{\includegraphics*[height=50mm]
  {plots/makeSVfitMEM_PerformancePlots_HiggsSUSYGluGlu125_hadhad_log.pdf}}}
\put(-5.5, 62.0){\mbox{\includegraphics*[height=50mm]
  {plots/makeSVfitMEM_PerformancePlots_HiggsSUSYGluGlu300_hadhad_log.pdf}}}
\put(64.0, 62.0){\mbox{\includegraphics*[height=50mm]
  {plots/makeSVfitMEM_PerformancePlots_HiggsSUSYGluGlu500_hadhad_log.pdf}}}
\put(-5.5, 4.0){\mbox{\includegraphics*[height=50mm]
  {plots/makeSVfitMEM_PerformancePlots_HiggsSUSYGluGlu800_hadhad_log.pdf}}}
\put(64.0, 4.0){\mbox{\includegraphics*[height=50mm]
  {plots/makeSVfitMEM_PerformancePlots_legend_hadhad.pdf}}}
\put(28.0, 116.0){\small (a)}
\put(97.5, 116.0){\small (b)}
\put(28.0, 58.0){\small (c)}
\put(97.5, 58.0){\small (d)}
\put(29.5, 0.0){\small (e)}
\fi
\ifx\ver\verPreprint
\begin{picture}(160,216)(0,0)
\put(-2.5, 152.0){\mbox{\includegraphics*[height=64mm]
  {plots/makeSVfitMEM_PerformancePlots_DYJets_hadhad_log.pdf}}}
\put(79.0, 152.0){\mbox{\includegraphics*[height=64mm]
  {plots/makeSVfitMEM_PerformancePlots_HiggsSUSYGluGlu125_hadhad_log.pdf}}}
\put(-2.5, 77.0){\mbox{\includegraphics*[height=64mm]
  {plots/makeSVfitMEM_PerformancePlots_HiggsSUSYGluGlu300_hadhad_log.pdf}}}
\put(79.0, 77.0){\mbox{\includegraphics*[height=64mm]
  {plots/makeSVfitMEM_PerformancePlots_HiggsSUSYGluGlu500_hadhad_log.pdf}}}
\put(-2.5, 2.0){\mbox{\includegraphics*[height=64mm]
  {plots/makeSVfitMEM_PerformancePlots_HiggsSUSYGluGlu800_hadhad_log.pdf}}}
\put(79.0, 2.0){\mbox{\includegraphics*[height=64mm]
  {plots/makeSVfitMEM_PerformancePlots_legend_hadhad.pdf}}}
\put(35.5, 150.0){\small (a)}
\put(117.0, 150.0){\small (b)}
\put(35.5, 75.0){\small (c)}
\put(117.0, 75.0){\small (d)}
\put(35.5, 0.0){\small (e)}
\fi
\end{picture}
\end{center}
\caption{
  Distributions of alternative mass variables in simulated $\PZ/\Pggx \to \Pgt\Pgt$ background events (a) 
  and in $\PHiggs \to \Pgt\Pgt$ signal events of different mass:
  $125$~\GeV (b), $300$~\GeV (c), $500$~\GeV (d), and $800$~\GeV (e).
  The events are selected in the $\tauh\tauh$ decay channel.
}
\label{fig:massDistributions_tautau}
\end{figure}

\begin{figure}
\setlength{\unitlength}{1mm}
\begin{center}
\ifx\ver\verPAPER
\begin{picture}(160,170)(0,0)
\put(-5.5, 120.0){\mbox{\includegraphics*[height=50mm]
  {plots/makeSVfitMEM_PerformancePlots_DYJets_muhad_log.pdf}}}
\put(64.0, 120.0){\mbox{\includegraphics*[height=50mm]
  {plots/makeSVfitMEM_PerformancePlots_HiggsSUSYGluGlu125_muhad_log.pdf}}}
\put(-5.5, 62.0){\mbox{\includegraphics*[height=50mm]
  {plots/makeSVfitMEM_PerformancePlots_HiggsSUSYGluGlu300_muhad_log.pdf}}}
\put(64.0, 62.0){\mbox{\includegraphics*[height=50mm]
  {plots/makeSVfitMEM_PerformancePlots_HiggsSUSYGluGlu500_muhad_log.pdf}}}
\put(-5.5, 4.0){\mbox{\includegraphics*[height=50mm]
  {plots/makeSVfitMEM_PerformancePlots_HiggsSUSYGluGlu800_muhad_log.pdf}}}
\put(64.0, 4.0){\mbox{\includegraphics*[height=50mm]
  {plots/makeSVfitMEM_PerformancePlots_legend_muhad.pdf}}}
\put(28.0, 116.0){\small (a)}
\put(97.5, 116.0){\small (b)}
\put(28.0, 58.0){\small (c)}
\put(97.5, 58.0){\small (d)}
\put(29.5, 0.0){\small (e)}
\fi
\ifx\ver\verPreprint
\begin{picture}(160,216)(0,0)
\put(-2.5, 152.0){\mbox{\includegraphics*[height=64mm]
  {plots/makeSVfitMEM_PerformancePlots_DYJets_muhad_log.pdf}}}
\put(79.0, 152.0){\mbox{\includegraphics*[height=64mm]
  {plots/makeSVfitMEM_PerformancePlots_HiggsSUSYGluGlu125_muhad_log.pdf}}}
\put(-2.5, 77.0){\mbox{\includegraphics*[height=64mm]
  {plots/makeSVfitMEM_PerformancePlots_HiggsSUSYGluGlu300_muhad_log.pdf}}}
\put(79.0, 77.0){\mbox{\includegraphics*[height=64mm]
  {plots/makeSVfitMEM_PerformancePlots_HiggsSUSYGluGlu500_muhad_log.pdf}}}
\put(-2.5, 2.0){\mbox{\includegraphics*[height=64mm]
  {plots/makeSVfitMEM_PerformancePlots_HiggsSUSYGluGlu800_muhad_log.pdf}}}
\put(79.0, 2.0){\mbox{\includegraphics*[height=64mm]
  {plots/makeSVfitMEM_PerformancePlots_legend_muhad.pdf}}}
\put(35.5, 150.0){\small (a)}
\put(117.0, 150.0){\small (b)}
\put(35.5, 75.0){\small (c)}
\put(117.0, 75.0){\small (d)}
\put(35.5, 0.0){\small (e)}
\fi
\end{picture}
\end{center}
\caption{
  Distributions of alternative mass variables in simulated $\PZ/\Pggx \to \Pgt\Pgt$ background events (a) 
  and in $\PHiggs \to \Pgt\Pgt$ signal events of different mass:
  $125$~\GeV (b), $300$~\GeV (c), $500$~\GeV (d), and $800$~\GeV (e).
  The events are selected in the $\Pgm\tauh$ decay channel.
}
\label{fig:massDistributions_mutau}
\end{figure}

\begin{figure}
\setlength{\unitlength}{1mm}
\begin{center}
\ifx\ver\verPAPER
\begin{picture}(160,170)(0,0)
\put(-5.5, 120.0){\mbox{\includegraphics*[height=50mm]
  {plots/makeSVfitMEM_PerformancePlots_DYJets_emu_log.pdf}}}
\put(64.0, 120.0){\mbox{\includegraphics*[height=50mm]
  {plots/makeSVfitMEM_PerformancePlots_HiggsSUSYGluGlu125_emu_log.pdf}}}
\put(-5.5, 62.0){\mbox{\includegraphics*[height=50mm]
  {plots/makeSVfitMEM_PerformancePlots_HiggsSUSYGluGlu300_emu_log.pdf}}}
\put(64.0, 62.0){\mbox{\includegraphics*[height=50mm]
  {plots/makeSVfitMEM_PerformancePlots_HiggsSUSYGluGlu500_emu_log.pdf}}}
\put(-5.5, 4.0){\mbox{\includegraphics*[height=50mm]
  {plots/makeSVfitMEM_PerformancePlots_HiggsSUSYGluGlu800_emu_log.pdf}}}
\put(64.0, 4.0){\mbox{\includegraphics*[height=50mm]
  {plots/makeSVfitMEM_PerformancePlots_legend_emu.pdf}}}
\put(28.0, 116.0){\small (a)}
\put(97.5, 116.0){\small (b)}
\put(28.0, 58.0){\small (c)}
\put(97.5, 58.0){\small (d)}
\put(29.5, 0.0){\small (e)}
\fi
\ifx\ver\verPreprint
\begin{picture}(160,216)(0,0)
\put(-2.5, 152.0){\mbox{\includegraphics*[height=64mm]
  {plots/makeSVfitMEM_PerformancePlots_DYJets_emu_log.pdf}}}
\put(79.0, 152.0){\mbox{\includegraphics*[height=64mm]
  {plots/makeSVfitMEM_PerformancePlots_HiggsSUSYGluGlu125_emu_log.pdf}}}
\put(-2.5, 77.0){\mbox{\includegraphics*[height=64mm]
  {plots/makeSVfitMEM_PerformancePlots_HiggsSUSYGluGlu300_emu_log.pdf}}}
\put(79.0, 77.0){\mbox{\includegraphics*[height=64mm]
  {plots/makeSVfitMEM_PerformancePlots_HiggsSUSYGluGlu500_emu_log.pdf}}}
\put(-2.5, 2.0){\mbox{\includegraphics*[height=64mm]
  {plots/makeSVfitMEM_PerformancePlots_HiggsSUSYGluGlu800_emu_log.pdf}}}
\put(79.0, 2.0){\mbox{\includegraphics*[height=64mm]
  {plots/makeSVfitMEM_PerformancePlots_legend_emu.pdf}}}
\put(35.5, 150.0){\small (a)}
\put(117.0, 150.0){\small (b)}
\put(35.5, 75.0){\small (c)}
\put(117.0, 75.0){\small (d)}
\put(35.5, 0.0){\small (e)}
\fi
\end{picture}
\end{center}
\caption{
  Distributions of alternative mass variables in simulated $\PZ/\Pggx \to \Pgt\Pgt$ background events (a) 
  and in $\PHiggs \to \Pgt\Pgt$ signal events of different mass:
  and $\PHiggs \to \Pgt\Pgt$ events of different mass:
  $125$~\GeV (b), $300$~\GeV (c), $500$~\GeV (d), and $800$~\GeV (e).
  The events are selected in the $\Pe\Pgm$ decay channel.
}
\label{fig:massDistributions_emu}
\end{figure}

The distributions in $m_{\Pgt\Pgt}$ reconstructed by the SVfitMEM and the classic SVfit algorithm,
as well as by the CA method, peak close to the true mass of the $\Pgt$ lepton pair,
while the distributions in $m_{\vis}$,
the mass of the visible $\Pgt$ decay products,
are shifted towards lower values.
The magnitude of the shift depends on the decay channel.
The shift is largest in the $\Pe\Pgm$ channel and smallest in the $\tauh\tauh$ channel.
This is due to the fact that the fraction $z$ of $\Pgt$ lepton energy that is carried by the visible $\Pgt$ decay products
is typically high in case of hadronic $\Pgt$ decays and is typically low in case of leptonic $\Pgt$ decays,
\cf Fig.~\ref{fig:tauDecay_z}.
The shift in peak position needs to be accounted for when comparing the resolutions of the different algorithms.
We quantify the resolution in terms of the ratio
of the standard deviation $\sigma$ to the median $\textrm{M}$
of the mass distributions.
Note that all algorithms can be trivially calibrated such that the median of each mass distribution coincides with the true mass of the $\Pgt$ lepton pair,
by scaling the output of the algorithm by a suitably chosen factor that is constant for all events reconstructed in a given channel.
Such trivial calibrations do not change the ratio $\sigma/\textrm{M}$.

The distributions in $m_{\Pgt\Pgt}$ reconstructed by the CA method
exhibits pronounced high mass tails, which reduce the sensitivity of the SM $\PHiggs \to \Pgt\Pgt$ analysis,
as, due to resolution effects, a sizeable fraction of $\PZ/\Pggx \to \Pgt\Pgt$ background events
is reconstructed near the signal region $m_{\Pgt\Pgt} \approx 125$~\GeV.
The resolution, quantified by the ratio $\sigma/\textrm{M}$,
is worse for $m_{\Pgt\Pgt}$ reconstructed by the CA method compared to $m_{\vis}$.
A further disadvantage of the CA method is that it fails to yield a physical solution for approximately half of the events,
while all other algorithms provide a physical solution for every event.
The fraction of events for which the CA method fails to find a physical solution is reflected by the normalization of the distributions.

In all three decay channels, the SVfitMEM algorithm provides a substantial improvement in the mass resolution.
Numerical values of the median $\textrm{M}$ and of the standard deviation $\sigma$ 
of the $m_{\Pgt\Pgt}$ and $m_{\vis}$ distributions in simulated $\PZ/\Pggx \to \Pgt\Pgt$ background events
and in $\PHiggs \to \Pgt\Pgt$ signal events of mass $125$, $200$, $300$, $500$, $800$, $1200$, $1800$, and $2600$~GeV
are given in Table~\ref{tab:resolutions_mVis_vs_SVfit}.
In the SM $\PHiggs \to \Pgt\Pgt$ analysis,
the SVfitMEM algorithm significantly improves the separation of the signal 
from the irreducible $\PZ/\Pggx \to \Pgt\Pgt$ background, yielding a substantial gain in analysis sensitivity.
The improvement in mass resolution achieved by the SVfitMEM algorithm
further increases in events containing $\PHiggs$ bosons of high $\pT$,
a fact which has been utilized to increase the sensitivity of the SM $\PHiggs \to \Pgt\Pgt$ analysis performed by the CMS collaboration during LHC run $1$~\cite{HIG-13-004}.
In the latter analysis, the usage of $m_{\Pgt\Pgt}$ reconstructed by the SVfitMEM algorithm
has increased the sensitivity for measuring the signal rate by $\approx 40\%$,
corresponding to a gain by about a factor of two in integrated luminosity of the analyzed dataset.
Compared to the classic SVfit algorithm, the SVfitMEM algorithm yields an improvement in $m_{\Pgt\Pgt}$ resolution by $5$--$10\%$.
The relative resolution for the SVfitMEM algorithm with and without the artificial regularization term is similar.
The motivation for adding the artificial regularization term is to avoid high mass tails in the $m_{\Pgt\Pgt}$ distribution
reconstructed in $\PZ/\Pggx \to \Pgt\Pgt$ background events,
albeit on the expense of increasing the low mass tails in the $m_{\Pgt\Pgt}$ distribution reconstructed in signal events.
The reduction of high mass tails is important not only for the SM $\PHiggs \to \Pgt\Pgt$ analysis,
but also in the context of searches for new heavy resonances,
as the signal cross section is expected to fall steeply as function of resonance mass,
such that potential high mass signals would likely be buried underneath the high mass tail of the background otherwise.

The distributions in $m_{\Pgt\Pgt}$ reconstructed by the SVfitMEM
algorithm with $\kappa = 3$ in the $\tauh\tauh$ and $\Pgm\tauh$ channels and with $\kappa = 2$ in the $\Pe\Pgm$ channel
are similar to the distributions in $m_{\Pgt\Pgt}$ reconstructed by the classic SVfit algorithm.
We conclude that the absence of a proper normalization of the likelihood function in the classic SVfit algorithm
has an effect that is similar to using an artificial regularization term of the form described in Section~\ref{sec:mem_logM} with small positive $\kappa$.

We want to add two further remarks:
First, the distributions in $m_{\Pgt\Pgt}$ reconstructed by the SVfitMEM algorithm 
exhibit the best resolution in the $\tauh\tauh$ channel and the worst resolution in the $\Pe\Pgm$ channel.
The difference in resolution between the $\tauh\tauh$, $\Pgm\tauh$, and $\Pe\Pgm$ channels
is due to the fact that the fraction $z$ of $\Pgt$ lepton energy carried by the visible $\Pgt$ decay
products is typically higher in hadronic compared to leptonic $\Pgt$ decays.
The presence, in the event, of a $\tauh$ of high $\pT$ signals the decay of a
$\PHiggs$ boson of high mass,
as in this case either the $\pT$ of the visible decay products of the
other $\Pgt$ lepton is high too,
or the event contains high $\pT$ neutrinos, the momenta of which are imbalanced in the transverse plane,
indicating the decay of a $\PHiggs$ boson of high mass  by means of large $\MET$.
Note that the imbalance in $\pX^{\miss}$ and $\pY^{\miss}$ enters the SVfitMEM algorithm via Eq.~(\ref{eq:met}).
The low mass tail of the $m_{\Pgt\Pgt}$ distribution arises from events in which the visible decay products of both $\Pgt$ leptons have low $\pT$.
If in addition the $\Pgt$ leptons are ``back-to-back'' in the transverse plane ($\Delta\phi_{\Pgt\Pgt} \approx \pi$),
which is typically the case in $\PHiggs \to \Pgt\Pgt$ events produced via the gluon fusion process,
the neutrinos produced in the two $\Pgt$ decays are emitted in opposite hemispheres, 
with the effect that their contribution to $\MET$ cancels.
Such events are indistinguishable from the decays of $\PHiggs$ bosons of low mass
and are hence assigned low $m_{\Pgt\Pgt}$ values by the SVfitMEM algorithm.

Second, we remark that the ratio of the median of the $m_{\Pgt\Pgt}$ distribution to the true mass of the $\Pgt$ lepton pair 
is closer to unity and the ratio $\sigma/\textrm{M}$ is smaller
for the $\PZ/\Pggx \to \Pgt\Pgt$ background and for low mass $\PHiggs \to \Pgt\Pgt$ signals, compared to $\PHiggs \to \Pgt\Pgt$ signals of high mass.
The reason for this behaviour are the $\pT$ cuts that are applied on the visible $\Pgt$ decay products.
The $\pT$ cuts have the effect of removing events in which both $\Pgt$ leptons have low $\pT$.
In the majority of those events,
the $\Pgt$ leptons are back-to-back in the transverse plane and hence $\MET$ is small.
The effect of the $\pT$ cuts is present in the $\tauh\tauh$, $\Pgm\tauh$, and $\Pe\Pgm$ channels and can be seen in the $m_{\vis}$ as well as in the $m_{\Pgt\Pgt}$ distributions,
except for the $\PZ/\Pggx \to \Pgt\Pgt$ background in the $\tauh\tauh$ channel.
In the latter case,
the $\pT > 45$~\GeV requirement on the $\tauh$ removes all $\PZ/\Pggx \to \Pgt\Pgt$ background in which the $\PZ$ boson has low $\pT$,
as only $\PZ/\Pggx \to \Pgt\Pgt$ events in which the $\tauh$ are boosted in $\PZ$ boson direction have a chance to satisfy the condition $\pT > 45$~\GeV.
In events containing a $\PZ$ boson of high $\pT$,
the $\Pgt$ leptons are typically not back-to-back in the transverse plane.
The mass of the visible $\Pgt$ decay products decreases proportional to the cosine of the angle between the $\Pgt$ leptons,
$m_{\vis} \approx \pT^{\vis(1)} \, \cosh\eta_{\vis(1)} \cdot \pT^{\vis(2)} \, \cosh\eta_{\vis(2)} \cdot \left( 1 - \cos\sphericalangle(\Pgt,\Pgt) \right)$
and, in events with $\PZ$ bosons of high $\pT$, may be significantly smaller than two times $45$~\GeV.

\begin{figure}
\setlength{\unitlength}{1mm}
\begin{center}
\begin{picture}(160,214)(0,0)
\put(-2.5, 150.0){\mbox{\includegraphics*[height=70mm]
  {plots/svFitPerformance_hadhad_visMass.pdf}}}
\put(80.0, 150.0){\mbox{\includegraphics*[height=70mm]
  {plots/svFitPerformance_hadhad_svFitMass.pdf}}}
\put(-2.5, 75.0){\mbox{\includegraphics*[height=70mm]
  {plots/svFitPerformance_muhad_visMass.pdf}}}
\put(80.0, 75.0){\mbox{\includegraphics*[height=70mm]
  {plots/svFitPerformance_muhad_svFitMass.pdf}}}
\put(-2.5, 0.0){\mbox{\includegraphics*[height=70mm]
  {plots/svFitPerformance_emu_visMass.pdf}}}
\put(80.0, 0.0){\mbox{\includegraphics*[height=70mm]
  {plots/svFitPerformance_emu_svFitMass.pdf}}}
\end{picture}
\end{center}
\caption{
  Distributions in $m_{\vis}$ (left) and in $m_{\Pgt\Pgt}$ reconstructed by the SVfitMEM algorithm with small positive $\kappa$ (right)
  in simulated $\PZ/\Pggx \to \Pgt\Pgt$ background events and $\PHiggs \to \Pgt\Pgt$ signal events,
  selected in the decay channels $\tauh\tauh$ ($\kappa = 3$, top), $\Pgm\tauh$ ($\kappa = 3$, centre), and $\Pe\Pgm$ ($\kappa = 2$, bottom).
  The signal events are generated for $\PHiggs$ boson masses of $m_{\PHiggs} = 125$, $200$, and $300$~\GeV. 
}
\label{fig:distributions_mVis_vs_SVfit}
\end{figure}

\begin{table}
\begin{center}
\begin{tabular}{|l|cc|cc|}
\hline
\multicolumn{5}{|c|}{$\tauh\tauh$ decay channel} \\
\hline
\hline
\multirow{2}{17mm}{Sample} & \multicolumn{2}{c|}{$m_{\vis}$} & \multicolumn{2}{c|}{$m_{\Pgt\Pgt}$ (SVfitMEM, $\kappa = 3$)} \\
\cline{2-5}
 & $\textrm{M}$~[\GeV\unskip] & $\sigma$/$\textrm{M}$ & $\textrm{M}$~[\GeV\unskip] & $\sigma$/$\textrm{M}$ \\
\hline
$\PZ/\Pggx \to \Pgt\Pgt$         &   $81.5$ & $0.200$ &  $96.5$ & $0.170$ \\
$\PHiggs \to \Pgt\Pgt$: & & & & \\
 $\quad m_{\PHiggs} = 125$~\GeV  &  $106.5$ & $0.138$ &  $128.5$ & $0.108$ \\
 $\quad m_{\PHiggs} = 200$~\GeV  &  $146.5$ & $0.170$ &  $191.5$ & $0.131$ \\
 $\quad m_{\PHiggs} = 300$~\GeV  &  $201.5$ & $0.217$ &  $278.5$ & $0.159$ \\
 $\quad m_{\PHiggs} = 500$~\GeV  &  $312.5$ & $0.261$ &  $459.5$ & $0.176$ \\
 $\quad m_{\PHiggs} = 800$~\GeV  &  $487.5$ & $0.282$ &  $732.5$ & $0.176$ \\
 $\quad m_{\PHiggs} = 1200$~\GeV &  $721.5$ & $0.295$ & $1103.5$ & $0.173$ \\
 $\quad m_{\PHiggs} = 1800$~\GeV & $1074.5$ & $0.308$ & $1626.5$ & $0.166$ \\
 $\quad m_{\PHiggs} = 2600$~\GeV & $1547.5$ & $0.311$ & $2339.5$ & $0.170$ \\
\hline
\end{tabular}

\vspace*{0.4 cm}

\begin{tabular}{|l|cc|cc|}
\hline
\multicolumn{5}{|c|}{$\Pgm\tauh$ decay channel} \\
\hline
\hline
\multirow{2}{17mm}{Sample} & \multicolumn{2}{c|}{$m_{\vis}$} & \multicolumn{2}{c|}{$m_{\Pgt\Pgt}$ (SVfitMEM, $\kappa = 3$)} \\
\cline{2-5}
 & $\textrm{M}$~[\GeV\unskip] & $\sigma$/$\textrm{M}$ & $\textrm{M}$~[\GeV\unskip] & $\sigma$/$\textrm{M}$ \\
\hline
$\PZ/\Pggx \to \Pgt\Pgt$         &   $66.5$ & $0.137$ &   $96.5$ & $0.138$ \\
$\PHiggs \to \Pgt\Pgt$: & & & & \\
 $\quad m_{\PHiggs} = 125$~\GeV  &   $80.5$ & $0.186$ &  $122.5$ & $0.155$ \\
 $\quad m_{\PHiggs} = 200$~\GeV  &  $111.5$ & $0.256$ &  $187.5$ & $0.182$ \\
 $\quad m_{\PHiggs} = 300$~\GeV  &  $151.5$ & $0.312$ &  $277.5$ & $0.203$ \\
 $\quad m_{\PHiggs} = 500$~\GeV  &  $237.5$ & $0.360$ &  $464.5$ & $0.210$ \\
 $\quad m_{\PHiggs} = 800$~\GeV  &  $360.5$ & $0.395$ &  $740.5$ & $0.207$ \\
 $\quad m_{\PHiggs} = 1200$~\GeV &  $522.5$ & $0.418$ & $1118.5$ & $0.197$ \\
 $\quad m_{\PHiggs} = 1800$~\GeV &  $771.5$ & $0.418$ & $1644.5$ & $0.190$ \\
 $\quad m_{\PHiggs} = 2600$~\GeV & $1105.5$ & $0.446$ & $2343.5$ & $0.195$ \\
\hline
\end{tabular}

\vspace*{0.4 cm}

\begin{tabular}{|l|cc|cc|}
\hline
\multicolumn{5}{|c|}{$\Pe\Pgm$ decay channel} \\
\hline
\hline
\multirow{2}{17mm}{Sample} & \multicolumn{2}{c|}{$m_{\vis}$} & \multicolumn{2}{c|}{$m_{\Pgt\Pgt}$ (SVfitMEM, $\kappa = 2$)} \\
\cline{2-5}
 & $\textrm{M}$~[\GeV\unskip] & $\sigma$/$\textrm{M}$ & $\textrm{M}$~[\GeV\unskip] & $\sigma$/$\textrm{M}$ \\
\hline
$\PZ/\Pggx \to \Pgt\Pgt$         &   $48.5$ & $0.227$ &   $95.5$ & $0.238$ \\
$\PHiggs \to \Pgt\Pgt$: & & & & \\
 $\quad m_{\PHiggs} = 125$~\GeV  &   $54.5$ & $0.283$ &  $122.5$ & $0.242$ \\
 $\quad m_{\PHiggs} = 200$~\GeV  &   $75.5$ & $0.368$ &  $191.5$ & $0.252$ \\
 $\quad m_{\PHiggs} = 300$~\GeV  &  $103.5$ & $0.425$ &  $286.5$ & $0.256$ \\
 $\quad m_{\PHiggs} = 500$~\GeV  &  $161.5$ & $0.485$ &  $481.5$ & $0.249$ \\
 $\quad m_{\PHiggs} = 800$~\GeV  &  $245.5$ & $0.518$ &  $764.5$ & $0.240$ \\
 $\quad m_{\PHiggs} = 1200$~\GeV &  $361.5$ & $0.538$ & $1156.5$ & $0.222$ \\
 $\quad m_{\PHiggs} = 1800$~\GeV &  $520.5$ & $0.570$ & $1674.5$ & $0.208$ \\
 $\quad m_{\PHiggs} = 2600$~\GeV &  $764.5$ & $0.573$ & $2376.5$ & $0.219$ \\
\hline
\end{tabular}
\end{center}
\caption{
  Median $\textrm{M}$ and standard deviation $\sigma$ of the distributions in $m_{\vis}$ 
  and in $m_{\Pgt\Pgt}$ reconstructed by the SVfitMEM algorithm
  in simulated $\PZ/\Pggx \to \Pgt\Pgt$ background and in $\PHiggs \to
  \Pgt\Pgt$ signal events selected in the decay channels $\tauh\tauh$
  (top), $\Pgm\tauh$ (centre) and $\Pe\Pgm$ (bottom).
  The signal events are generated for different $\PHiggs$ boson masses
  $m_{\PHiggs}$.
  The $\textrm{M}$ and $\sigma$ of the $m_{\vis}$ ($m_{\Pgt\Pgt}$) distributions are computed for
  events in which the reconstructed $m_{\vis}$ ($m_{\Pgt\Pgt}$) value
  is within the range $0$ to $5$ times the true mass of the $\Pgt$
  lepton pair.
  Very few events have reconstructed $m_{\vis}$ ($m_{\Pgt\Pgt}$)
  values outside of this range.
}
\label{tab:resolutions_mVis_vs_SVfit}
\end{table}


