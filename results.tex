\section{Results}
\label{sec:results}

The performance of the $m_{\Pgt\Pgt}$ reconstruction is studied using
simulated samples of $\PHiggs \to \Pgt\Pgt$ signal and $\PZ/\Pggx \to
\Pgt\Pgt$ background events.
The $\PHiggs \to \Pgt\Pgt$ signal sample is generated with the next-to-leading-order (NLO) program \textsc{POWHEG}~\cite{POWHEG1,POWHEG2,POWHEG3}.
We study the $m_{\Pgt\Pgt}$ reconstruction in SM $\PHiggs \to \Pgt\Pgt$ events with $m_{\PHiggs} = 125$~\GeV
and in events containing pseudoscalar $\PHiggs$ bosons of mass $200$, $300$, $500$, and $700$~\GeV.
In all cases, the $\PHiggs$ boson are produced by the gluon fusion process.
The $\PZ/\Pggx \to \Pgt\Pgt$ DY background sample is generated with the LO \textsc{MadGraph} program~\cite{MadGraph}.
Signal and background events are generated for proton-proton collisions at $\sqrt{s} = 13$~\TeV centre-of-mass energy.
The samples produced by \textsc{MadGraph} are generated with the \textsc{CTEQ6L1} set of PDF,
while the samples produced by \textsc{POWHEG} use \textsc{CTEQ6M}~\cite{CTEQ6}.
Parton shower and hadronization processes are modelled using the generator \textsc{PYTHIA}~\cite{pythia8} with the \emph{Z2$^{*}$} tune.
The \textsc{PYTHIA} \emph{Z2$^{*}$} tune is obtained from the \emph{Z1} tune~\cite{PYTHIA_Z1tune_CMS},
which uses the \textsc{CTEQ5L} PDF, 
whereas \emph{Z2$^{*}$} adopts \textsc{CTEQ6L}~\cite{CTEQ6}.
The decays of $\Pgt$ leptons, including polarization effects, are modelled with \textsc{TAUOLA}~\cite{tauola}.

The experimental resolutions on the $\pT$ of the visible decay products in hadronic $\Pgt$ decays
and on the hadronic recoil are simulated by sampling from the TF described in Sections~\ref{sec:mem_TF_tauToHadDecays} and~\ref{sec:mem_TF_hadRecoil}.
The $\eta$ and $\phi$ of the visible $\Pgt$ decay products as well as the $\pT$ of electrons and muons are assumed to be reconstructed perfectly.

The $m_{\Pgt\Pgt}$ distributions for the $\PHiggs$ boson signal and for the DY background
are computed separately for the decay channels 
$\Pgt\Pgt \to \textrm{hadrons} + \Pnut \, \textrm{hadrons} + \Pnut$ ($\tauh\tauh$), 
$\Pgt\Pgt \to \mununu \, \textrm{hadrons} + \Pnut$ ($\Pgm\tauh$), 
and $\Pgt\Pgt \to \mununu \, \enunu$ ($\Pe\Pgm$).
The events are required to pass selection criteria on $\pT$ and $\eta$ of the visible $\Pgt$ decay products
which are motivated by the SM $\PHiggs \to \Pgt\Pgt$ analysis performed by the CMS collaboration~\cite{HIG-13-004}.
Events selected in the $\tauh\tauh$ decay channel are required to contain
two $\tauh$ with $\pT > 45$~\GeV and $\vert\eta\vert < 2.1$,
while events selected in the $\Pgm\tauh$ channel
are required to contain one muon with $\pT > 20$~\GeV and $\vert\eta\vert < 2.1$ and one $\tauh$ with $\pT > 30$~\GeV and $\vert\eta\vert < 2.3$.
The events selected in the $\Pe\Pgm$ channel are required to contain a muon within $\vert\eta\vert < 2.1$ plus an electron within $\vert\eta\vert < 2.4$.
The lepton of higher $\pT$ is required to satisfy the condition $\pT >
20$~\GeV, while the lepton of lower $\pT$ is required to satisfy $\pT > 10$~\GeV.
Similar selection criteria on $\pT$ and $\eta$ are applied in the $\PHiggs \to \Pgt\Pgt$
analyses performed by the ATLAS
collaboration~\cite{ATLAS_HiggsTauTau_SM,ATLAS_HiggsTauTau_MSSM}.

The improved version of the \textsc{SVfit} algorithm described in this paper (\textsc{SVfitMEM})
is used with two different values of the $\kappa$ parameter described in Section~\ref{sec:mem_logM}: $0$ and $5$.
The $m_{\Pgt\Pgt}$ distributions reconstructed by the
SVfitMEM algorithm are compared to the distributions of
$m_{\Pgt\Pgt}$ reconstructed by the previous version of the
\textsc{SVfit} algorithm described in Ref.~\cite{SVfit} (classic \textsc{SVfit}) and by the ``collinear-approximation'' (CA)
method~\cite{massRecoCollinearApprox},
as well as to the distribution of $m_{\vis}$, the mass of the visible $\Pgt$ decay products.
The distributions are shown in Figs.~\ref{fig:massDistributions_tautau} to~\ref{fig:massDistributions_emu}.
The ordinate is drawn in logarithmic scale to better visualize differences in the high mass tails.

\begin{figure}
\setlength{\unitlength}{1mm}
\begin{center}
\ifx\ver\verPAPER
\begin{picture}(160,170)(0,0)
\put(-5.5, 120.0){\mbox{\includegraphics*[height=50mm]
  {plots/makeSVfitMEM_PerformancePlots_forPaper_svFitMEM_HiggsSUSYGluGlu120_xSection_log.pdf}}}
\put(64.0, 120.0){\mbox{\includegraphics*[height=50mm]
  {plots/makeSVfitMEM_PerformancePlots_forPaper_svFitMEM_HiggsSUSYGluGlu200_xSection_log.pdf}}}
\put(-5.5, 62.0){\mbox{\includegraphics*[height=50mm]
  {plots/makeSVfitMEM_PerformancePlots_forPaper_svFitMEM_HiggsSUSYGluGlu300_xSection_log.pdf}}}
\put(64.0, 62.0){\mbox{\includegraphics*[height=50mm]
  {plots/makeSVfitMEM_PerformancePlots_forPaper_svFitMEM_HiggsSUSYGluGlu500_xSection_log.pdf}}}
\put(-5.5, 4.0){\mbox{\includegraphics*[height=50mm]
  {plots/makeSVfitMEM_PerformancePlots_forPaper_svFitMEM_HiggsSUSYGluGlu700_xSection_log.pdf}}}
\put(64.0, 4.0){\mbox{\includegraphics*[height=50mm]
  {plots/makeSVfitMEM_PerformancePlots_forPaper_svFitMEM_HiggsSUSYGluGlu1000_xSection_log.pdf}}}
\put(28.0, 116.0){\small (a)}
\put(97.5, 116.0){\small (b)}
\put(28.0, 58.0){\small (c)}
\put(97.5, 58.0){\small (d)}
\put(29.5, 0.0){\small (e)}
\put(97.5, 0.0){\small (f)}
\fi
\ifx\ver\verPreprint
\begin{picture}(160,180)(0,0)
\put(0.5, 128.0){\mbox{\includegraphics*[height=54mm]
  {plots/makeSVfitMEM_PerformancePlots_forPaper_svFitMEM_HiggsSUSYGluGlu120_xSection_log.pdf}}}
\put(74.0, 128.0){\mbox{\includegraphics*[height=54mm]
  {plots/makeSVfitMEM_PerformancePlots_forPaper_svFitMEM_HiggsSUSYGluGlu200_xSection_log.pdf}}}
\put(0.5, 66.0){\mbox{\includegraphics*[height=54mm]
  {plots/makeSVfitMEM_PerformancePlots_forPaper_svFitMEM_HiggsSUSYGluGlu300_xSection_log.pdf}}}
\put(74.0, 66.0){\mbox{\includegraphics*[height=54mm]
  {plots/makeSVfitMEM_PerformancePlots_forPaper_svFitMEM_HiggsSUSYGluGlu500_xSection_log.pdf}}}
\put(0.5, 4.0){\mbox{\includegraphics*[height=54mm]
  {plots/makeSVfitMEM_PerformancePlots_forPaper_svFitMEM_HiggsSUSYGluGlu700_xSection_log.pdf}}}
\put(74.0, 4.0){\mbox{\includegraphics*[height=54mm]
  {plots/makeSVfitMEM_PerformancePlots_forPaper_svFitMEM_HiggsSUSYGluGlu1000_xSection_log.pdf}}}
\put(34.0, 124.0){\small (a)}
\put(107.5, 124.0){\small (b)}
\put(34.0, 62.0){\small (c)}
\put(107.5, 62.0){\small (d)}
\put(35.5, 0.0){\small (e)}
\put(107.5, 0.0){\small (f)}
\fi
\end{picture}
\end{center}
\caption{
  Distribution of alternative mass variables in simulated $\PZ/\Pggx
  \to \Pgt\Pgt$ (a) and $\PHiggs \to \Pgt\Pgt$ events of different mass:
  $125$~\GeV (b), $200$~\GeV (c), $300$~\GeV (d), $500$~\GeV (e), and
  $700$~\GeV (f).
  The events are selected in the $\tauh\tauh$ decay channel.
  {\textbf CV: OLD PLOTS, TO BE UPDATED !!}
}
\label{fig:massDistributions_tautau}
\end{figure}

\begin{figure}
\setlength{\unitlength}{1mm}
\begin{center}
\ifx\ver\verPAPER
\begin{picture}(160,170)(0,0)
\put(-5.5, 120.0){\mbox{\includegraphics*[height=50mm]
  {plots/makeSVfitMEM_PerformancePlots_forPaper_svFitMEM_HiggsSUSYGluGlu120_xSection_log.pdf}}}
\put(64.0, 120.0){\mbox{\includegraphics*[height=50mm]
  {plots/makeSVfitMEM_PerformancePlots_forPaper_svFitMEM_HiggsSUSYGluGlu200_xSection_log.pdf}}}
\put(-5.5, 62.0){\mbox{\includegraphics*[height=50mm]
  {plots/makeSVfitMEM_PerformancePlots_forPaper_svFitMEM_HiggsSUSYGluGlu300_xSection_log.pdf}}}
\put(64.0, 62.0){\mbox{\includegraphics*[height=50mm]
  {plots/makeSVfitMEM_PerformancePlots_forPaper_svFitMEM_HiggsSUSYGluGlu500_xSection_log.pdf}}}
\put(-5.5, 4.0){\mbox{\includegraphics*[height=50mm]
  {plots/makeSVfitMEM_PerformancePlots_forPaper_svFitMEM_HiggsSUSYGluGlu700_xSection_log.pdf}}}
\put(64.0, 4.0){\mbox{\includegraphics*[height=50mm]
  {plots/makeSVfitMEM_PerformancePlots_forPaper_svFitMEM_HiggsSUSYGluGlu1000_xSection_log.pdf}}}
\put(28.0, 116.0){\small (a)}
\put(97.5, 116.0){\small (b)}
\put(28.0, 58.0){\small (c)}
\put(97.5, 58.0){\small (d)}
\put(29.5, 0.0){\small (e)}
\put(97.5, 0.0){\small (f)}
\fi
\ifx\ver\verPreprint
\begin{picture}(160,180)(0,0)
\put(0.5, 128.0){\mbox{\includegraphics*[height=54mm]
  {plots/makeSVfitMEM_PerformancePlots_forPaper_svFitMEM_HiggsSUSYGluGlu120_xSection_log.pdf}}}
\put(74.0, 128.0){\mbox{\includegraphics*[height=54mm]
  {plots/makeSVfitMEM_PerformancePlots_forPaper_svFitMEM_HiggsSUSYGluGlu200_xSection_log.pdf}}}
\put(0.5, 66.0){\mbox{\includegraphics*[height=54mm]
  {plots/makeSVfitMEM_PerformancePlots_forPaper_svFitMEM_HiggsSUSYGluGlu300_xSection_log.pdf}}}
\put(74.0, 66.0){\mbox{\includegraphics*[height=54mm]
  {plots/makeSVfitMEM_PerformancePlots_forPaper_svFitMEM_HiggsSUSYGluGlu500_xSection_log.pdf}}}
\put(0.5, 4.0){\mbox{\includegraphics*[height=54mm]
  {plots/makeSVfitMEM_PerformancePlots_forPaper_svFitMEM_HiggsSUSYGluGlu700_xSection_log.pdf}}}
\put(74.0, 4.0){\mbox{\includegraphics*[height=54mm]
  {plots/makeSVfitMEM_PerformancePlots_forPaper_svFitMEM_HiggsSUSYGluGlu1000_xSection_log.pdf}}}
\put(34.0, 124.0){\small (a)}
\put(107.5, 124.0){\small (b)}
\put(34.0, 62.0){\small (c)}
\put(107.5, 62.0){\small (d)}
\put(35.5, 0.0){\small (e)}
\put(107.5, 0.0){\small (f)}
\fi
\end{picture}
\end{center}
\caption{
  Distribution of alternative mass variables in simulated $\PZ/\Pggx
  \to \Pgt\Pgt$ (a) and $\PHiggs \to \Pgt\Pgt$ events of different mass:
  $125$~\GeV (b), $200$~\GeV (c), $300$~\GeV (d), $500$~\GeV (e), and
  $700$~\GeV (f).
  The events are selected in the $\Pgm\tauh$ decay channel.
  {\textbf CV: OLD PLOTS, TO BE UPDATED !!}
}
\label{fig:massDistributions_mutau}
\end{figure}

\begin{figure}
\setlength{\unitlength}{1mm}
\begin{center}
\ifx\ver\verPAPER
\begin{picture}(160,170)(0,0)
\put(-5.5, 120.0){\mbox{\includegraphics*[height=50mm]
  {plots/makeSVfitMEM_PerformancePlots_forPaper_svFitMEM_HiggsSUSYGluGlu120_xSection_log.pdf}}}
\put(64.0, 120.0){\mbox{\includegraphics*[height=50mm]
  {plots/makeSVfitMEM_PerformancePlots_forPaper_svFitMEM_HiggsSUSYGluGlu200_xSection_log.pdf}}}
\put(-5.5, 62.0){\mbox{\includegraphics*[height=50mm]
  {plots/makeSVfitMEM_PerformancePlots_forPaper_svFitMEM_HiggsSUSYGluGlu300_xSection_log.pdf}}}
\put(64.0, 62.0){\mbox{\includegraphics*[height=50mm]
  {plots/makeSVfitMEM_PerformancePlots_forPaper_svFitMEM_HiggsSUSYGluGlu500_xSection_log.pdf}}}
\put(-5.5, 4.0){\mbox{\includegraphics*[height=50mm]
  {plots/makeSVfitMEM_PerformancePlots_forPaper_svFitMEM_HiggsSUSYGluGlu700_xSection_log.pdf}}}
\put(64.0, 4.0){\mbox{\includegraphics*[height=50mm]
  {plots/makeSVfitMEM_PerformancePlots_forPaper_svFitMEM_HiggsSUSYGluGlu1000_xSection_log.pdf}}}
\put(28.0, 116.0){\small (a)}
\put(97.5, 116.0){\small (b)}
\put(28.0, 58.0){\small (c)}
\put(97.5, 58.0){\small (d)}
\put(29.5, 0.0){\small (e)}
\put(97.5, 0.0){\small (f)}
\fi
\ifx\ver\verPreprint
\begin{picture}(160,180)(0,0)
\put(0.5, 128.0){\mbox{\includegraphics*[height=54mm]
  {plots/makeSVfitMEM_PerformancePlots_forPaper_svFitMEM_HiggsSUSYGluGlu120_xSection_log.pdf}}}
\put(74.0, 128.0){\mbox{\includegraphics*[height=54mm]
  {plots/makeSVfitMEM_PerformancePlots_forPaper_svFitMEM_HiggsSUSYGluGlu200_xSection_log.pdf}}}
\put(0.5, 66.0){\mbox{\includegraphics*[height=54mm]
  {plots/makeSVfitMEM_PerformancePlots_forPaper_svFitMEM_HiggsSUSYGluGlu300_xSection_log.pdf}}}
\put(74.0, 66.0){\mbox{\includegraphics*[height=54mm]
  {plots/makeSVfitMEM_PerformancePlots_forPaper_svFitMEM_HiggsSUSYGluGlu500_xSection_log.pdf}}}
\put(0.5, 4.0){\mbox{\includegraphics*[height=54mm]
  {plots/makeSVfitMEM_PerformancePlots_forPaper_svFitMEM_HiggsSUSYGluGlu700_xSection_log.pdf}}}
\put(74.0, 4.0){\mbox{\includegraphics*[height=54mm]
  {plots/makeSVfitMEM_PerformancePlots_forPaper_svFitMEM_HiggsSUSYGluGlu1000_xSection_log.pdf}}}
\put(34.0, 124.0){\small (a)}
\put(107.5, 124.0){\small (b)}
\put(34.0, 62.0){\small (c)}
\put(107.5, 62.0){\small (d)}
\put(35.5, 0.0){\small (e)}
\put(107.5, 0.0){\small (f)}
\fi
\end{picture}
\end{center}
\caption{
  Distribution of alternative mass variables in simulated $\PZ/\Pggx
  \to \Pgt\Pgt$ (a) and $\PHiggs \to \Pgt\Pgt$ events of different mass:
  $125$~\GeV (b), $200$~\GeV (c), $300$~\GeV (d), $500$~\GeV (e), and
  $700$~\GeV (f).
  The events are selected in the $\Pe\Pgm$ decay channel.
  {\textbf CV: OLD PLOTS, TO BE UPDATED !!}
}
\label{fig:massDistributions_emu}
\end{figure}

The distributions reconstructed by the \textsc{SVfitMEM} algorithm with $\kappa = 0$ are very similar to the distributions of $m_{\Pgt\Pgt}$ reconstructed by the CA method.
In both cases, there exist pronounced high mass tails, which reduce the sensitivity of the SM $\PHiggs \to \Pgt\Pgt$ analysis,
as a sizeable number of $\PZ/\Pggx \to \Pgt\Pgt$ DY background events
are reconstructed within the signal region at $m_{\Pgt\Pgt} \approx 125$~\GeV.
The main advantage of the \textsc{SVfitMEM} algorithm with $\kappa =
0$ is that it provides a physical solution for every event,
while the CA method yields physical solutions for less than $50\%$ of
the events only.
The number of events for which the CA method fails to find a physical solution is reflected in the normalization of the distribution.
The high mass tail in the $m_{\Pgt\Pgt}$ distribution reconstructed by
the \textsc{SVfitMEM} algorithm is reduced substantially in case
$\kappa = 5$ is used instead of $\kappa = 0$.
The distributions of $m_{\Pgt\Pgt}$ reconstructed by the \textsc{SVfitMEM} algorithm with $\kappa = 5$ and by the classic \textsc{SVfit} algorithm are very similar.
The effect of the arbitrary normalization used by the previous version
of the \textsc{SVfit} algorithm is about equivalent to adding 
an artificial regularization term of form $5 \,
m_{\Pgt\Pgt}^{\textrm{test}(i)}$ to the logarithm of the probability
density given by Eq.~(\ref{eq:mem_with_hadRecoil}).

The distributions of $m_{\Pgt\Pgt}$ reconstructed by the
\textsc{SVfitMEM} algorithm with $\kappa = 5$ exhibit the best
resolution in the $\tauh\tauh$ channel and the worst resolution in the $\Pe\Pgm$ channel.
The difference in resolution is due to the fact that the fraction of
$\Pgt$ lepton energy that is carried by the visible $\Pgt$ decay
products is typically high for hadronic $\Pgt$ decays
and low for leptonic $\Pgt$ decays (\cf Fig.~\ref{fig:tauDecay_z}).
The presence of a $\tauh$ of high $\pT$ indicates the decay of a $\PHiggs$ boson of high mass,
as in this case either the $\pT$ of the visible decay products is high
for the second $\Pgt$ also or the event exhibits a significant
imbalance in transverse momentum and the momentum,
indicating the presence of high $\pT$ neutrinos, which signalizes also the decay of a $\PHiggs$ boson of high mass.
Note that the imbalance in transverse momentum, $\vecMET$, enters the
\textsc{SVfitMEM} algorithm via the relation between the
$\bm{p^{vis(i)}}$ and the momentum of the hadronic recoil, $\bm{p^{\rec}}$, \cf Eq.~(\ref{eq:met}).
The low mass tail that is visible in the $m_{\Pgt\Pgt}$ distribution
reconstructed by the \textsc{SVfitMEM} algorithm in the $\Pe\Pgm$
channel arises from events in which the electron as well as the muon
both have low $\pT$.
The $\Pgt$ leptons in $\PHiggs \to \Pgt\Pgt$ events produced via the
gluon fusion process are typically separated by $\Delta\phi \approx
\pi$ in the transverse plane,
causing the neutrinos produced in the $\Pgt$ decays to be emitted into
opposite hemispheres, with the effect that their contribution to $\vecMET$ cancels.
Events featuring a low $\pT$ electron, a low $\pT$ muon and low $\MET$
are indistinguishable from the decays of $\PHiggs$ bosons of low mass
and are hence assigned low $m_{\Pgt\Pgt}$ values by the \textsc{SVfitMEM} algorithm.
The effect can be seen too in the distribution of $m_{\vis}$ 
reconstructed in the $\Pe\Pgm$ channel.

In all three decay channels, the \textsc{SVfitMEM} algorithm significantly improves the separation of the $\PHiggs \to \Pgt\Pgt$ signal 
from the irreducible $\PZ/\Pggx \to \Pgt\Pgt$ DY background, yielding a substantial gain in analysis sensitivity.
The increase in signal-to-background separation is illustrated in Fig.~\ref{fig:distributions_mVis_vs_SVfit}
that compares the signal and background distributions for $m_{\Pgt\Pgt}$ and $m_{\vis}$.
Numerical values of the means of the $m_{\Pgt\Pgt}$ and $m_{\vis}$
distributions of their root-mean-square (RMS) are given in
Table~\ref{tab:resolutions_mVis_vs_SVfit}.

\begin{figure}
\setlength{\unitlength}{1mm}
\begin{center}
\begin{picture}(160,170)(0,0)
\put(-5.5, 120.0){\mbox{\includegraphics*[height=54mm]
  {plots/svFitPerformance_aLaLorenzo_ditau_visMass.pdf}}}
\put(64.0, 120.0){\mbox{\includegraphics*[height=54mm]
  {plots/svFitPerformance_aLaLorenzo_ditau_svFitMass.pdf}}}
\put(-5.5, 62.0){\mbox{\includegraphics*[height=54mm]
  {plots/svFitPerformance_aLaLorenzo_mutau_visMass.pdf}}}
\put(64.0, 62.0){\mbox{\includegraphics*[height=54mm]
  {plots/svFitPerformance_aLaLorenzo_mutau_svFitMass.pdf}}}
\put(-5.5, 0.0){\mbox{\includegraphics*[height=54mm]
  {plots/svFitPerformance_aLaLorenzo_emu_visMass.pdf}}}
\put(64.0, 0.0){\mbox{\includegraphics*[height=54mm]
  {plots/svFitPerformance_aLaLorenzo_emu_svFitMass.pdf}}}
\end{picture}
\end{center}
\caption{
  Distribution of $m_{\vis}$ (left) and of $m_{\Pgt\Pgt}$ reconstructed by the SVfitMEM algorithm with $\kappa = 5$ (right)
  in simulated $\PZ/\Pggx \to \Pgt\Pgt$ DY background and $\PHiggs \to \Pgt\Pgt$ signal events,
  for events selected in the decay channels $\tauh\tauh$ (top), $\Pgm\tauh$ (center), and $\Pe\Pgm$ (bottom).
  The signal events are generated for $\PHiggs$ boson masses of $m_{\PHiggs} = 125$, $200$, and $300$~\GeV. 
  {\textbf CV: OLD PLOTS, TO BE UPDATED !!}
}
\label{fig:distributions_mVis_vs_SVfit}
\end{figure}

\begin{table}
\begin{center}
\begin{tabular}{|l|cc|cc|}
\hline
\multicolumn{5}{|c|}{$\tauh\tauh$ decay channel} \\
\hline
\hline
\multirow{2}{17mm}{Sample} & \multicolumn{2}{c|}{$m_{\vis}$} & \multicolumn{2}{c|}{$m_{\Pgt\Pgt}$ (SVfitMEM, $\kappa = 5$)} \\
\cline{2-5}
 & Mean & RMS/Mean & Mean & RMS/Mean \\
\hline
$\PZ/\Pggx \to \Pgt\Pgt$ & $XXX.X$~\GeV & $0.XXX$ & $XXX.X$~\GeV & $0.XXX$ \\
$\PHiggs \to \Pgt\Pgt$: & & & & \\
 $\quad m_{\PHiggs} = 120$~GeV & $XXX.X$~\GeV & $0.XXX$ & $XXX.X$~\GeV & $0.XXX$ \\
 $\quad m_{\PHiggs} = 200$~GeV & $XXX.X$~\GeV & $0.XXX$ & $XXX.X$~\GeV & $0.XXX$ \\
 $\quad m_{\PHiggs} = 300$~GeV & $XXX.X$~\GeV & $0.XXX$ & $XXX.X$~\GeV & $0.XXX$ \\
 $\quad m_{\PHiggs} = 500$~GeV & $XXX.X$~\GeV & $0.XXX$ & $XXX.X$~\GeV & $0.XXX$ \\
 $\quad m_{\PHiggs} = 700$~GeV & $XXX.X$~\GeV & $0.XXX$ & $XXX.X$~\GeV & $0.XXX$ \\
 $\quad m_{\PHiggs} = 1000$~GeV & $XXX.X$~\GeV & $0.XXX$ & $XXX.X$~\GeV & $0.XXX$ \\ 
 $\quad m_{\PHiggs} = 1500$~GeV & $XXX.X$~\GeV & $0.XXX$ & $XXX.X$~\GeV & $0.XXX$ \\ 
 $\quad m_{\PHiggs} = 2500$~GeV & $XXX.X$~\GeV & $0.XXX$ & $XXX.X$~\GeV & $0.XXX$ \\ 
\hline
\end{tabular}

\vspace*{0.4 cm}

\begin{tabular}{|l|cc|cc|}
\hline
\multicolumn{5}{|c|}{$\Pgm\tauh$ decay channel} \\
\hline
\hline
\multirow{2}{17mm}{Sample} & \multicolumn{2}{c|}{$m_{\vis}$} & \multicolumn{2}{c|}{$m_{\Pgt\Pgt}$ (SVfitMEM, $\kappa = 5$)} \\
\cline{2-5}
 & Mean & RMS/Mean & Mean & RMS/Mean \\
\hline
$\PZ/\Pggx \to \Pgt\Pgt$ & $XXX.X$~\GeV & $0.XXX$ & $XXX.X$~\GeV & $0.XXX$ \\
$\PHiggs \to \Pgt\Pgt$: & & & & \\
 $\quad m_{\PHiggs} = 120$~GeV & $XXX.X$~\GeV & $0.XXX$ & $XXX.X$~\GeV & $0.XXX$ \\
 $\quad m_{\PHiggs} = 200$~GeV & $XXX.X$~\GeV & $0.XXX$ & $XXX.X$~\GeV & $0.XXX$ \\
 $\quad m_{\PHiggs} = 300$~GeV & $XXX.X$~\GeV & $0.XXX$ & $XXX.X$~\GeV & $0.XXX$ \\
 $\quad m_{\PHiggs} = 500$~GeV & $XXX.X$~\GeV & $0.XXX$ & $XXX.X$~\GeV & $0.XXX$ \\
 $\quad m_{\PHiggs} = 700$~GeV & $XXX.X$~\GeV & $0.XXX$ & $XXX.X$~\GeV & $0.XXX$ \\
 $\quad m_{\PHiggs} = 1000$~GeV & $XXX.X$~\GeV & $0.XXX$ & $XXX.X$~\GeV & $0.XXX$ \\ 
 $\quad m_{\PHiggs} = 1500$~GeV & $XXX.X$~\GeV & $0.XXX$ & $XXX.X$~\GeV & $0.XXX$ \\ 
 $\quad m_{\PHiggs} = 2500$~GeV & $XXX.X$~\GeV & $0.XXX$ & $XXX.X$~\GeV & $0.XXX$ \\ 
\hline
\end{tabular}

\vspace*{0.4 cm}

\begin{tabular}{|l|cc|cc|}
\hline
\multicolumn{5}{|c|}{$\Pe\Pgm$ decay channel} \\
\hline
\hline
\multirow{2}{17mm}{Sample} & \multicolumn{2}{c|}{$m_{\vis}$} & \multicolumn{2}{c|}{$m_{\Pgt\Pgt}$ (SVfitMEM, $\kappa = 5$)} \\
\cline{2-5}
 & Mean & RMS/Mean & Mean & RMS/Mean \\
\hline
$\PZ/\Pggx \to \Pgt\Pgt$ & $XXX.X$~\GeV & $0.XXX$ & $XXX.X$~\GeV & $0.XXX$ \\
$\PHiggs \to \Pgt\Pgt$: & & & & \\
 $\quad m_{\PHiggs} = 120$~GeV & $XXX.X$~\GeV & $0.XXX$ & $XXX.X$~\GeV & $0.XXX$ \\
 $\quad m_{\PHiggs} = 200$~GeV & $XXX.X$~\GeV & $0.XXX$ & $XXX.X$~\GeV & $0.XXX$ \\
 $\quad m_{\PHiggs} = 300$~GeV & $XXX.X$~\GeV & $0.XXX$ & $XXX.X$~\GeV & $0.XXX$ \\
 $\quad m_{\PHiggs} = 500$~GeV & $XXX.X$~\GeV & $0.XXX$ & $XXX.X$~\GeV & $0.XXX$ \\
 $\quad m_{\PHiggs} = 700$~GeV & $XXX.X$~\GeV & $0.XXX$ & $XXX.X$~\GeV & $0.XXX$ \\
 $\quad m_{\PHiggs} = 1000$~GeV & $XXX.X$~\GeV & $0.XXX$ & $XXX.X$~\GeV & $0.XXX$ \\ 
 $\quad m_{\PHiggs} = 1500$~GeV & $XXX.X$~\GeV & $0.XXX$ & $XXX.X$~\GeV & $0.XXX$ \\ 
 $\quad m_{\PHiggs} = 2500$~GeV & $XXX.X$~\GeV & $0.XXX$ & $XXX.X$~\GeV & $0.XXX$ \\ 
\hline
\end{tabular}
\end{center}
\caption{
  Mean and root-mean-square (RMS) of the $m_{\vis}$ and $m_{\Pgt\Pgt}$ distributions
  reconstructed in $\PZ/\Pggx \to \Pgt\Pgt$ DY background events and
  in $\PHiggs \to \Pgt\Pgt$ signal events of different mass
  $m_{\PHiggs}$.
  The distributions of $m_{\Pgt\Pgt}$ are reconstructed by the
  SVfitMEM algorithm with $\kappa = 5$.
  The results for the $\tauh\tauh$ channel are given at the top, the
  results for the $\Pgm\tauh$ channel at the centre and the results
  for the $\Pe\Pgm$ channel at the bottom.
  {\textbf CV: NUMBERS TO BE UPDATED !!}
}
\label{tab:resolutions_mVis_vs_SVfit}
\end{table}


