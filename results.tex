\section{Results}
\label{sec:results}

The performance of the $m_{\Pgt\Pgt}$ reconstruction is studied using
simulated samples of $\PHiggs \to \Pgt\Pgt$ and $\PZ/\Pggx \to
\Pgt\Pgt$ events.
A SM $\PHiggs \to \Pgt\Pgt$ signal sample for a $\PHiggs$ boson of $m_{\PHiggs} = 125$~\GeV is generated with the next-to-leading-order (NLO) program POWHEG v2~\cite{POWHEG1,POWHEG2,POWHEG3}.
We also study the $m_{\Pgt\Pgt}$ reconstruction in events containing pseudoscalar $\PHiggs$ bosons of mass $200$, $300$, $500$, $800$, $1200$, $1800$, and $2600$~\GeV.
The latter samples are generated with the LO generator PYTHIA 8.2~\cite{pythia8}.
In all cases, the $\PHiggs$ bosons are produced by the gluon fusion process.
The $\PZ/\Pggx \to \Pgt\Pgt$ background sample is generated with the LO MadGraph program, in the version MadGraph\_aMCatNLO 2.2.2~\cite{MadGraph_aMCatNLO}.
All events are generated for proton-proton collisions at $\sqrt{s} = 13$~\TeV centre-of-mass energy.
The samples produced by MadGraph and POWHEG are generated with the NNPDF3.0 set of parton distribution functions,
while the samples produced by PYTHIA use the NNPDF2.3LO set~\cite{NNPDF1,NNPDF2,NNPDF3}.
Parton shower and hadronization processes are modelled using the generator PYTHIA with the tune CUETP8M1~\cite{PYTHIA_CUETP8M1tune_CMS}.
The latter is based on the Monash tune~\cite{PYTHIA_MonashTune}.
The decays of $\Pgt$ leptons, including polarization effects, are modelled by PYTHIA.

The experimental resolutions on the $\pT$ of $\tauh$ and on the $\pXhat^{\rec}$ and $\pYhat^{\rec}$ of the hadronic recoil 
are simulated by sampling from the TF described in
Sections~\ref{sec:mem_TF_tauToHadDecays}
and~\ref{sec:mem_TF_hadRecoil}.
The $\eta$ and $\phi$ of electrons, muons and $\tauh$,
as well as the $\pT$ of electrons and muons are assumed to be reconstructed perfectly.

Distributions in $m_{\Pgt\Pgt}$ are computed separately for events in which 
both $\Pgt$ leptons decay hadronically ($\tauh\tauh$), 
events in which one $\Pgt$ lepton decays hadronically and the other into a muon ($\Pgm\tauh$),
and events in which one $\Pgt$ lepton decays into a muon and the other into an electron ($\Pe\Pgm$).
The visible $\Pgt$ decay products are required to pass selection criteria on $\pT$ and $\eta$ 
that are motivated by the SM $\PHiggs \to \Pgt\Pgt$ analysis performed by the CMS collaboration during LHC run $1$~\cite{HIG-13-004}.
Events in the $\tauh\tauh$ decay channel are required to contain
two $\tauh$ with $\pT > 45$~\GeV and $\vert\eta\vert < 2.1$.
Events in the $\Pgm\tauh$ channel
are required to contain one muon with $\pT > 20$~\GeV and $\vert\eta\vert < 2.1$ and one $\tauh$ with $\pT > 30$~\GeV and $\vert\eta\vert < 2.3$.
Events selected in the $\Pe\Pgm$ channel are required to contain a muon of $\pT > 10$~\GeV and $\vert\eta\vert < 2.1$ and an electron of $\pT > 10$~\GeV and $\vert\eta\vert < 2.4$.
Either the electron or the muon is required to pass a higher $\pT$ threshold of $\pT > 20$~\GeV.
Similar selection criteria on $\pT$ and $\eta$ of the visible $\Pgt$ decay products were applied in the $\PHiggs \to \Pgt\Pgt$
analyses performed by the ATLAS
collaboration during LHC run $1$~\cite{ATLAS_HiggsTauTau_SM,ATLAS_HiggsTauTau_MSSM}.

The $m_{\Pgt\Pgt}$ distributions reconstructed using the 
improved version of the SVfit algorithm described in this paper
are compared to the distributions in $m_{\Pgt\Pgt}$ reconstructed by the previous version of the
SVfit algorithm described in Ref.~\cite{SVfit} and to the distribution of the $\PHiggs$ boson
mass reconstructed by the ``collinear-approximation'' (CA)
method~\cite{massRecoCollinearApprox}.
The improved version of the SVfit algorithm is denoted by SVfitMEM 
and is run with two values of the $\kappa$ parameter.
For the $\Pgm\tauh$ and $\tauh\tauh$ channel we use $\kappa = 0$ and $\kappa = 3$,
and for the $\Pe\Pgm$ channel we use $\kappa = 0$ and $\kappa = 2$.
The non-zero $\kappa$ value has been chosen to yield the optimal resolution on $m_{\Pgt\Pgt}$,
while the value $\kappa = 0$ is used for comparison, to demonstrate the effect of the artificial regularization term described in Section~\ref{sec:mem_logM}.
We refer to the previous version of the SVfit algorithm as ``classic'' SVfit.
The results are shown in Figs.~\ref{fig:massDistributions_tautau} to~\ref{fig:massDistributions_emu}.
The axis of abscissae and the ordinate are drawn in logarithmic scale to better visualize differences in the high mass tails.

\begin{figure}
\setlength{\unitlength}{1mm}
\begin{center}
\ifx\ver\verPAPER
\begin{picture}(160,170)(0,0)
\put(-5.5, 120.0){\mbox{\includegraphics*[height=50mm]
  {plots/makeSVfitMEM_PerformancePlots_DYJets_hadhad_log.pdf}}}
\put(64.0, 120.0){\mbox{\includegraphics*[height=50mm]
  {plots/makeSVfitMEM_PerformancePlots_HiggsSUSYGluGlu125_hadhad_log.pdf}}}
\put(-5.5, 62.0){\mbox{\includegraphics*[height=50mm]
  {plots/makeSVfitMEM_PerformancePlots_HiggsSUSYGluGlu300_hadhad_log.pdf}}}
\put(64.0, 62.0){\mbox{\includegraphics*[height=50mm]
  {plots/makeSVfitMEM_PerformancePlots_HiggsSUSYGluGlu500_hadhad_log.pdf}}}
\put(-5.5, 4.0){\mbox{\includegraphics*[height=50mm]
  {plots/makeSVfitMEM_PerformancePlots_HiggsSUSYGluGlu800_hadhad_log.pdf}}}
\put(64.0, 4.0){\mbox{\includegraphics*[height=50mm]
  {plots/makeSVfitMEM_PerformancePlots_legend_hadhad.pdf}}}
\put(28.0, 116.0){\small (a)}
\put(97.5, 116.0){\small (b)}
\put(28.0, 58.0){\small (c)}
\put(97.5, 58.0){\small (d)}
\put(29.5, 0.0){\small (e)}
\fi
\ifx\ver\verPreprint
\begin{picture}(160,216)(0,0)
\put(-2.5, 152.0){\mbox{\includegraphics*[height=64mm]
  {plots/makeSVfitMEM_PerformancePlots_DYJets_hadhad_log.pdf}}}
\put(79.0, 152.0){\mbox{\includegraphics*[height=64mm]
  {plots/makeSVfitMEM_PerformancePlots_HiggsSUSYGluGlu125_hadhad_log.pdf}}}
\put(-2.5, 77.0){\mbox{\includegraphics*[height=64mm]
  {plots/makeSVfitMEM_PerformancePlots_HiggsSUSYGluGlu300_hadhad_log.pdf}}}
\put(79.0, 77.0){\mbox{\includegraphics*[height=64mm]
  {plots/makeSVfitMEM_PerformancePlots_HiggsSUSYGluGlu500_hadhad_log.pdf}}}
\put(-2.5, 2.0){\mbox{\includegraphics*[height=64mm]
  {plots/makeSVfitMEM_PerformancePlots_HiggsSUSYGluGlu800_hadhad_log.pdf}}}
\put(79.0, 2.0){\mbox{\includegraphics*[height=64mm]
  {plots/makeSVfitMEM_PerformancePlots_legend_hadhad.pdf}}}
\put(35.5, 150.0){\small (a)}
\put(117.0, 150.0){\small (b)}
\put(35.5, 75.0){\small (c)}
\put(117.0, 75.0){\small (d)}
\put(35.5, 0.0){\small (e)}
\fi
\end{picture}
\end{center}
\caption{
  Distributions of alternative mass variables in simulated $\PZ/\Pggx \to \Pgt\Pgt$ background events (a) 
  and $\PHiggs \to \Pgt\Pgt$ signal events of different mass:
  $125$~\GeV (b), $300$~\GeV (c), $500$~\GeV (d), and $800$~\GeV (e).
  The events are selected in the $\tauh\tauh$ decay channel.
}
\label{fig:massDistributions_tautau}
\end{figure}

\begin{figure}
\setlength{\unitlength}{1mm}
\begin{center}
\ifx\ver\verPAPER
\begin{picture}(160,170)(0,0)
\put(-5.5, 120.0){\mbox{\includegraphics*[height=50mm]
  {plots/makeSVfitMEM_PerformancePlots_DYJets_muhad_log.pdf}}}
\put(64.0, 120.0){\mbox{\includegraphics*[height=50mm]
  {plots/makeSVfitMEM_PerformancePlots_HiggsSUSYGluGlu125_muhad_log.pdf}}}
\put(-5.5, 62.0){\mbox{\includegraphics*[height=50mm]
  {plots/makeSVfitMEM_PerformancePlots_HiggsSUSYGluGlu300_muhad_log.pdf}}}
\put(64.0, 62.0){\mbox{\includegraphics*[height=50mm]
  {plots/makeSVfitMEM_PerformancePlots_HiggsSUSYGluGlu500_muhad_log.pdf}}}
\put(-5.5, 4.0){\mbox{\includegraphics*[height=50mm]
  {plots/makeSVfitMEM_PerformancePlots_HiggsSUSYGluGlu800_muhad_log.pdf}}}
\put(64.0, 4.0){\mbox{\includegraphics*[height=50mm]
  {plots/makeSVfitMEM_PerformancePlots_legend_muhad.pdf}}}
\put(28.0, 116.0){\small (a)}
\put(97.5, 116.0){\small (b)}
\put(28.0, 58.0){\small (c)}
\put(97.5, 58.0){\small (d)}
\put(29.5, 0.0){\small (e)}
\fi
\ifx\ver\verPreprint
\begin{picture}(160,216)(0,0)
\put(-2.5, 152.0){\mbox{\includegraphics*[height=64mm]
  {plots/makeSVfitMEM_PerformancePlots_DYJets_muhad_log.pdf}}}
\put(79.0, 152.0){\mbox{\includegraphics*[height=64mm]
  {plots/makeSVfitMEM_PerformancePlots_HiggsSUSYGluGlu125_muhad_log.pdf}}}
\put(-2.5, 77.0){\mbox{\includegraphics*[height=64mm]
  {plots/makeSVfitMEM_PerformancePlots_HiggsSUSYGluGlu300_muhad_log.pdf}}}
\put(79.0, 77.0){\mbox{\includegraphics*[height=64mm]
  {plots/makeSVfitMEM_PerformancePlots_HiggsSUSYGluGlu500_muhad_log.pdf}}}
\put(-2.5, 2.0){\mbox{\includegraphics*[height=64mm]
  {plots/makeSVfitMEM_PerformancePlots_HiggsSUSYGluGlu800_muhad_log.pdf}}}
\put(79.0, 2.0){\mbox{\includegraphics*[height=64mm]
  {plots/makeSVfitMEM_PerformancePlots_legend_muhad.pdf}}}
\put(35.5, 150.0){\small (a)}
\put(117.0, 150.0){\small (b)}
\put(35.5, 75.0){\small (c)}
\put(117.0, 75.0){\small (d)}
\put(35.5, 0.0){\small (e)}
\fi
\end{picture}
\end{center}
\caption{
  Distributions of alternative mass variables in simulated $\PZ/\Pggx \to \Pgt\Pgt$ background events (a) 
  and $\PHiggs \to \Pgt\Pgt$ signal events of different mass:
  $125$~\GeV (b), $300$~\GeV (c), $500$~\GeV (d), and $800$~\GeV (e).
  The events are selected in the $\Pgm\tauh$ decay channel.
}
\label{fig:massDistributions_mutau}
\end{figure}

\begin{figure}
\setlength{\unitlength}{1mm}
\begin{center}
\ifx\ver\verPAPER
\begin{picture}(160,170)(0,0)
\put(-5.5, 120.0){\mbox{\includegraphics*[height=50mm]
  {plots/makeSVfitMEM_PerformancePlots_DYJets_emu_log.pdf}}}
\put(64.0, 120.0){\mbox{\includegraphics*[height=50mm]
  {plots/makeSVfitMEM_PerformancePlots_HiggsSUSYGluGlu125_emu_log.pdf}}}
\put(-5.5, 62.0){\mbox{\includegraphics*[height=50mm]
  {plots/makeSVfitMEM_PerformancePlots_HiggsSUSYGluGlu300_emu_log.pdf}}}
\put(64.0, 62.0){\mbox{\includegraphics*[height=50mm]
  {plots/makeSVfitMEM_PerformancePlots_HiggsSUSYGluGlu500_emu_log.pdf}}}
\put(-5.5, 4.0){\mbox{\includegraphics*[height=50mm]
  {plots/makeSVfitMEM_PerformancePlots_HiggsSUSYGluGlu800_emu_log.pdf}}}
\put(64.0, 4.0){\mbox{\includegraphics*[height=50mm]
  {plots/makeSVfitMEM_PerformancePlots_legend_emu.pdf}}}
\put(28.0, 116.0){\small (a)}
\put(97.5, 116.0){\small (b)}
\put(28.0, 58.0){\small (c)}
\put(97.5, 58.0){\small (d)}
\put(29.5, 0.0){\small (e)}
\fi
\ifx\ver\verPreprint
\begin{picture}(160,216)(0,0)
\put(-2.5, 152.0){\mbox{\includegraphics*[height=64mm]
  {plots/makeSVfitMEM_PerformancePlots_DYJets_emu_log.pdf}}}
\put(79.0, 152.0){\mbox{\includegraphics*[height=64mm]
  {plots/makeSVfitMEM_PerformancePlots_HiggsSUSYGluGlu125_emu_log.pdf}}}
\put(-2.5, 77.0){\mbox{\includegraphics*[height=64mm]
  {plots/makeSVfitMEM_PerformancePlots_HiggsSUSYGluGlu300_emu_log.pdf}}}
\put(79.0, 77.0){\mbox{\includegraphics*[height=64mm]
  {plots/makeSVfitMEM_PerformancePlots_HiggsSUSYGluGlu500_emu_log.pdf}}}
\put(-2.5, 2.0){\mbox{\includegraphics*[height=64mm]
  {plots/makeSVfitMEM_PerformancePlots_HiggsSUSYGluGlu800_emu_log.pdf}}}
\put(79.0, 2.0){\mbox{\includegraphics*[height=64mm]
  {plots/makeSVfitMEM_PerformancePlots_legend_emu.pdf}}}
\put(35.5, 150.0){\small (a)}
\put(117.0, 150.0){\small (b)}
\put(35.5, 75.0){\small (c)}
\put(117.0, 75.0){\small (d)}
\put(35.5, 0.0){\small (e)}
\fi
\end{picture}
\end{center}
\caption{
  Distributions of alternative mass variables in simulated $\PZ/\Pggx \to \Pgt\Pgt$ background events (a) 
  and $\PHiggs \to \Pgt\Pgt$ signal events of different mass:
  and $\PHiggs \to \Pgt\Pgt$ events of different mass:
  $125$~\GeV (b), $300$~\GeV (c), $500$~\GeV (d), and $800$~\GeV (e).
  The events are selected in the $\Pe\Pgm$ decay channel.
}
\label{fig:massDistributions_emu}
\end{figure}

The distributions reconstructed by the SVfitMEM algorithm with $\kappa = 0$ are similar to the distributions in $m_{\Pgt\Pgt}$ reconstructed by the CA method.
In both cases, pronounced high mass tails reduce the sensitivity of the SM $\PHiggs \to \Pgt\Pgt$ analysis,
as a sizeable fraction of $\PZ/\Pggx \to \Pgt\Pgt$ background events
are reconstructed near the signal region $m_{\Pgt\Pgt} \approx 125$~\GeV, due to resolution effects.
The advantage of the SVfitMEM algorithm with $\kappa =
0$ is that it provides a physical solution for every event,
while the CA method fails to yield a physical solution for about $50\%$ of the events.
The large fraction of events for which the CA method fails to find a physical solution is reflected by the normalization of the distribution.

The high mass tail in the $m_{\Pgt\Pgt}$ distribution reconstructed by
the SVfitMEM algorithm is reduced substantially by using
a non-zero $\kappa$.
The distributions in $m_{\Pgt\Pgt}$ reconstructed by the SVfitMEM
algorithm with non-zero $\kappa$ and by the classic SVfit algorithm are
very similar.
We conclude that the arbitrary normalization used by the classic SVfit
algorithm has an effect that is similar to adding 
an artificial regularization term of the form described in Section~\ref{sec:mem_logM}
to the logarithm of the probability density 
$P(\bm{p}^{\vis(1)},\bm{p}^{\vis(2)};\pX^{\rec},\pY^{\rec}|m_{\PHiggs}^{\textrm{test}(i)})$.

In all three decay channels, the SVfitMEM algorithm significantly improves the separation of the $\PHiggs \to \Pgt\Pgt$ signal 
from the irreducible $\PZ/\Pggx \to \Pgt\Pgt$ background, yielding a substantial gain in analysis sensitivity
compared to alternative mass observables.
The increase in signal-to-background separation is illustrated in
Fig.~\ref{fig:distributions_mVis_vs_SVfit},
which compares the signal and background distributions for $m_{\Pgt\Pgt}$ reconstructed by the SVfitMEM algorithm with non-zero $\kappa$ and for $m_{\vis}$,
the mass of the visible $\Pgt$ decay products.
The mass of the $\Pgt$ lepton pair reconstructed by the CA method performs worse compared to $m_{\vis}$ in terms of resolution
and has the further disadvantage of not yielding a physical solution for every event.
Numerical values of the mean and root-mean-square (RMS) values
of the $m_{\Pgt\Pgt}$ and $m_{\vis}$ distributions are given in
Table~\ref{tab:resolutions_mVis_vs_SVfit}.
Compared to $m_{\vis}$,
the SVfitMEM algorithm with non-zero $\kappa$ improves the relative resolution,
quantified by the ratio of RMS to mean, by $20$--$30\%$.
The improvement in resolution is even higher in events containing $\PHiggs$ bosons of high $\pT$,
a fact which has been utilized to further increase the sensitivity of the SM $\PHiggs \to \Pgt\Pgt$ analysis performed by the CMS collaboration during LHC run $1$~\cite{HIG-13-004}.
Compared to the classic SVfit algorithm, the SVfitMEM algorithm improves the resolution by $5$--$10\%$.

The distributions in $m_{\Pgt\Pgt}$ reconstructed by the
SVfitMEM algorithm with non-zero $\kappa$ exhibit the best
resolution in the $\tauh\tauh$ channel and the worst resolution in the $\Pe\Pgm$ channel.
The difference in resolution between the $\tauh\tauh$, $\Pgm\tauh$,
and $\Pe\Pgm$ channels
is due to the fact that the fraction $z$ of
$\Pgt$ lepton energy that is carried by the visible $\Pgt$ decay
products is typically high in case of hadronic $\Pgt$ decays
and low in case of leptonic $\Pgt$ decays, \cf Fig.~\ref{fig:tauDecay_z}.
The presence of a $\tauh$ of high $\pT$ in the event signals the decay of a
$\PHiggs$ boson of high mass.
In case the $\pT$ of the visible decay products is high
for one $\Pgt$, there are two cases to distinguish:
Either the visible decay products of the second $\Pgt$ have high $\pT$ or they have low $\pT$.
In the latter case, the event contains high $\pT$ neutrinos, the momenta of which are imbalanced in the transverse plane.
In this case, the decay of a $\PHiggs$ boson of high mass is signaled by large $\MET$.
The imbalance in $\pX^{\miss}$ and $\pY^{\miss}$ enters the
SVfitMEM algorithm via Eq.~(\ref{eq:met}).
The low mass tail of the $m_{\Pgt\Pgt}$ distribution in the $\Pe\Pgm$
channel arises from events in which the electron as well as the muon
both have low $\pT$.
The $\Pgt$ leptons in $\PHiggs \to \Pgt\Pgt$ events produced via the
gluon fusion process are typically ``back-to-back'' in the transverse plane ($\Delta\phi_{\Pgt\Pgt} \approx \pi$),
causing the neutrinos produced in the $\Pgt$ decays to be emitted in
opposite hemispheres, with the effect that their contribution to $\pX^{\miss}$ and $\pY^{\miss}$ cancels.
Events featuring a low $\pT$ electron, a low $\pT$ muon and low $\MET$
are indistinguishable from the decays of $\PHiggs$ bosons of low mass
and are hence assigned low $m_{\Pgt\Pgt}$ values by the SVfitMEM algorithm.

The distributions of $m_{\Pgt\Pgt}$ reconstructed by the SVfitMEM algorithm with non-zero $\kappa$ 
peak close to the true mass of $\Pgt$ lepton pair.
The ratio of the mean of the $m_{\Pgt\Pgt}$ distribution to the true mass is closer to unity and the ratio of RMS to mean is smaller
in case of the $\PZ/\Pggx \to \Pgt\Pgt$ background and of low mass $\PHiggs \to \Pgt\Pgt$ signals, compared to $\PHiggs \to \Pgt\Pgt$ signals of high mass.
The reason for this behaviour are the $\pT$ cuts that are applied on the visible $\Pgt$ decay products.
The $\pT$ cuts have the effect of removing events in which the $\Pgt$ leptons are back-to-back in the transverse plane,
the visible decay products of both $\Pgt$ leptons have low $\pT$ and the $\MET$ is small.
The effect is present in the $\tauh\tauh$, $\Pgm\tauh$, and $\Pe\Pgm$ channels and can be seen in the $m_{\vis}$ as well as in the $m_{\Pgt\Pgt}$ distributions,
except for the $\PZ/\Pggx \to \Pgt\Pgt$ background in the $\tauh\tauh$ channel.
The $\pT > 45$~\GeV cut that both $\tauh$ are required to pass in the $\tauh\tauh$ channel removes all $\PZ/\Pggx \to \Pgt\Pgt$ background in which the $\PZ$ boson has low $\pT$,
as only the events in which the $\tauh$ are boosted in $\PZ$ boson direction have a chance to satisfy the condition $\pT > 45$~\GeV.
As a consequence of the boost, the $\Pgt$ leptons are typically not back-to-back in the transverse plane in these events.
As the angle between the $\Pgt$ leptons decreases with increasing $\PZ$ boson $\pT$,
$m_{\vis} \approx \vert\bm{p}^{\vis(1)}\vert \cdot \vert\bm{p}^{\vis(2)}\vert \, \left( 1 - \cos\sphericalangle(\Pgt,\Pgt) \right)$
may be significantly smaller than two times $45$~\GeV in events with $\PZ$ bosons of high $\pT$.
The distribution in $\PZ$ boson $\pT$ of $\PZ/\Pggx \to \Pgt\Pgt$ background events passing the $\pT > 45$~\GeV cut
has the effect of broadening the $m_{\vis}$ distribution.
In the $\Pgm\tauh$ and $\Pe\Pgm$ channels, as well as for $\PHiggs \to \Pgt\Pgt$ signal events of mass $m_{\PHiggs} \geq 125$~\GeV in the $\tauh\tauh$ channel,
the $\pT$ cuts that are applied on the visible $\Pgt$ decay products 
are sufficiently low to reduce the ``sculpting'' effect on the $\PZ$ and $\PHiggs$ boson $\pT$ distribution to a small level 
and most $\Pgt$ leptons are back-to-back.

\begin{figure}
\setlength{\unitlength}{1mm}
\begin{center}
\begin{picture}(160,214)(0,0)
\put(-2.5, 150.0){\mbox{\includegraphics*[height=70mm]
  {plots/svFitPerformance_hadhad_visMass.pdf}}}
\put(80.0, 150.0){\mbox{\includegraphics*[height=70mm]
  {plots/svFitPerformance_hadhad_svFitMass.pdf}}}
\put(-2.5, 75.0){\mbox{\includegraphics*[height=70mm]
  {plots/svFitPerformance_muhad_visMass.pdf}}}
\put(80.0, 75.0){\mbox{\includegraphics*[height=70mm]
  {plots/svFitPerformance_muhad_svFitMass.pdf}}}
\put(-2.5, 0.0){\mbox{\includegraphics*[height=70mm]
  {plots/svFitPerformance_emu_visMass.pdf}}}
\put(80.0, 0.0){\mbox{\includegraphics*[height=70mm]
  {plots/svFitPerformance_emu_svFitMass.pdf}}}
\end{picture}
\end{center}
\caption{
  Distributions of $m_{\vis}$ (left) and of $m_{\Pgt\Pgt}$ reconstructed by the SVfitMEM algorithm with non-zero $\kappa$ (right)
  in simulated $\PZ/\Pggx \to \Pgt\Pgt$ background events and $\PHiggs \to \Pgt\Pgt$ signal events,
  selected in the decay channels $\tauh\tauh$ (top), $\Pgm\tauh$ (center), and $\Pe\Pgm$ (bottom).
  The signal events are generated for $\PHiggs$ boson masses of $m_{\PHiggs} = 125$, $200$, and $300$~\GeV. 
}
\label{fig:distributions_mVis_vs_SVfit}
\end{figure}

\begin{table}
\begin{center}
\begin{tabular}{|l|cc|cc|}
\hline
\multicolumn{5}{|c|}{$\tauh\tauh$ decay channel} \\
\hline
\hline
\multirow{2}{17mm}{Sample} & \multicolumn{2}{c|}{$m_{\vis}$} & \multicolumn{2}{c|}{$m_{\Pgt\Pgt}$ (SVfitMEM, $\kappa = 3$)} \\
\cline{2-5}
 & Mean~[\GeV] & RMS/Mean & Mean~[\GeV] & RMS/Mean \\
\hline
$\PZ/\Pggx \to \Pgt\Pgt$         &   $82.8$ & $0.197$ &  $101.1$ & $0.162$ \\
$\PHiggs \to \Pgt\Pgt$: & & & & \\
 $\quad m_{\PHiggs} = 125$~\GeV  &  $105.0$ & $0.140$ &  $130.1$ & $0.107$ \\
 $\quad m_{\PHiggs} = 200$~\GeV  &  $147.0$ & $0.169$ &  $190.2$ & $0.132$ \\
 $\quad m_{\PHiggs} = 300$~\GeV  &  $202.4$ & $0.216$ &  $273.8$ & $0.162$ \\
 $\quad m_{\PHiggs} = 500$~\GeV  &  $313.5$ & $0.260$ &  $447.6$ & $0.181$ \\
 $\quad m_{\PHiggs} = 800$~\GeV  &  $487.4$ & $0.282$ &  $712.4$ & $0.181$ \\
 $\quad m_{\PHiggs} = 1200$~\GeV &  $718.5$ & $0.297$ & $1066.8$ & $0.179$ \\
 $\quad m_{\PHiggs} = 1800$~\GeV & $1067.0$ & $0.310$ & $1577.9$ & $0.171$ \\
 $\quad m_{\PHiggs} = 2600$~\GeV & $1534.7$ & $0.314$ & $2259.3$ & $0.176$ \\
\hline
\end{tabular}

\vspace*{0.4 cm}

\begin{tabular}{|l|cc|cc|}
\hline
\multicolumn{5}{|c|}{$\Pgm\tauh$ decay channel} \\
\hline
\hline
\multirow{2}{17mm}{Sample} & \multicolumn{2}{c|}{$m_{\vis}$} & \multicolumn{2}{c|}{$m_{\Pgt\Pgt}$ (SVfitMEM, $\kappa = 3$)} \\
\cline{2-5}
 & Mean~[\GeV] & RMS/Mean & Mean~[\GeV] & RMS/Mean \\
\hline
$\PZ/\Pggx \to \Pgt\Pgt$         &   $67.2$ & $0.135$ &   $97.2$ & $0.137$ \\
$\PHiggs \to \Pgt\Pgt$: & & & & \\
 $\quad m_{\PHiggs} = 125$~\GeV  &   $80.9$ & $0.185$ &  $123.1$ & $0.154$ \\
 $\quad m_{\PHiggs} = 200$~\GeV  &  $113.1$ & $0.252$ &  $184.5$ & $0.185$ \\
 $\quad m_{\PHiggs} = 300$~\GeV  &  $156.0$ & $0.303$ &  $269.3$ & $0.209$ \\
 $\quad m_{\PHiggs} = 500$~\GeV  &  $243.8$ & $0.351$ &  $443.9$ & $0.220$ \\
 $\quad m_{\PHiggs} = 800$~\GeV  &  $372.1$ & $0.383$ &  $708.0$ & $0.216$ \\
 $\quad m_{\PHiggs} = 1200$~\GeV &  $538.7$ & $0.406$ & $1060.3$ & $0.208$ \\
 $\quad m_{\PHiggs} = 1800$~\GeV &  $796.7$ & $0.423$ & $1562.6$ & $0.200$ \\
 $\quad m_{\PHiggs} = 2600$~\GeV & $1132.7$ & $0.435$ & $2213.1$ & $0.206$ \\
\hline
\end{tabular}

\vspace*{0.4 cm}

\begin{tabular}{|l|cc|cc|}
\hline
\multicolumn{5}{|c|}{$\Pe\Pgm$ decay channel} \\
\hline
\hline
\multirow{2}{17mm}{Sample} & \multicolumn{2}{c|}{$m_{\vis}$} & \multicolumn{2}{c|}{$m_{\Pgt\Pgt}$ (SVfitMEM, $\kappa = 2$)} \\
\cline{2-5}
 & Mean~[\GeV] & RMS/Mean & Mean~[\GeV] & RMS/Mean \\
\hline
$\PZ/\Pggx \to \Pgt\Pgt$         &   $49.6$ & $0.222$ &   $98.7$ & $0.230$ \\
$\PHiggs \to \Pgt\Pgt$: & & & & \\
 $\quad m_{\PHiggs} = 125$~\GeV  &   $57.2$ & $0.270$ &  $123.5$ & $0.240$ \\
 $\quad m_{\PHiggs} = 200$~\GeV  &   $79.7$ & $0.349$ &  $188.3$ & $0.257$ \\
 $\quad m_{\PHiggs} = 300$~\GeV  &  $109.6$ & $0.401$ &  $274.7$ & $0.267$ \\
 $\quad m_{\PHiggs} = 500$~\GeV  &  $171.4$ & $0.457$ &  $454.1$ & $0.264$ \\
 $\quad m_{\PHiggs} = 800$~\GeV  &  $260.6$ & $0.488$ &  $721.4$ & $0.254$ \\
 $\quad m_{\PHiggs} = 1200$~\GeV &  $385.0$ & $0.505$ & $1079.7$ & $0.238$ \\
 $\quad m_{\PHiggs} = 1800$~\GeV &  $562.4$ & $0.527$ & $1566.4$ & $0.222$ \\
 $\quad m_{\PHiggs} = 2600$~\GeV &  $815.4$ & $0.538$ & $2190.5$ & $0.238$ \\
\hline
\end{tabular}
\end{center}
\caption{
  Mean and root-mean-square (RMS) of the $m_{\vis}$ and $m_{\Pgt\Pgt}$
  distributions
  reconstructed by the SVfitMEM algorithm
  in simulated $\PZ/\Pggx \to \Pgt\Pgt$ DY background and $\PHiggs \to
  \Pgt\Pgt$ signal events selected in the decay channels $\tauh\tauh$
  (top), $\Pgm\tauh$ (centre) and $\Pe\Pgm$ (bottom).
  The signal events are generated for different $\PHiggs$ boson masses $m_{\PHiggs}$.
}
\label{tab:resolutions_mVis_vs_SVfit}
\end{table}


