\subsection{Matrix element}
\label{sec:mem_ME}

We decompose the squared modulus of the ME, $\vert \mathcal{M}(\bm{p},m_{\PHiggs}) \vert^{2}$, for the process $\Pp\Pp \to \PHiggs \to \Pgt\Pgt$
with subsequent decay of the $\Pgt$ leptons into electrons, muons, or
hadrons into three parts:
\begin{equation}
\vert \mathcal{M}(\bm{p},m_{\PHiggs}) \vert^{2} = 
 \vert \mathcal{M}_{\Pp\Pp \to \PHiggs \to \Pgt\Pgt}(\bm{p},m_{\PHiggs}) \vert^{2} 
\cdot \vert \mathcal{M}^{(1)}_{\Pgt}(\bm{p}) \vert^{2} 
\cdot \vert \mathcal{M}^{(2)}_{\Pgt}(\bm{p}) \vert^{2} .
 \label{eq:meFactorization}
\end{equation}
The first term, $\vert \mathcal{M}_{\Pp\Pp \to \PHiggs \to
  \Pgt\Pgt}(\bm{p},m_{\PHiggs}) \vert^{2}$, represents the squared
modulus of the ME for $\PHiggs$ boson production with subsequent decay of the $\PHiggs$ boson into a pair of $\Pgt$ leptons.
This term can alternatively be computed using automatized tools such as CompHEP or MadGraph or taken from the literature.
We use the labels $^{(1)}$ and $^{(2)}$ to refer to the first and second $\Pgt$ lepton produced in the $\PHiggs$ boson decay, respectively.

\begin{figure}
\begin{center}
\includegraphics*[height=54mm]{figures/ggH_FeynmanDiagram.pdf}
\end{center}
\caption{
  Leading order Feynman diagram for $\PHiggs$ boson production in $\Pp\Pp$ collisions via the gluon fusion process.
}
\label{fig:ggH_FeynmanDiagram}
\end{figure}

We take $\vert \mathcal{M}_{\Pp\Pp \to \PHiggs \to
  \Pgt\Pgt}(\bm{p},m_{\PHiggs}) \vert^{2}$ from the literature.
More specifically, we decompose it into a product of three factors:
\begin{equation}
\vert \mathcal{M}_{\Pp\Pp \to \PHiggs \to \Pgt\Pgt} \vert^{2} =
 \vert \mathcal{M}_{\Pg\Pg \to \PHiggs} \vert^{2} 
\cdot \vert \BW_{\PHiggs} \vert^{2} 
\cdot \vert \mathcal{M}_{\PHiggs \to \Pgt\Pgt} \vert^{2} \, .
\label{eq:meHiggsProduction_and_Decay}
\end{equation}
We model the $\PHiggs$ boson production using the LO ME for the gluon fusion process $\Pg\Pg \to \PHiggs$,
which accounts for about $90\%$ of the total $\PHiggs$ boson production rate at the LHC.
The corresponding Feynman diagram is shown in Fig.~\ref{fig:ggH_FeynmanDiagram}.
The squared modulus of the ME reads~\cite{me_ggHprod}:
\begin{equation}
\vert \mathcal{M}_{\Pg\Pg \to \PHiggs} \vert^{2} = 
 \frac{\sqrt{2} \, G_{F}}{256 \, \pi^{2}} \, \alpha_{s}^{2} \, \tau^{2} \, m_{\PHiggs}^{4} \vert 1 + (1 - \tau) \, f(\tau) \vert^{2} \, ,
\label{eq:meHiggsProduction}
\end{equation}
with $\tau = 4\, \frac{m_{\Pqt}^{2}}{m_{\PHiggs}^{2}}$ and:
\begin{equation}
f(\tau) = 
\begin{cases} 
\arcsin^{2} \frac{1}{\sqrt{\tau}}  & \mbox{if } \tau \geq 1 \, , \\
-\frac{1}{4} \, \left( \log\frac{1 + \sqrt{1 - \tau}}{1 - \sqrt{1 - \tau}} - i\pi \right)^{2} & \mbox{if } \tau < 1 \, .
\end{cases}
\label{eq:meHiggsProduction_ftau}
\end{equation}
The symbol $G_{F}$ denotes the Fermi constant. Its numerical value is:
\begin{equation} 
G_{F} = 1.166 \times 10^{-5}\mbox{~GeV}^{-2} \, (\hbar \, c)^3 \, ,
\label{eq:def_G_F} 
\end{equation}
with $\hbar \, c = 0.1973$~GeV~fm~\cite{PDG}.
The squared modulus of the Breit-Wigner propagator $\vert \BW_{\PHiggs} \vert^{2}$ associates $\PHiggs$ boson production and decay.
It is given by:
\begin{equation}
\vert \BW_{\PHiggs} \vert^{2} = \frac{1}{(q_{\PHiggs}^{2} -
  m_{\PHiggs}^{2})^{2} + m_{\PHiggs}^{2}\Gamma_{\PHiggs}^{2}} \, ,
\label{eq:meHiggsBreitWigner}
\end{equation}
where $q_{\PHiggs}^{2} = (E_{\Pgt(1)} + E_{\Pgt(2)})^{2} - (\bm{p}^{\Pgt(1)} + \bm{p}^{\Pgt(2)})^{2}$ denotes the mass of the $\Pgt$ lepton pair.
We use the symbols $E_{\Pgt(1)}$ and $\bm{p}^{\Pgt(1)}$ ($E_{\Pgt(2)}$ and $\bm{p}^{\Pgt(2)}$) to refer to
the energy and momentum of the $\Pgt$ lepton of positive (negative) charge.
The squared modulus of the ME for the decay of the $\PHiggs$ boson
into $\Pgt$ leptons is given by~\cite{me_HtoTauTau}:
\begin{equation}
\vert \mathcal{M}_{\PHiggs \to \Pgt\Pgt} \vert^{2} = 
 \frac{2 \, m_{\Pgt}^{2}}{v^{2}} \, m_{\PHiggs}^{2} \left( 1 - \frac{4 \, m_{\Pgt}^{2}}{m_{\PHiggs}^{2}} \right) \, ,
\label{eq:meHiggsDecay}
\end{equation}
with $v = \frac{1}{\sqrt{2} \, G_{F}} = 246.2$~\GeV.
The squared modulus of the ME for the decay of the $\PHiggs$ boson
into $\Pgt$ leptons is related to the branching ratio $\mathcal{B}(\PHiggs \to \Pgt\Pgt)$
by~\cite{me_HtoTauTau}:
\begin{equation}
\mathcal{B}(\PHiggs \to \Pgt\Pgt) 
 = \frac{1}{16\pi \, m_{\PHiggs}} \, \sqrt{1 - \frac{4 \, m_{\Pgt}^{2}}{m_{\PHiggs}^{2}}} \, \vert \mathcal{M}_{\PHiggs \to \Pgt\Pgt} \vert^{2} \, .
\label{eq:meHiggsDecay_by_BR}
\end{equation}
For a SM $\PHiggs$ boson,
the branching ratio $\mathcal{B}(\PHiggs \to \Pgt\Pgt)$ becomes small and the total width $\Gamma_{\PHiggs}$ becomes large
once the decay into a pair of $\PW$ bosons is kinematically possible,
\ie for $m_{\PHiggs} \gtrsim 2 \, m_{\PW}$.
In theories beyond the SM, which motivate the search for heavy $\PHiggs$ bosons,
the branching ratio and total width may be very different from the SM
values, however.
In this paper, we assume $\mathcal{B}(\PHiggs \to \Pgt\Pgt) = 100\%$ and compute $\vert \mathcal{M}_{\PHiggs \to \Pgt\Pgt} \vert^{2}$ according to Eq.~(\ref{eq:meHiggsDecay_by_BR}).
Note that the value of the branching ratio for the decay $\PHiggs \to \Pgt\Pgt$ 
has no effect on value of $m_{\Pgt\Pgt}$ reconstructed by the algorithm, 
provided that $\mathcal{B}(\PHiggs \to \Pgt\Pgt)$ is accounted for in
a consistent way in the evaluation of the integral in
Eq.~(\ref{eq:mem}), including the computation of the normalization factor $1/\sigma(m_{\PHiggs})$.

The terms $\vert \mathcal{M}^{\Pgt(1)}(\bm{p}) \vert^{2}$ and $\vert
\mathcal{M}^{\Pgt(2)}(\bm{p}) \vert^{2}$ in Eq.~\ref{eq:meFactorization} model the decays of the two $\Pgt$ leptons.
We use the narrow-width approximation (NWA) and assume the $\Pgt$ leptons to be unpolarized,
effectively ignoring the correlation of spins between $\Pgt$ production and decay.
This allows us to decompose the squared moduli $\vert
\mathcal{M}^{\Pgt(i)}(\bm{p}) \vert^{2}$ for the first ($i=1$) and
second ($i=2$) $\Pgt$ lepton into the product of two factors:
\begin{equation}
\vert \mathcal{M}^{(i)}_{\Pgt} \vert^{2} = \vert \BW_{\Pgt} \vert^{2} \cdot \vert \mathcal{M}^{(i)}_{\textrm{decay}} \vert^{2} \, .
\end{equation}
The factor $\vert \BW_{\Pgt} \vert^{2}$ is equal to:
\begin{equation}
\vert \BW_{\Pgt} \vert^{2} = \frac{\pi}{m_{\Pgt}\Gamma_{\Pgt}} \,
\delta ( q_{\Pgt}^{2} - m_{\Pgt}^{2} ) \, \mbox{ with } \, 
\Gamma_{\Pgt} = \frac{\hbar}{\Delta t} =
 2.267 \cdot 10^{-12}\textrm{~\GeV} \, ,
\end{equation}
where $\Delta t = 290 \times 10^{-15}$~s denotes the lifetime of the
$\Pgt$ lepton~\cite{PDG}.
Concerning the factor $\vert \mathcal{M}^{(i)}_{\textrm{decay}}
\vert^{2}$, we take the ME for the decays $\Pgt \to \enunu$ and $\Pgt
\to \mununu$ from the literature, for the case that the $\Pgt$ leptons
are unpolarized.
We refer to these decays as ``leptonic'' $\Pgt$ decays.
Its squared modulus is given by~\cite{Barger:1987nn}:
\begin{equation}
\vert\mathcal{M}_{\Pgt \to \ellnunu} \vert^{2} = 128 \, G^{2}_{F} \,
\left( E_{\Pgt} \, E_{\APnu_{\Plepton}} - \bm{p}^{\Pgt} \cdot
  \bm{p}^{\APnu_{\Plepton}} \right) \, \left( E_{\Plepton} \,
  E_{\Pnut} - \bm{p}^{\Plepton} \cdot \bm{p}^{\Pnut} \right) \, , 
\label{eq:leptonic_tau_decays_ME}
\end{equation}
The modelling of the decays $\Pgt \to \textrm{hadrons} + \Pnut$ 
by ME is more difficult, 
due to the fact that $\Pgt$ leptons decay to a variety of hadronic
final states, and because some of the decays proceed via intermediate vector
meson resonances~\cite{PDG}.
We refer to these decays as ``hadronic'' $\Pgt$ decays.
The ME for the dominant hadronic $\Pgt$ decay modes are discussed in the literature~\cite{Bullock:1992yt,Raychaudhuri:1995kv}.
We use a simplified formalism and instead treat hadronic $\Pgt$ decays as two-body decays into a hadronic system $\tauh$ of momentum $\bm{p}^{\vis}$ and mass $m_{\vis}$ and a $\Pnut$.
The squared modulus of the ME for the decay is taken to be constant and denoted by $\vert\mathcal{M}^{\eff}_{\Pgt \to \tauhnu}\vert^{2}$.
The value of $\vert\mathcal{M}^{\eff}_{\Pgt \to \tauhnu}\vert^{2}$ is
chosen such that it reproduces the branching fraction for hadronic $\Pgt$ decays.
The following relation holds for the considered case of a two-body decay~\cite{Barger:1987nn}:
\begin{equation}
\mathcal{B}(\Pgt \to \textrm{hadrons} + \Pnut) = \frac{\Delta
  t}{\hbar} \, \frac{1}{16 \pi \, m_{\Pgt}^{3}} \cdot (m_{\Pgt}^{2} - m_{\vis}^{2}) \cdot \vert \mathcal{M}^{\textrm{eff}}_{\Pgt \to
  \tauhnu} \vert^{2} \, ,
\end{equation}
from which it follows that:
\begin{equation}
\vert \mathcal{M}^{\textrm{eff}}_{\Pgt \to \tauh\Pnut} \vert^{2} = \frac{16 \pi \, m_{\Pgt}^{3}}{m_{\Pgt}^{2} - m_{\vis}^{2}} \, \frac{\hbar}{\Delta t} \, \mathcal{B}(\Pgt \to \textrm{hadrons} + \Pnut) \, , 
\end{equation}
with $\mathcal{B}(\Pgt \to \textrm{hadrons} + \Pnut) = 0.648$~\cite{PDG}.
We have verified that the sum of all hadronic final states produced in $\Pgt$ lepton decays
is well reproduced by our simplified model.
Fig.~\ref{fig:tauDecay_z} shows the fraction of $\Pgt$ lepton energy,
in the laboratory frame, carried by the ``visible'' $\Pgt$ decay
products:
\begin{equation}
z = \frac{E_{\vis}}{E_{\Pgt}} \, .
\label{eq:def_z}
\end{equation}
We use the term ``visible'' $\Pgt$ decay products to refer to the sum
of all hadrons produced in decays of the type $\Pgt \to \textrm{hadrons} + \Pnut$ 
as well as to the electron respectively muon produced in the decays $\Pgt \to \enunu$ and $\Pgt \to \mununu$.

\begin{figure}[h]
\setlength{\unitlength}{1mm}
\begin{center}
%%\begin{picture}(150,52)(0,0)
%%\put(-5.5, 0.0){\mbox{\includegraphics*[height=48mm]
%%  {figures/makeSVfitToyMCplots_X1_m90_beforeVisPtCuts.pdf}}}
%%\put(58.0, 0.0){\mbox{\includegraphics*[height=48mm]
%%  {figures/makeSVfitToyMCplots_X2_m90_beforeVisPtCuts.pdf}}}
\begin{picture}(150,52)(0,0)
\put(-4.5, -4.0){\mbox{\includegraphics*[height=56mm]
  {figures/makeSVfitToyMCplots_X1_m90_beforeVisPtCuts.pdf}}}
\put(77.0, -4.0){\mbox{\includegraphics*[height=56mm]
  {figures/makeSVfitToyMCplots_X2_m90_beforeVisPtCuts.pdf}}}
\end{picture}
\end{center}
\caption{
  Fraction $z$ of $\Pgt$ lepton energy, in the laboratory frame, carried by the visible $\Pgt$ decay products.
  The case of $\Pgt \to \mununu$ decays is shown on the left and the case of $\Pgt \to \textrm{hadrons} + \Pnut$ decays on the right.
  Our simplified model, which treats hadronic $\Pgt$ decays as
  two-body decays into a hadronic system $\tauh$ and a $\Pnut$,
  reproduces the distribution in $z$ obtained with a detailed Monte Carlo simulation based on TAUOLA~\cite{tauola}.
}
\label{fig:tauDecay_z}
\end{figure} 
