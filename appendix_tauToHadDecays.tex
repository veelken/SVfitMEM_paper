\subsection{The decay $\Pgt \to \textrm{hadrons} + \Pnut$}
\label{sec:appendix_tauToHadDecays}

We treat hadronic $\Pgt$ decays as a two-body decay into a hadronic
system $\tauh$ plus a tau neutrino,
as explained in Section~\ref{sec:mem_ME},
and take the squared modulus of the ME to be a constant,
which we denote by $\vert\mathcal{M}^{\eff}_{\Pgt \to \tauhnu}\vert^{2}$.
We further denote the momentum of the neutrino produced in the $\Pgt$ decay by
$\bm{p}^{\inv}$ and its energy by $E_{\inv}$ (\cf Section~\ref{sec:mem_PSintegration}).
For reasons that will become clear later, we allow the neutrino to
have non-zero mass $m_{\inv}$.

The product of the squared modulus of the ME and the phase space
element $d\Phi^{(i)}_{\tauhnu}$ reads:
\begin{align}
 & \, \vert \BW_{\Pgt} \vert^{2} \cdot \vert\mathcal{M}_{\textrm{decay}}\vert^{2} \,
 d\Phi_{\tauhnu} = \vert \BW_{\Pgt} \vert^{2} \cdot \vert \mathcal{M}^{(i)}_{\textrm{decay}}
\vert^{2} \, \frac{d^{3}\bm{p}^{\vis}}{(2\pi)^{3} \, 2
   E_{\vis}} \, \frac{d^{3}\bm{p}^{\inv}}{(2\pi)^{3} \, 2 E_{\inv}} \nonumber \\
= & \, (2\pi)^{3} \, \int \, \frac{\pi}{m_{\Pgt} \, \Gamma_{\Pgt}} \,
\delta ( q_{\Pgt}^{2} - m_{\Pgt}^{2} ) \cdot \vert\mathcal{M}^{\eff}_{\Pgt \to
  \tauh\Pnut}\vert^{2} \, \delta \left( E_{\Pgt} - E_{\vis} -
  E_{\inv} \right) \, \delta^{3} \left( \bm{p}^{\Pgt} - \bm{p}^{\vis}
  - \bm{p}^{\inv} \right) \nonumber \\
& \qquad \frac{d^{3}\bm{p}^{\Pgt}}{(2\pi)^{3} \, 2 E_{\Pgt}} \, 
  \frac{d^{3}\bm{p}^{\vis}}{(2\pi)^{3} \, 2E_{\vis}} \, \frac{d^{3}\bm{p}^{\inv}}{(2\pi)^{3} \, 2 E_{\inv}} \, dq^{2}_{\Pgt} \nonumber \\
= & \frac{8\pi^{4}}{m_{\Pgt} \, \Gamma_{\Pgt}} \, \vert\mathcal{M}^{\eff}_{\Pgt \to
  \tauh\Pnut}\vert^{2} \, \delta \left( E_{\Pgt} - E_{\vis} -
  E_{\inv} \right) \, \delta^{3} \left( \bm{p}^{\Pgt} - \bm{p}^{\vis}
  - \bm{p}^{\inv} \right) \nonumber \\
& \qquad \frac{d^{3}\bm{p}^{\Pgt}}{(2\pi)^{3} \, 2 E_{\Pgt}} \, 
  \frac{d^{3}\bm{p}^{\vis}}{2 E_{\vis}} \, \frac{d^{3}\bm{p}^{\inv}}{2
    E_{\inv}} \nonumber \\
= & \, \frac{\pi}{m_{\Pgt} \, \Gamma_{\Pgt}} \, \frac{\vert\mathcal{M}^{\eff}_{\Pgt \to
  \tauh\Pnut}\vert^{2}}{(2\pi)^{6}} 
 \cdot \frac{1}{2 E_{\Pgt}(\bm{p}^{\vis}, \bm{p}^{\inv})} \, \delta
 \left( E_{\Pgt}(\bm{p}^{\vis}, \bm{p}^{\inv}) - E_{\vis} - E_{\inv}
 \right) \nonumber \\
& \qquad
  \frac{d^{3}\bm{p}_{\vis}}{2 E_{\vis}} \, \frac{\vert\bm{p}_{\inv}\vert}{2} \, dE_{\inv} \, d\cos\theta_{\inv} \, d\phi_{\inv} \, ,
\label{eq:hadTauDecaysPSint}
\end{align}
where we have used the formula for recursive phase space generation,
given by Eq.~(43.12) in Ref.~\cite{PDG}, for transforming the first line into the second
and the identity:
\begin{equation} 
d^{3}\bm{p}^{\inv} = \vert\bm{p}^{\inv}\vert^{2} \,
dp^{inv} \, d\cos\theta_{\inv} \, d\phi_{\inv} =
\vert\bm{p}^{\inv}\vert \, E_{\inv} \, dE_{\inv} \, d\cos\theta_{\inv}
\, d\phi_{\inv}
\end{equation} 
for rewriting the third line by the fourth.
The factor $\BW_{\Pgt} \vert^{2} = \frac{\pi}{m_{\Pgt} \,
  \Gamma_{\Pgt}} \, \delta ( q^{2}_{\Pgt} - m^{2}_{\Pgt} )$ removes
the integration over $dq^{2}_{\Pgt}$, enforcing the $\Pgt$ lepton
energy and momentum to be related by $E_{\Pgt} =
\sqrt{\vert\bm{p}^{\Pgt}\vert^{2} + m_{\Pgt}^{2}}$.
The symbol $E_{\Pgt}(\bm{p}^{\vis}, \bm{p}^{\inv})$
indicates that $E_{\Pgt}$ is a function of $\bm{p}^{\vis}$
and $\bm{p}^{\inv}$, as is neccessary to satisfy the $
\delta$-function $\delta^{3} ( \bm{p}^{\Pgt} - \bm{p}^{\vis} - \bm{p}^{\inv} )$.

We define $z = E_{\vis}/E_{\Pgt}$ according to Eq.~(\ref{eq:def_z}) and replace the integration over $dE_{\inv}$ by an integration over $z$.
The Jacobi factor related to this transformation is:
\begin{equation}
E_{\inv} = E_{\Pgt} - E_{\vis} = \left( 1 - z \right)
\, E_{\Pgt} = \frac{1 - z}{z} \, E_{\vis}
  \quad \Longleftrightarrow \quad dE_{\inv} = \frac{\partial E_{\inv}}{\partial z} \, dz = \frac{E_{\vis}}{z^{2}} \, dz \, .
\label{eq:hadTauDecaysJacobi}
\end{equation}

We then perform the integration over $d\cos\theta_{\inv}$.
As in Section~\ref{sec:mem_PSintegration}, we choose the coordinate system such that
$\theta_{\inv}$ is equal to the angle between the $\bm{p}^{\vis}$ and $\bm{p}^{\inv}$ vectors.
The $\delta$-function $\delta \left( E_{\Pgt}(\bm{p}^{\vis}, \bm{p}^{\inv}) - E_{\vis} - E_{\inv} \right)$ depends on $\cos\theta_{\inv}$ via:
\begin{align}
& \, E_{\Pgt}(\bm{p}^{\vis}, \bm{p}^{\inv}) 
= \sqrt{\vert\bm{p}^{\Pgt}\vert^{2} + m^{2}_{\Pgt}} = \sqrt{\left( \bm{p}^{\vis} + \bm{p}^{\inv} \right)^2 + m^{2}_{\Pgt}} \nonumber \\
& \quad = \sqrt{\vert\bm{p}^{\vis}\vert^{2} + \vert\bm{p}^{\inv}\vert^{2} + 2 \bm{p}^{\vis}
  \cdot \bm{p}^{\inv} + m^{2}_{\Pgt}} \nonumber \\
& \quad = \sqrt{\vert\bm{p}^{\vis}\vert^{2} + \vert\bm{p}^{\inv}\vert^{2} + 2 \vert\bm{p}^{\vis}\vert \, \vert\bm{p}^{\inv}\vert \, \cos\theta_{\inv} + m^{2}_{\Pgt}}.
\end{align}
The $\delta$-function argument vanishes if $E_{\Pgt}(\bm{p}^{\vis}, \bm{p}^{\inv}) - E_{\vis} - E_{\inv} = 0$. 
This yields:
\begin{equation}
\cos\theta_{\inv} 
  = \frac{E_{\vis} E_{\inv} - \frac{1}{2} \left(m^{2}_{\Pgt} - \left(
        m^{2}_{\vis} + m^{2}_{\inv} \right)
    \right)}{\vert\bm{p}^{\vis}\vert \, \vert\bm{p}^{\inv}\vert} \, .
\label{eq:hadTauDecaysCosTheta}
\end{equation}

When substituting the expressions of Eqs.~(\ref{eq:hadTauDecaysJacobi}) and~(\ref{eq:hadTauDecaysCosTheta}) into Eq.~(\ref{eq:hadTauDecaysPSint}),
we need to account for the $\delta$-function rule:
\begin{equation} 
\delta \left( g(x) \right) = \sum_{k} \frac{\delta \left( x - x_{k}
  \right)}{\vert g'(x_{k}) \vert} \, .
\label{eq:deltaFuncRule}
\end{equation}
We identify:
\begin{equation} 
g(\cos\theta_{\inv}) = \sqrt{\vert\bm{p}^{\vis}\vert^{2} + \vert\bm{p}^{\inv}\vert^{2}
  + 2 \vert\bm{p}^{\vis}\vert \, \vert\bm{p}^{\inv}\vert \,
  \cos\theta_{\inv} + m^{2}_{\Pgt}} - E_{\vis} - E_{\inv}
\end{equation}
and obtain $\vert g'(x_{0}) \vert = \vert\bm{p}^{\vis}\vert \,
\vert\bm{p}^{\inv}\vert / \left( E_{\vis} + E_{\inv} \right) = \vert\bm{p}^{\vis}\vert \,
\vert\bm{p}^{\inv}\vert / E_{\Pgt}$.

This yields:
\begin{equation}
 \vert\mathcal{M}_{\Pgt}\vert^{2} \,
 d\Phi^{(i)}_{\tauhnu} = \frac{\vert\mathcal{M}^{\eff}_{\Pgt \to
  \tauh\Pnut}\vert^{2}}{256\pi^{5}\, m_{\Pgt} \, \Gamma_{\Pgt}} \cdot 
    \frac{1}{\vert\bm{p}^{\vis}\vert \, z^{2}} \, 
    \frac{d^{3}\bm{p}^{\vis}}{2 E_{\vis}} \, dz \, d\phi_{\inv} \, .
\label{eq:hadTauDecaysResult}
\end{equation}

We define:
\begin{equation}
f_{h}\left(\bm{p}^{\vis}, m_{\vis}, \bm{p}^{\inv}\right) = 
  \frac{\vert\mathcal{M}^{\eff}_{\Pgt \to
  \tauh\Pnut}\vert^{2}}{512\pi^{6} \, \vert\bm{p}^{\vis}\vert \, z^{2}} 
\label{eq:hadTauDecays_f}
\end{equation}
to obtain:
\begin{equation}
\vert\mathcal{M}_{\Pgt}\vert^{2} \,
 d\Phi^{(i)}_{\tauhnu} = \frac{\pi}{m_{\Pgt} \, \Gamma_{\Pgt}} \,
 f_{h}(\bm{p}^{\vis}, m_{\vis}, \bm{p}^{\inv}) \, \frac{d^{3}\bm{p}^{\vis}}{2 E_{\vis}} \, dz \, d\phi_{\inv}
 \, ,
\end{equation}
which is the result that we quote in Eq.~(\ref{eq:PSint}).
