\def\verPreprint{1}
\def\verPAPER{2}
\def\ver{1}

\ifx\ver\verPreprint
\documentclass[a4paper,english,11pt]{article}
\usepackage[bindingoffset=0.5cm,left=2.5cm,right=2.5cm,top=2.5cm,bottom=2.5cm,footskip=1.0cm]{geometry}
\usepackage{lineno,hyperref,hepnames,bm,multirow,amssymb,authblk,graphicx}
\fi
\ifx\ver\verPAPER
\documentclass[1p]{elsarticle}
\usepackage{lineno,hyperref,hepnames,bm,multirow,amssymb}
\fi

\modulolinenumbers[5]

%%\journal{Journal of \LaTeX\ Templates}

%%%%%%%%%%%%%%%%%%%%%%%
%% Elsevier bibliography styles
%%%%%%%%%%%%%%%%%%%%%%%
%% To change the style, put a % in front of the second line of the current style and
%% remove the % from the second line of the style you would like to use.
%%%%%%%%%%%%%%%%%%%%%%%

%% Numbered
%\bibliographystyle{model1-num-names}

%% Numbered without titles
%\bibliographystyle{model1a-num-names}

%% Harvard
%\bibliographystyle{model2-names.bst}\biboptions{authoryear}

%% Vancouver numbered
%\usepackage{numcompress}\bibliographystyle{model3-num-names}

%% Vancouver name/year
%\usepackage{numcompress}\bibliographystyle{model4-names}\biboptions{authoryear}

%% APA style
%\bibliographystyle{model5-names}\biboptions{authoryear}

%% AMA style
%\usepackage{numcompress}\bibliographystyle{model6-num-names}

%% `Elsevier LaTeX' style
\bibliographystyle{elsarticle-num}
%%%%%%%%%%%%%%%%%%%%%%%

%%%%%%%%%%%%%%%%%%%%%%%
%% Custom latex macros
%%%%%%%%%%%%%%%%%%%%%%%

\renewcommand{\Plepton}{\ensuremath{\ell}}
\newcommand{\Phadron}{\ensuremath{\textrm{h}}}
\newcommand{\tauh}{\ensuremath{\Pgt_{\textrm{h}}}\xspace}
\newcommand{\tauhnu}{\ensuremath{\tauh \, \Pnu_{\kern-0.10em \Pgt}}\xspace}
\newcommand{\tauhnuOne}{\ensuremath{\tauh^{(1)} \, \Pnu^{(1)}_{\kern-0.10em \Pgt}}\xspace}
\newcommand{\tauhnuTwo}{\ensuremath{\tauh^{(2)} \, \Pnu^{(2)}_{\kern-0.10em \Pgt}}\xspace}
\newcommand{\ellnunu}{\ensuremath{\Plepton \, \APnu_{\kern-0.10em \Plepton} \, \Pnu_{\kern-0.10em \Pgt}}\xspace}
\newcommand{\ellnunuOne}{\ensuremath{\Plepton^{(1)} \, \APnu^{(1)}_{\kern-0.10em \Plepton} \, \Pnu^{(1)}_{\kern-0.10em \Pgt}}\xspace}
\newcommand{\ellnunuTwo}{\ensuremath{\Plepton^{(2)} \, \APnu^{(2)}_{\kern-0.10em \Plepton} \, \Pnu^{(2)}_{\kern-0.10em \Pgt}}\xspace}
\newcommand{\enunu}{\ensuremath{\Pe \, \APnu_{\kern-0.10em \Pe} \, \Pnu_{\kern-0.10em \Pgt}}\xspace}
\newcommand{\mununu}{\ensuremath{\Pgm \, \APnu_{\kern-0.10em \Pgm} \, \Pnu_{\kern-0.10em \Pgt}}\xspace}
\newcommand{\nunu}{\ensuremath{\Pnu \, \APnu}\xspace}
\newcommand{\phat}{\ensuremath{\bm{\hat{p}}}\xspace}
\newcommand{\pT}{\ensuremath{p_{\textrm{T}}}\xspace}
\newcommand{\pThat}{\ensuremath{\hat{p}_{\textrm{T}}}\xspace}
\newcommand{\etahat}{\ensuremath{\hat{\eta}}\xspace}
\newcommand{\thetahat}{\ensuremath{\hat{\theta}}\xspace}
\newcommand{\phihat}{\ensuremath{\hat{\phi}}\xspace}
\newcommand{\Ehat}{\ensuremath{\hat{E}}\xspace}
\newcommand{\ET}{\ensuremath{E_{\textrm{T}}}\xspace}
\newcommand{\EThat}{\ensuremath{\hat{E}_{\textrm{T}}}\xspace}
\newcommand{\pX}{\ensuremath{p_{\textrm{x}}}\xspace}
\newcommand{\pXhat}{\ensuremath{\hat{p}_{\textrm{x}}}\xspace}
\newcommand{\pY}{\ensuremath{p_{\textrm{y}}}\xspace}
\newcommand{\pYhat}{\ensuremath{\hat{p}_{\textrm{y}}}\xspace}
\newcommand{\pZ}{\ensuremath{p_{\textrm{z}}}\xspace}
\newcommand{\pZhat}{\ensuremath{\hat{p}_{\textrm{z}}}\xspace}
\newcommand{\uX}{\ensuremath{u_{\textrm{x}}}\xspace}
\newcommand{\uY}{\ensuremath{u_{\textrm{y}}}\xspace}
\newcommand{\vX}{\ensuremath{v_{\textrm{x}}}\xspace}
\newcommand{\vY}{\ensuremath{v_{\textrm{y}}}\xspace}
\newcommand{\MET}{\ensuremath{p_{\textrm{T}}^{\textrm{\kern0.10em miss}}}\xspace}
\newcommand{\vecMET}{\ensuremath{\bm{p}_{\textrm{T}}^{\textrm{\kern0.10em miss}}}\xspace}
\newcommand{\METx}{\ensuremath{p_{\textrm{x}}^{\textrm{\kern0.10em miss}}}\xspace}
\newcommand{\METy}{\ensuremath{p_{\textrm{y}}^{\textrm{\kern0.10em miss}}}\xspace}
\newcommand{\MeV}{\ensuremath{\textrm{MeV}}\xspace}
\newcommand{\GeV}{\ensuremath{\textrm{GeV}}\xspace}
\newcommand{\TeV}{\ensuremath{\textrm{TeV}}\xspace}
\newcommand{\rec}{\ensuremath{\textrm{rec}}}
\newcommand{\true}{\ensuremath{\textrm{true}}}
\newcommand{\vis}{\ensuremath{\textrm{vis}}}
\newcommand{\inv}{\ensuremath{\textrm{inv}}}
\newcommand{\eff}{\ensuremath{\textrm{eff}}}
\newcommand{\T}{\ensuremath{\textrm{T}}}
\newcommand{\cf}{cf.\xspace}
\newcommand{\ie}{i.e.\xspace}
\newcommand{\BW}{\ensuremath{\textrm{BW}}}
%%%%%%%%%%%%%%%%%%%%%%%

\begin{document}

\ifx\ver\verPAPER
\begin{frontmatter}
\fi

\title{Reconstruction of the Higgs mass in events with Higgs bosons
  decaying into a pair of $\Pgt$ leptons using matrix element techniques}

%% Group authors per affiliation:

\ifx\ver\verPreprint
\author[1]{Lorenzo Bianchini}
\author[2]{Betty Calpas}
\author[3]{John Conway}
\author[2]{Andrew Fowlie}
\author[4]{Luca Marzola}
\author[2]{Christian Veelken}
\affil[1]{Institute for Particle Physics, ETH Zurich, 8093 Zurich, Switzerland}
\affil[2]{National Institute for Chemical Physics and Biophysics, 10143 Tallinn, Estonia}
\affil[3]{Department of Physics, University of California, Davis, CA 95616}
\affil[4]{Institute of Physics, University of Tartu, 51014 Tartu, Estonia}
\fi
\ifx\ver\verPAPER
\author[eth]{Lorenzo Bianchini}
\ead{lorenzo.bianchini@cern.ch}
\author[tallinn]{Betty Calpas}
\ead{betty.calpas@cern.ch}
\author[ucd]{John Conway}
\ead{conway@physics.ucdavis.edu}
\author[tallinn]{Andrew Fowlie}
\ead{andrew.fowlie@kbfi.ee}
\author[tartu]{Luca Marzola}
\ead{luca.marzola@ut.ee}
\author[tallinn]{Christian Veelken}
\ead{christian.veelken@cern.ch}
\address[eth]{Institute for Particle Physics, ETH Zurich, 8093 Zurich, Switzerland}
\address[ucd]{Department of Physics, University of California, Davis, CA 95616}
\address[tallinn]{National Institute for Chemical Physics and Biophysics, 10143 Tallinn, Estonia}
\address[tartu]{Institute of Physics, University of Tartu, 51014 Tartu, Estonia}
\fi

\ifx\ver\verPreprint
\maketitle
\fi

\begin{abstract}
An algorithm for reconstruction of the Higgs mass in events with Higgs
bosons decaying into a pair of $\Pgt$ leptons is presented.
The algorithm is based on matrix element (ME) techniques and achieves
a relative resolution on the Higgs boson mass of typically $15$--$20\%$.
A previous version of the algorithm has been used in analyses of Higgs
boson production performed by the CMS collaboration during LHC run
$1$.
The algorithm is described in detail and its performance on simulated
events is assessed.
The development of techniques to handle $\Pgt$ decays in the ME
formalism represents an important result of this paper.
\end{abstract}

\ifx\ver\verPAPER
\end{frontmatter}
\fi

\clearpage

\linenumbers

\let\clearpage\relax
\section{Introduction}
\label{sec:introduction}

A new boson of mass $125$~\GeV has been observed by the ATLAS and CMS collaborations~\cite{Higgs-Discovery_CMS,Higgs-Discovery_ATLAS}.
The properties of the new particle are compatible with the predictions for the Standard Model (SM) 
Higgs ($\PHiggs$) boson~\cite{Englert:1964et,Higgs:1964ia,Higgs:1964pj,Guralnik:1964eu,Higgs:1966ev,Kibble:1967sv}
within the present experimental uncertainties~\cite{HIG-14-014,Chatrchyan:2014tja,Khachatryan:2014iha,HIG-14-009}.
The observation of its decay into a pair of $\Pgt$ leptons, at a rate that is compatible with the SM expectation, has been reported recently~\cite{HIG-13-004,Aad:2015vsa,HIG-15-002}.
The decay into a pair of $\Pgt$ leptons allows for the most precise measurement of the direct coupling of the SM $\PHiggs$ boson to fermions.
Decays of heavy resonances into $\Pgt$ lepton pairs furthermore provide high sensitivity to search for models with an extended Higgs sector,
constituting an important experimental signature at the LHC.

The sensitivity of the SM $\PHiggs \to \Pgt\Pgt$ analysis critically depends on
the capability to distinguish the signal from a large irreducible background, arising from $\PZ/\Pggx \to \Pgt\Pgt$ Drell--Yan (DY) production.
An important handle to separate the signal from the background is the mass of the $\Pgt$ lepton pair, which we denote by $m_{\Pgt\Pgt}$.
The signal is expected to show up as a small bump on the high mass tail of the $m_{\Pgt\Pgt}$ distribution of the background
(see e.g. Figs. 8, 9, and 11 of Ref.~\cite{HIG-13-004}).

The separation of the signal from the background improves if the mass distribution for the signal is narrow.
The SM predicts the total width of the $\PHiggs$ boson to be $\approx 4$~\MeV.
Present experimental upper limits on the total width amount to $\approx 10$ times the SM value~\cite{HIG-14-002,Aad:2015xua}.
These limits have been obtained by comparing the rates for off-shell versus on-shell $\PHiggs$ boson production and depend on certain assumptions.
Direct, model independent, upper limits on the total $\PHiggs$ width, 
obtained by analyzing the mass spectra in $\PHiggs \to \PZ\PZ \to 4 \Plepton$ ($\Plepton = \Pe, \Pgm$) and $\PHiggs \to \Pgamma\Pgamma$ events, are $\approx 1$~\GeV.
In contrast, the width of the $m_{\Pgt\Pgt}$ distribution reconstructed in SM $\PHiggs \to \Pgt\Pgt$ events typically amounts to $\approx 20$~\GeV 
and is dominated purely by the experimental resolution.

Different methods for the reconstruction of $m_{\Pgt\Pgt}$ 
have been discussed in the literature~\cite{massRecoCollinearApprox,neutrinoRecByVertexInfo,MMC,Barr:2011he,Gripaios:2012th}. 
The SVfit algorithm~\cite{SVfit} has been used to reconstruct the $\PHiggs$ boson mass in the SM $\PHiggs \to \Pgt\Pgt$ analysis
as well as in searches for further $\PHiggs$ bosons predicted by models beyond the SM 
performed by the CMS collaboration during LHC Run $1$~\cite{HIG-10-002,HIG-11-029,HIG-13-004,HIG-13-021,HIG-14-029,HIG-14-033,HIG-14-034,HIG-15-001,HIG-15-013}.
Compared to alternative mass variables,
the usage of the SVfit algorithm has improved the sensitivity of the SM $\PHiggs \to \Pgt\Pgt$ analysis for measuring the signal rate by $\approx 40\%$~\cite{HIG-13-004}.
The improvement in sensitivity corresponds to a gain by about a factor of two in integrated luminosity of the analysed dataset.

In this paper we report on the development of an improved version of the SVfit algorithm.
Two variants of the improved algorithm have been implemented.
The first variant allows to reconstruct the mass $m_{\Pgt\Pgt}$ of the $\Pgt$ lepton pair.
It has been developed within the paradigm of the matrix element (ME) method~\cite{Kondo:1988yd,Kondo:1991dw}.
Whereas the algorithm described in Ref.~\cite{SVfit} uses a likelihood function of arbitrary normalization,
the improved algorithm is based on a proper normalization within the formalism of the ME method.
The second variant of the improved algorithm uses a likelihood function of arbitrary normalization.
The algorithm allows for the reconstruction of not only the mass $m_{\Pgt\Pgt}$ of the $\Pgt$ lepton pair,
but of any kinematic function of the two $\Pgt$ leptons,
including the $\pT$, $\eta$, $\phi$, and transverse mass of the $\Pgt$ lepton pair.
A further improvement concerns the extension of the algorithm to account for the experimental resolution 
on the reconstruction of hadrons that are produced in the $\Pgt$ decays.

The development of the formalism to handle $\Pgt$ decays in the ME method constitutes an important result of this paper,
which has not been discussed in the literature so far.
The formalism described in this paper allows one to extend the ME generated by automatized tools such as CompHEP~\cite{CompHEP1,CompHEP2} or MadGraph~\cite{MadGraph},
which treat $\Pgt$ leptons as stable particles, by the capability to handle $\Pgt$ decays.

The paper is organized as follows. 
In Section~\ref{sec:mem} we describe the first variant of the improved algorithm, 
in particular the formalism that we developed to handle $\Pgt$ decays in the ME method
and our treatment of the experimental resolution on the reconstruction of hadrons that are produced in the $\Pgt$ decays.
The second variant of the algorithm is presented in Section~\ref{sec:classicSVfit}.
The performance of both variants of the improved algorithm in terms of achieved $m_{\Pgt\Pgt}$ resolution 
is compared to the previous version of the SVfit algorithm, used during LHC Run $1$, 
and to selected alternative mass observables in Section~\ref{sec:performance}.
The results are discussed in Section~\ref{sec:discussion}.
The paper concludes with a summary in Section~\ref{sec:summary}.



\section{The Matrix element method}
\label{sec:mem}

An estimate for the unknown model parameter $a$
is obtained in the ME method by maximizing the probability density:
\begin{align}
P(\bm{y}|a) = & \frac{1}{\sigma(a)} \, \int \, dx_{a} \, dx_{b} \,
d\bm{p} \, \frac{f(x_{a}) f(x_{b})}{2 \, x_{a} \, x_{b} \, s} \, (2\pi)^{4} \,
\delta^{4}( x_{a} \, \bm{p}^{a} + x_{b} \, \bm{p}^{b} - \sum_{i}^{n} \bm{p}^{(i)}) \, \nonumber \\
  & \quad \vert \mathcal{M}(\bm{p},a) \vert^{2} \, W(\bm{y}|\bm{p}) 
\label{eq:mem}
\end{align}
with respect to $a$.
The symbols $p^{a}$ and $p^{b}$ represent the momenta of the two colliding
protons and $\sqrt{s}$ the centre-of-mass energy;
$x_{a}$ and $x_{b}$ denote the Bjorken scaling variables~\cite{Bjorkenx}
and $f(x_{a})$ and $f(x_{b})$ the corresponding parton distribution
functions (PDF).
The term $1/(2 \, x_{a} \, x_{b} \, s)$ is referred to as ``flux factor'' in the literature.
We denote by $n$ the number of particles in the final state,
by $\bm{p}^{(i)}$ the momentum of the $i$-th final state particle
and by $d\bm{p}$ the differential $n$-particle
phase space element.
We use bold letters to indicate vector quantities.
The symbol $\vert \mathcal{M}(\bm{p},a) \vert^{2}$ represents the
squared modulus of the ME for
the process.
The $\delta$-function $\delta^{4}( x_{a} \, \bm{p}^{a} + x_{b} \,
\bm{p}^{b} - \sum_{i}^{n} \bm{p}^{(i)})$ imposes energy and momentum
conservation.
The set of observables measured in the
detector are denoted by $\bm{y}$.
The function $W(\bm{y}|\bm{p})$ represents the probability to
observe the measured values $\bm{y}$ in the detector, given a point $\bm{p}$ in the
$n$-particle phase space and
is referred to as ``transfer function'' (TF) in the
literature~\cite{Fiedler:2010sg,Volobouev:2011vb}.
The symbol $\sigma(a)$ corresponds to the cross section for the process under study.
In case the process includes the production of an unstable resonance,
the term $\sigma(a)$ needs to include 
the branching ratio for the decay of the resonance into the observable final state particles.
Division by $\sigma(a)$ ensures that $P(\bm{y}|a)$ has
the correct normalization for a probability density, 
i.e. $\int \, d\bm{y} \, P(\bm{y}|a) = 1$ for every $a$, 
provided that the TF satisfy the normalization condition
$\int \, d\bm{y}\, W(\bm{y}|\bm{p}) = 1$
for every $\bm{p}$.

The meaning of Eq.~\ref{eq:mem} is as follows.
The best estimate for the unknown model parameter $a$ is given by the
value $\hat{a}$ that maximizes the probability to observe precisely the 
values $\bm{y}$ that are measured in the detector. 
Within the scope of this paper, the unknown model parameter $a$
corresponds to the mass $m_{\PHiggs}$ of the $\PHiggs$ boson that decays into $\Pgt$ pairs.
The $\PHiggs$ boson mass $m_{\PHiggs}$ is equivalent to the mass $m_{\Pgt\Pgt}$ of the $\Pgt$ pair.

The individual terms of Eq.~\ref{eq:mem} are described in
Sections~\ref{sec:mem_ME} to~\ref{sec:mem_logM}.
The ME for the process $\Pp\Pp \to \PHiggs \to \Pgt\Pgt$
with subsequent decay of the $\Pgt$ leptons 
via $\Pgt \to \enunu$, $\Pgt \to \mununu$, and $\Pgt \to \textrm{hadrons} + \Pnut$
is described in Section~\ref{sec:mem_ME}.
A complication arises from the fact that we use a leading order (LO)
ME to model the $\PHiggs$ boson production process $\Pp\Pp \to
\PHiggs$, while the production of $\PHiggs$ bosons at the LHC typically
proceeds in association with jets~\cite{Alwall:2010cq}.
The treatment of hadronic activity in the event is detailed in Section~\ref{sec:mem_hadRecoil}.
The TF are described in section~\ref{sec:mem_TF} and
the integration over the $n$-particle phase space is described in
Section~\ref{sec:mem_PSintegration}.
The computation of the cross section $\sigma(m_{\PHiggs})$ that is needed for proper normalization of the probability density $P(\bm{y}|m_{\PHiggs})$
in Eq.~\ref{eq:mem} is described in Section~\ref{sec:mem_xSection}.
The numerical maximization of the probability density $P(\bm{y}|m_{\PHiggs})$
with respect to the model parameter $m_{\PHiggs}$ is described in
Section~\ref{sec:mem_numericalMaximization}.
We conclude this section with a description of an extra term that we
choose to add artificially to the probability density $P(\bm{y}|m_{\PHiggs})$, in
order to reduce tails in the $m_{\Pgt\Pgt}$ distribution reconstructed
by the algorithm. The structure of this ``regularization'' term is described in
Section~\ref{sec:mem_logM}.
Distributions of $m_{\Pgt\Pgt}$ reconstructed with and without this
term in simulated events are shown in Section~\ref{sec:results}.




\subsection{Matrix element}
\label{sec:mem_ME}

We decompose the squared modulus of the ME, $\vert \mathcal{M}(\bm{p},m_{\PHiggs}) \vert^{2}$, for the process $\Pp\Pp \to \PHiggs \to \Pgt\Pgt$
with subsequent decay of the $\Pgt$ leptons into electrons, muons, or
hadrons into three parts:
\begin{equation}
\vert \mathcal{M}(\bm{p},m_{\PHiggs}) \vert^{2} = 
 \vert \mathcal{M}_{\Pp\Pp \to \PHiggs \to \Pgt\Pgt}(\bm{p},m_{\PHiggs}) \vert^{2} 
\cdot \vert \mathcal{M}^{(1)}_{\Pgt}(\bm{p}) \vert^{2} 
\cdot \vert \mathcal{M}^{(2)}_{\Pgt}(\bm{p}) \vert^{2} .
 \label{eq:meFactorization}
\end{equation}
The first term, $\vert \mathcal{M}_{\Pp\Pp \to \PHiggs \to
  \Pgt\Pgt}(\bm{p},m_{\PHiggs}) \vert^{2}$, represents the squared
modulus of the ME for $\PHiggs$ boson production with subsequent decay of the $\PHiggs$ boson into a pair of $\Pgt$ leptons.
This term can alternatively be computed using automatized tools such as CompHEP or MadGraph or taken from the literature.
We use the labels $^{(1)}$ and $^{(2)}$ to refer to the first and second $\Pgt$ lepton produced in the $\PHiggs$ boson decay, respectively.

\begin{figure}
\begin{center}
\includegraphics*[height=48mm]{figures/ggH_FeynmanDiagram.pdf}
\end{center}
\caption{
  Leading order Feynman diagram for $\PHiggs$ boson production in $\Pp\Pp$ collisions via the gluon fusion process.
}
\label{fig:ggH_FeynmanDiagram}
\end{figure}

We take $\vert \mathcal{M}_{\Pp\Pp \to \PHiggs \to
  \Pgt\Pgt}(\bm{p},m_{\PHiggs}) \vert^{2}$ from the literature.
More specifically, we decompose it into a product of three factors:
\begin{equation}
\vert \mathcal{M}_{\Pp\Pp \to \PHiggs \to \Pgt\Pgt} \vert^{2} =
 \vert \mathcal{M}_{\Pg\Pg \to \PHiggs} \vert^{2} 
\cdot \vert \BW_{\PHiggs} \vert^{2} 
\cdot \vert \mathcal{M}_{\PHiggs \to \Pgt\Pgt} \vert^{2} \, .
\label{eq:meHiggsProduction_and_Decay}
\end{equation}
We model the $\PHiggs$ boson production using the LO ME for the gluon fusion process $\Pg\Pg \to \PHiggs$,
which accounts for about $90\%$ of the total $\PHiggs$ boson production rate at the LHC.
The corresponding Feynman diagram is shown in Fig.~\ref{fig:ggH_FeynmanDiagram}.
The squared modulus of the ME reads~\cite{me_ggHprod}:
\begin{equation}
\vert \mathcal{M}_{\Pg\Pg \to \PHiggs} \vert^{2} = 
 \frac{\sqrt{2} \, G_{F}}{256 \, \pi^{2}} \, \alpha_{s}^{2} \, \tau^{2} \, m_{\PHiggs}^{4} \vert 1 + (1 - \tau) \, f(\tau) \vert^{2} \, ,
\label{eq:meHiggsProduction}
\end{equation}
with $\tau = 4\, \frac{m_{\Pqt}^{2}}{m_{\PHiggs}^{2}}$ and:
\begin{equation}
f(\tau) = 
\begin{cases} 
\arcsin^{2} \frac{1}{\sqrt{\tau}}  & \mbox{if } \tau \geq 1 \, , \\
-\frac{1}{4} \, \left( \log\frac{1 + \sqrt{1 - \tau}}{1 - \sqrt{1 - \tau}} - i\pi \right)^{2} & \mbox{if } \tau < 1 \, .
\end{cases}
\label{eq:meHiggsProduction_ftau}
\end{equation}
The symbol $G_{F}$ denotes the Fermi constant. Its numerical value is:
\begin{equation} 
G_{F} = 1.166 \times 10^{-5}\mbox{~GeV}^{-2} \, (\hbar \, c)^3 \, ,
\label{eq:def_G_F} 
\end{equation}
with $\hbar \, c = 0.1973$~GeV~fm~\cite{PDG}.
The squared modulus of the Breit-Wigner propagator $\vert \BW_{\PHiggs} \vert^{2}$ associates $\PHiggs$ boson production and decay.
It is given by:
\begin{equation}
\vert \BW_{\PHiggs} \vert^{2} = \frac{1}{(q_{\PHiggs}^{2} -
  m_{\PHiggs}^{2})^{2} + m_{\PHiggs}^{2}\Gamma_{\PHiggs}^{2}} \, ,
\label{eq:meHiggsBreitWigner}
\end{equation}
where $q_{\PHiggs}^{2} = (E_{\Pgt(1)} + E_{\Pgt(2)})^{2} - (\bm{p}^{\Pgt(1)} + \bm{p}^{\Pgt(2)})^{2}$ denotes the mass of the $\Pgt$ lepton pair.
We use the symbols $E_{\Pgt(1)}$ and $\bm{p}^{\Pgt(1)}$ ($E_{\Pgt(2)}$ and $\bm{p}^{\Pgt(2)}$) to refer to
the energy and momentum of the $\Pgt$ lepton of positive (negative) charge.
The squared modulus of the ME for the decay of the $\PHiggs$ boson
into $\Pgt$ leptons is given by~\cite{me_HtoTauTau}:
\begin{equation}
\vert \mathcal{M}_{\PHiggs \to \Pgt\Pgt} \vert^{2} = 
 \frac{2 \, m_{\Pgt}^{2}}{v^{2}} \, m_{\PHiggs}^{2} \left( 1 - \frac{4 \, m_{\Pgt}^{2}}{m_{\PHiggs}^{2}} \right) \, ,
\label{eq:meHiggsDecay}
\end{equation}
with $v = \frac{1}{\sqrt{2} \, G_{F}} = 246.2$~\GeV.
The squared modulus of the ME for the decay of the $\PHiggs$ boson
into $\Pgt$ leptons is related to the branching ratio $\mathcal{B}(\PHiggs \to \Pgt\Pgt)$
by~\cite{me_HtoTauTau}:
\begin{equation}
\mathcal{B}(\PHiggs \to \Pgt\Pgt) 
 = \frac{1}{16\pi \, m_{\PHiggs}} \, \sqrt{1 - \frac{4 \, m_{\Pgt}^{2}}{m_{\PHiggs}^{2}}} \, \vert \mathcal{M}_{\PHiggs \to \Pgt\Pgt} \vert^{2} \, .
\label{eq:meHiggsDecay_by_BR}
\end{equation}
For a SM $\PHiggs$ boson,
the branching ratio $\mathcal{B}(\PHiggs \to \Pgt\Pgt)$ becomes small and the total width $\Gamma_{\PHiggs}$ becomes large
once the decay into a pair of $\PW$ bosons is kinematically possible,
\ie for $m_{\PHiggs} \gtrsim 2 \, m_{\PW}$.
In theories beyond the SM, which motivate the search for heavy $\PHiggs$ bosons,
the branching ratio and total width may be very different from the SM
values, however.
In this paper, we assume $\mathcal{B}(\PHiggs \to \Pgt\Pgt) = 100\%$ and compute $\vert \mathcal{M}_{\PHiggs \to \Pgt\Pgt} \vert^{2}$ according to Eq.~(\ref{eq:meHiggsDecay_by_BR}).
Note that the value of the branching ratio for the decay $\PHiggs \to \Pgt\Pgt$ 
has no effect on value of $m_{\Pgt\Pgt}$ reconstructed by the algorithm, 
provided that $\mathcal{B}(\PHiggs \to \Pgt\Pgt)$ is accounted for in
a consistent way in the evaluation of the integral in
Eq.~(\ref{eq:mem}), including the computation of the normalization factor $1/\sigma(m_{\PHiggs})$.

The terms $\vert \mathcal{M}^{\Pgt(1)}(\bm{p}) \vert^{2}$ and $\vert
\mathcal{M}^{\Pgt(2)}(\bm{p}) \vert^{2}$ in Eq.~\ref{eq:meFactorization} model the decays of the two $\Pgt$ leptons.
We use the narrow-width approximation (NWA) and assume the $\Pgt$ leptons to be unpolarized,
effectively ignoring the correlation of spins between $\Pgt$ production and decay.
This allows us to decompose the squared moduli $\vert
\mathcal{M}^{\Pgt(i)}(\bm{p}) \vert^{2}$ for the first ($i=1$) and
second ($i=2$) $\Pgt$ lepton into the product of two factors:
\begin{equation}
\vert \mathcal{M}^{(i)}_{\Pgt} \vert^{2} = \vert \BW_{\Pgt} \vert^{2} \cdot \vert \mathcal{M}^{(i)}_{\textrm{decay}} \vert^{2} \, .
\end{equation}
The factor $\vert \BW_{\Pgt} \vert^{2}$ is equal to:
\begin{equation}
\vert \BW_{\Pgt} \vert^{2} = \frac{\pi}{m_{\Pgt}\Gamma_{\Pgt}} \,
\delta ( q_{\Pgt}^{2} - m_{\Pgt}^{2} ) \, \mbox{ with } \, 
\Gamma_{\Pgt} = \frac{\hbar}{\Delta t} =
 2.267 \cdot 10^{-12}\textrm{~\GeV} \, ,
\end{equation}
where $\Delta t = 290 \times 10^{-15}$~s denotes the lifetime of the
$\Pgt$ lepton~\cite{PDG}.
Concerning the factor $\vert \mathcal{M}^{(i)}_{\textrm{decay}}
\vert^{2}$, we take the ME for the decays $\Pgt \to \enunu$ and $\Pgt
\to \mununu$ from the literature, for the case that the $\Pgt$ leptons
are unpolarized.
We refer to these decays as ``leptonic'' $\Pgt$ decays.
Its squared modulus is given by~\cite{Barger:1987nn}:
\begin{equation}
\vert\mathcal{M}_{\Pgt \to \ellnunu} \vert^{2} = 128 \, G^{2}_{F} \,
\left( E_{\Pgt} \, E_{\APnu_{\Plepton}} - \bm{p}^{\Pgt} \cdot
  \bm{p}^{\APnu_{\Plepton}} \right) \, \left( E_{\Plepton} \,
  E_{\Pnut} - \bm{p}^{\Plepton} \cdot \bm{p}^{\Pnut} \right) \, , 
\label{eq:leptonic_tau_decays_ME}
\end{equation}
The modelling of the decays $\Pgt \to \textrm{hadrons} + \Pnut$ 
by ME is more difficult, 
due to the fact that $\Pgt$ leptons decay to a variety of hadronic
final states, and because some of the decays proceed via intermediate vector
meson resonances~\cite{PDG}.
We refer to these decays as ``hadronic'' $\Pgt$ decays.
The ME for the dominant hadronic $\Pgt$ decay modes are discussed in the literature~\cite{Bullock:1992yt,Raychaudhuri:1995kv}.
We use a simplified formalism and instead treat hadronic $\Pgt$ decays as two-body decays into a hadronic system $\tauh$ of momentum $\bm{p}^{\vis}$ and mass $m_{\vis}$ and a $\Pnut$.
The squared modulus of the ME for the decay is taken to be constant and denoted by $\vert\mathcal{M}^{\eff}_{\Pgt \to \tauhnu}\vert^{2}$.
The value of $\vert\mathcal{M}^{\eff}_{\Pgt \to \tauhnu}\vert^{2}$ is
chosen such that it reproduces the branching fraction for hadronic $\Pgt$ decays.
The following relation holds for the considered case of a two-body decay~\cite{Barger:1987nn}:
\begin{equation}
\mathcal{B}(\Pgt \to \textrm{hadrons} + \Pnut) = \frac{\Delta
  t}{\hbar} \, \frac{1}{16 \pi \, m_{\Pgt}^{3}} \cdot (m_{\Pgt}^{2} - m_{\vis}^{2}) \cdot \vert \mathcal{M}^{\textrm{eff}}_{\Pgt \to
  \tauhnu} \vert^{2} \, ,
\end{equation}
from which it follows that:
\begin{equation}
\vert \mathcal{M}^{\textrm{eff}}_{\Pgt \to \tauh\Pnut} \vert^{2} = \frac{16 \pi \, m_{\Pgt}^{3}}{m_{\Pgt}^{2} - m_{\vis}^{2}} \, \frac{\hbar}{\Delta t} \, \mathcal{B}(\Pgt \to \textrm{hadrons} + \Pnut) \, , 
\end{equation}
with $\mathcal{B}(\Pgt \to \textrm{hadrons} + \Pnut) = 0.648$~\cite{PDG}.
We have verified that the sum of all hadronic final states produced in $\Pgt$ lepton decays
is well reproduced by our simplified model.
Fig.~\ref{fig:tauDecay_z} shows the fraction of $\Pgt$ lepton energy,
in the laboratory frame, carried by the ``visible'' $\Pgt$ decay
products:
\begin{equation}
z = \frac{E_{\vis}}{E_{\Pgt}} \, .
\label{eq:def_z}
\end{equation}
We use the term ``visible'' $\Pgt$ decay products to refer to the sum
of all hadrons produced in decays of the type $\Pgt \to \textrm{hadrons} + \Pnut$ 
as well as to the electron respectively muon produced in the decays $\Pgt \to \enunu$ and $\Pgt \to \mununu$.

\begin{figure}[h]
\setlength{\unitlength}{1mm}
\begin{center}
%%\begin{picture}(150,52)(0,0)
%%\put(-5.5, 0.0){\mbox{\includegraphics*[height=48mm]
%%  {figures/makeSVfitToyMCplots_X1_m90_beforeVisPtCuts.pdf}}}
%%\put(58.0, 0.0){\mbox{\includegraphics*[height=48mm]
%%  {figures/makeSVfitToyMCplots_X2_m90_beforeVisPtCuts.pdf}}}
\begin{picture}(150,60)(0,0)
\put(-5.5, 0.0){\mbox{\includegraphics*[height=58mm]
  {figures/makeSVfitToyMCplots_X1_m90_beforeVisPtCuts.pdf}}}
\put(78.0, 0.0){\mbox{\includegraphics*[height=58mm]
  {figures/makeSVfitToyMCplots_X2_m90_beforeVisPtCuts.pdf}}}
\end{picture}
\end{center}
\caption{
  Fraction $z$ of $\Pgt$ lepton energy, in the laboratory frame, carried by the visible $\Pgt$ decay products.
  The case of $\Pgt \to \mununu$ decays is shown on the left and the case of $\Pgt \to \textrm{hadrons} + \Pnut$ decays on the right.
  Our simplified model, which treats hadronic $\Pgt$ decays as
  two-body decays into a hadronic system $\tauh$ and a $\Pnut$,
  reproduces the distribution in $z$ obtained with a detailed Monte Carlo simulation based on TAUOLA~\cite{tauola}.
  {\textbf CV: OLD PLOTS, TO BE UPDATED !!}
}
\label{fig:tauDecay_z}
\end{figure} 

\subsection{Treatment of hadronic activity}
\label{sec:mem_hadRecoil}

The phase-space for QCD radiation is large at the centre-of-mass
energies of the LHC
and as a consequence, particles with masses up to a few hundred GeV 
are typically produced in association with a sizeable hadronic activity~\cite{Alwall:2010cq}.
We refer to the vectorial sum of all particles in the event that do not originate from the $\PHiggs$ boson decay
as the ``hadronic recoil'' and denote the corresponding momentum by $\bm{p}^{\rec}$.
CMS as well as ATLAS have developed sophisticated techniques to improve the reconstruction 
of the hadronic recoil~\cite{CMS-JME-13-003,ATLAS-CONF-2014-019}.

The effect of the hadronic recoil is to modify the kinematics of the $\PHiggs$ boson decay
by boosting the $\Pgt$ leptons and their decay products,
in the transverse plane and in longitudinal direction.
The longitudinal component of $\bm{p}^{\rec}$ has a small effect on 
the Bjorken scaling variables $x_{a}$ and $x_{b}$ only and can be neglected.
The components of $\bm{p}^{\rec}$ in the transverse plane need to be accounted for, however.
The LO ME that we use to model the production of the $\PHiggs$ boson production 
is adequate for events without hadronic activity,
in which the $\PHiggs$ boson has zero $\pT$.
In order to use the LO ME in analyses at the LHC,
the momenta of the visible $\Pgt$ decay products and of the neutrinos produced in the $\Pgt$ decays
need to be transformed into a frame in which the $\PHiggs$ boson has zero $\pT$.
The transformation is achieved by a Lorentz boost in the transverse plane,
using $(\hat{E}^{\rec} = \pThat^{\rec}, \pXhat^{\rec}, \pYhat^{\rec}, 0)$ as boost
vector.
We denote the true values of energies and momenta by symbols with a
hat if they are given in the laboratory frame, and by symbols a tilde in case they are given in the zero transverse momentum frame of the $\PHiggs$ boson.
Symbols with neither hat nor tilde denote values measured in the
laboratory frame.

The hadronic recoil is reconstructed with a typical resolution of $10$--$20$~\GeV at the LHC~\cite{CMS-JME-13-003,ATLAS-CONF-2014-019}.
It is demonstrated in Ref.~\cite{Alwall:2010cq} that the imperfect reconstruction of the hadronic recoil
in the detector may cause a significant bias on the $\PHiggs$ boson mass reconstructed by the ME method in case one ignores
the experimental resolution on the
hadronic recoil. We account
for the experimental resolution by introduction
of a TF $W_{\rec}( \pX^{\rec}, \pY^{\rec} | \pXhat^{\rec},
\pYhat^{\rec} )$.
We then marginalize Eq.~(\ref{eq:mem}) with respect to $\pXhat^{\rec}$ and $\pYhat^{\rec}$:
\begin{align}
& \mathcal{P}(\bm{y};\pX^{\rec},\pY^{\rec}|m_{\PHiggs}) =
\frac{1}{\sigma(m_{\PHiggs})} \, \int \, dx_{a} \, dx_{b} \, d\Phi_{n} \,
d\pXhat^{\rec} \, d\pYhat^{\rec} \, \frac{f(x_{a}) f(x_{b})}{2 \,
  x_{a} \, x_{b} \, s} \nonumber \\
& \qquad (2\pi)^{4} \, \delta( x_{a} \, \Ehat_{a} + x_{b} \, \Ehat_{b} -
(\Ehat^{\rec} + \Ehat_{\Pgt(1)} + \Ehat_{\Pgt(2)}) ) \, \delta^{3}( x_{a} \,
\bm{\hat{p}}^{a} + x_{b} \, \bm{\hat{p}}^{b} - (\bm{\hat{p}}^{\rec} + \bm{\hat{p}}^{\Pgt(1)}
+ \bm{\hat{p}}^{\Pgt(2)}) ) \nonumber \\
& \qquad \vert \mathcal{M}(\bm{\tilde{p}},m_{\PHiggs}) \vert^{2} \, W(\bm{y}|\bm{\hat{p}}) \, W_{\rec}( \pX^{\rec}, \pY^{\rec} | \pXhat^{\rec}, \pYhat^{\rec} ) \, .
\label{eq:mem_mod_hadRecoil}
\end{align}
The symbol $\bm{y}$ refers to the measured momenta of the visible $\Pgt$ decay products, $\bm{p^{\vis(1)}}$ and $\bm{p^{\vis(2)}}$.
Using the relations $(\hat{E}_{a},\bm{\hat{p}}^{a}) = \frac{\sqrt{s}}{2} \, (1, 0,
0, 1)$ and $(\hat{E}_{b},\bm{\hat{p}}^{b}) = \frac{\sqrt{s}}{2} \, (1,
0, 0, -1)$
for the energies and momenta of the two colliding protons,
the integral over the $\delta$-function that ensures the conservation of energy and momentum can be expressed by:
\begin{align}
& \int \, dx_{a} \, dx_{b} \, d\pXhat^{\rec} \, d\pYhat^{\rec} \, \delta(
x_{a} \, \Ehat_{a} + x_{b} \, \Ehat_{b} - (\Ehat^{\rec} +
\Ehat_{\Pgt(1)} + \Ehat_{\Pgt(2)})) \nonumber \\
& \qquad \delta^{3}(
x_{a} \, \bm{p}^{a} + x_{b} \, \bm{\hat{p}}^{b} - (\bm{\hat{p}}^{\rec} +
\bm{\hat{p}}^{\Pgt(1)} + \bm{\hat{p}}^{\Pgt(2)})) \nonumber \\
= & \, \frac{2}{s} \, \int \,
d\pXhat^{\rec} \, d\pYhat^{\rec} \, 
\delta( \pXhat^{\rec} + \pXhat^{\Pgt(1)} + \pXhat^{\Pgt(2)} ) \,
\delta( \pYhat^{\rec} + \pYhat^{\Pgt(1)} + \pYhat^{\Pgt(2)} ) \, ,
\label{eq:delta4}
\end{align}
with:
\begin{align}
x_{a} = & \frac{1}{\sqrt{s}} \, \left( \Ehat^{\rec} + \Ehat_{\Pgt(1)} +
\Ehat_{\Pgt(2)} + (\pZhat^{\rec} + \pZhat^{\Pgt(1)} +
\pZhat^{\Pgt(2)}) \right)
\, , \nonumber \\
x_{b} = & \frac{1}{\sqrt{s}} \, \left( \Ehat^{\rec} + \Ehat_{\Pgt(1)}
  + \Ehat_{\Pgt(2)} - (\pZhat^{\rec} + \pZhat^{\Pgt(1)} +
  \pZhat^{\Pgt(2)}) \right) \, .
\label{eq:xa_and_xb}
\end{align}
The integration over $d\pXhat^{\rec}$ and $d\pYhat^{\rec}$ removes the $\delta$-functions 
$\delta( \pXhat^{\rec} + \pXhat^{\Pgt(1)} + \pXhat^{\Pgt(2)} )$ and
$\delta( \pYhat^{\rec} + \pYhat^{\Pgt(1)} + \pYhat^{\Pgt(2)} )$ in Eq.~(\ref{eq:delta4}).
Substituting Eq.~(\ref{eq:meFactorization}) into Eq.~(\ref{eq:mem_mod_hadRecoil}), we obtain:
\begin{align}
&
\mathcal{P}(\bm{p}^{\vis(1)},\bm{p}^{\vis(2)};\pX^{\rec},\pY^{\rec}|m_{\PHiggs})
= \frac{1}{\sigma(m_{\PHiggs})} \, \frac{32\pi^{4}}{s} \, \int \,
 d\Phi_{n} \, \frac{f(x_{a}) f(x_{b})}{2 \, x_{a} \, x_{b} \, s} \, 
 \vert \mathcal{M}_{\Pp\Pp \to \PHiggs \to \Pgt\Pgt}(\bm{\tilde{p}},m_{\PHiggs}) \vert^{2} \, \hspace{2cm} \nonumber \\
& \qquad \vert \BW^{(1)}_{\Pgt} \vert^{2} \cdot \vert \mathcal{M}^{(1)}_{\Pgt\to\cdots}(\bm{\tilde{p}}) \vert^{2} 
 \cdot \vert \BW^{(2)}_{\Pgt} \vert^{2} \cdot \vert \mathcal{M}^{(2)}_{\Pgt\to\cdots}(\bm{\tilde{p}}) \vert^{2} \, \nonumber \\
& \qquad W(\bm{p}^{\vis(1)}|\bm{\hat{p}}^{\vis(1)}) \, W(\bm{p}^{\vis(2)}|\bm{\hat{p}}^{\vis(2)}) \, W_{\rec}( \pX^{\rec},\pY^{\rec} | \pXhat^{\rec},\pYhat^{\rec} ) \, ,
\label{eq:mem_with_hadRecoil}
\end{align}
with $x_{a}$ and $x_{b}$ given by Eq.~(\ref{eq:xa_and_xb}) and:
\begin{equation}
\pXhat^{\rec} = -(\pXhat^{\Pgt(1)} + \pXhat^{\Pgt(2)}) \, ,
\qquad \pYhat^{\rec} = -(\pYhat^{\Pgt(1)} + \pYhat^{\Pgt(2)}) \, .
\label{eq:xpXhat_and_pYhat}
\end{equation}
The form of the TF for the hadronic recoil, $W_{\rec}( \pX^{\rec},\pY^{\rec} | \pXhat^{\rec},\pYhat^{\rec} )$, is
discussed in Section~\ref{sec:mem_TF_hadRecoil}.

In practice, the experimental resolution on the momenta of electrons, muons and also $\tauh$
is typically negligible compared to the resolution on the hadronic recoil.
In good approximation, the resolution on the hadronic recoil is equivalent
to the resolution on the vectorial sum of the momenta, in the transverse plane,
of all particles reconstructed in the event.
The latter is referred to as ``missing transverse momentum'' and denoted by $\vecMET$.
The following relations hold for the components of the $\vecMET$
vector:
\begin{equation}
\METx = - \left( \pX^{\rec} + \pX^{\vis(1)} + \pX^{\vis(2)} \right)
\, \mbox { and } \,
\METy = - \left( \pY^{\rec} + \pY^{\vis(1)} + \pY^{\vis(2)} \right) \, .
\label{eq:met}
\end{equation}
These relations are valid on reconstruction level as well as for the
true values of the momenta, \ie Eq.~(\ref{eq:met}) is valid also
in case all $\pX$ and $\pY$ are replaced by $\pXhat$ and $\pYhat$.
Approximating the resolution on the hadronic recoil by the resolution
on $\vecMET$ has the advantage that the resolutions on $\vecMET$ have
been studied in detail and published by the ATLAS as well as CMS collaborations~\cite{CMS-JME-13-003,ATLAS-CONF-2014-019}.


\subsection{Transfer functions}
\label{sec:mem_TF}

The TF $W(p^{\vis(i)}|\phat^{\vis(i)})$ relates the reconstructed
$\pT$, $\eta$, and $\phi$ components of the momentum vector of the
visible $\Pgt$ decay products to their respective true values,
given by the momentum vector $\phat^{\vis(i)}$.
The case of hadronic and leptonic $\Pgt$ decays is described in
Sections~\ref{sec:mem_TF_tauToHadDecays}
and~\ref{sec:mem_TF_tauToLepDecays}, respectively.
In both cases the neutrinos produced in the $\Pgt$ decays are not
included in these TF, but are treated separately, as detailed in Sections~\ref{sec:appendix_tauToHadDecays} and~\ref{sec:appendix_tauToLepDecays} of the appendix.
The TF $W_{\rec}( \pX^{\rec},\pY^{\rec} | \pXhat^{\rec},\pYhat^{\rec} )$ that we use to model the experimental resolution on the
hadronic recoil is discussed in Section~\ref{sec:mem_TF_hadRecoil}.


\subsubsection{$\Pgt \to \textrm{hadrons} + \Pnut$ decays}
\label{sec:mem_TF_tauToHadDecays}

The energy, or equivalently $\pT$, of the $\tauh$ 
is reconstructed with a
resolution of typically $5$--$25\%$ at the
LHC~\cite{Aad:2014rga,TAU-14-001}.
The resolution may vary as function of $\pT$ and $\eta$ and may depend
also on the multiplicity of charged and neutral hadrons produced in the $\Pgt$ decay.
The resolution on the $\tauh$ energy is of similar magnitude as the
resolution on $m_{\Pgt\Pgt}$ that we aim to achieve and needs to be
taken into account by suitable TF when evaluating the integral in
Eq.~\ref{eq:mem_with_hadRecoil}.
We denote the TF that models the $\pT$ of the $\tauh$ system by
$W_{\Phadron}( \bm{p}_{\T}^{\vis} | \phat_{\T}^{\vis} )$.

ATLAS as well as CMS report that the energy response for hadronic
$\Pgt$ decays may be asymmetric~\cite{ATLAS:2011tfa,PRF-14-001}.
For the purpose of this paper, we assume the TF for the $\pT$ of the
$\tauh$ to have the form:
\begin{equation}
W_{\Phadron}( \pT^{\vis} | \pThat^{\vis} ) = 
 \begin{cases}
   \mathcal{N} \, \zeta_{1} \, \left( \frac{\alpha_{1}}{x_{1}} - x_{1} - \frac{x - \mu}{\sigma} \right)^{-\alpha_{1}} \,  
 & \mbox{if } x < x_{1} \, , \\
   \mathcal{N} \, \exp\left( -\frac{1}{2} \, \left( \frac{x - \mu}{\sigma} \right)^{2} \right) \,
 & \mbox{if } x_{1} < x < x_{2} \, , \\
   \mathcal{N} \, \zeta_{2} \, \left( \frac{\alpha_{2}}{x_{2}} - x_{2} - \frac{x - \mu}{\sigma} \right)^{-\alpha_{2}} \,
 & \mbox{if } x > x_{2} \, ,
 \end{cases}
\label{eq:tf_tauToHadDecays_pT}
\end{equation}
with $\mu = 1.0$, $\sigma = 0.03$, $x_{1} = 0.97$, $\alpha_{1} = 7$,
$x_{2} = 1.03$, and $\alpha_{2} = 3.5$.
The factors $\zeta_{1}$ and $\zeta_{2}$ are chosen such that the
function $W_{\Phadron}( \pT^{\vis} | \pThat^{\vis} )$ is continuous at
the points $x = x_{1}$ and $x = x_{2}$. 
Their values are $\zeta_{1} = \left( \frac{\alpha}{x_{1}}
\right)^{\alpha_{1}} \, \exp\left( -\frac{1}{2} \, x_{1}^{2} \right)$ and $\zeta_{2} = \left( \frac{\alpha}{x_{2}}
\right)^{\alpha_{2}} \, \exp\left( -\frac{1}{2} \, x_{2}^{2} \right)$
, respectively.
The factor $\mathcal{N}$ is determined by the condition that the function $W_{\Phadron}( \pT^{\vis} | \pThat^{\vis} )$ 
satisfies the normalization condition $\int \, d\pT^{\vis} \, W_{\Phadron}( \pT^{\vis} | \pThat^{\vis} ) \equiv 1$.

The resolution on the direction of the $\tauh$ is on the level
of a few milliradians and is negligible in practice.
We hence model the TF for the momentum of the $\tauh$ by the product of Eq.~\ref{eq:tf_tauToHadDecays} and two
$\delta$-functions:
\begin{equation}
W_{\Phadron}( \bm{p}^{\vis} | \phat^{\vis} ) =
 \frac{\sin^{2}\theta_{\vis}}{\pT^{\vis}} \, 
  W_{\Phadron}( \pT^{\vis} | \pThat^{\vis} ) \,
  \delta( \theta_{\vis} - \thetahat_{\vis} ) \, 
  \delta( \phi_{\vis} - \phihat_{\vis} ) \, .
\label{eq:tf_tauToHadDecays}
\end{equation}
The factor $\frac{\sin^{2}\theta_{\vis}}{\pT^{\vis}}$ is needed
to ensure the correct normalization of the TF:
\begin{align}
& \int \, d^{3}\bm{p}  \, W_{\Phadron}( \bm{p}^{\vis} | \phat^{\vis} ) = \int \, d\pX^{\vis} \, d\pY^{\vis} \, d\pZ^{\vis} \, W_{\Phadron}( \bm{p}^{\vis} | \phat^{\vis} ) \nonumber \\
& \qquad = \int \, d\pT^{\vis} \, d\theta_{\vis} \, d\phi_{\vis} \,
\frac{\pT^{\vis}}{\sin^{2} \theta_{\vis}} \, W_{\Phadron}( \bm{p}^{\vis} | \phat^{\vis} ) \, 
  \equiv 1 \, ,
\end{align}
where the factor $\frac{\pT^{\vis}}{\sin^{2} \theta_{\vis}}$ corresponds to the Jacobian of the variable transformation 
from $(\pX^{\vis} = \pT^{\vis} \, \cos\phi_{\vis}, \pY^{\vis} = \pT^{\vis} \, \sin\phi_{\vis}, \pZ^{\vis} = \frac{\pT^{\vis}}{\tan\theta_{\vis}})$ 
to $(\pT^{\vis}, \theta_{\vis}, \phi_{\vis})$.

\subsubsection{$\Pgt \to \enunu$ and $\Pgt \to \mununu$ decays}
\label{sec:mem_TF_tauToLepDecays}

For leptonic $\Pgt$ decays,
we use the TF:
\begin{equation}
W_{\Plepton}( \bm{p}^{\vis} | \phat^{\vis} ) =  
 \frac{\sin^{2} \theta_{\vis}}{\pT^{\vis}} \, 
  \delta( \pT^{\vis} - \pThat^{\vis} ) \, 
  \delta( \theta_{\vis} - \thetahat_{\vis} ) \, 
  \delta( \phi_{\vis} - \phihat_{\vis} ) \, ,
\label{eq:tf_tauToLepDecays}
\end{equation}
\ie we assume that the experimental resolution on $\pT^{\vis}$,
$\eta_{\vis}$, and $\phi_{\vis}$ is negligible for electrons as well as
muons.


\subsubsection{Hadronic recoil}
\label{sec:mem_TF_hadRecoil}

We use a two-dimensional normal distribution:
\begin{align}
%%& W_{\rec}( \pX^{\rec},\pY^{\rec} | \pXhat^{\rec},\pYhat^{\rec} ) = \hspace{7cm} \nonumber \\
%%& \quad
%%  \frac{1}{2\pi \, \sqrt{\vert V \vert}} \, \exp \left( -\frac{1}{2}
%%  \left( \begin{array}{c} \Delta\pX^{\rec} \\ \Delta\pY^{\rec} \end{array} \right)^{T}
%%  \cdot V^{-1} \cdot
%%   \left( \begin{array}{c} \Delta\pX^{\rec} \\ \Delta\pY^{\rec} \end{array} \right)
%%  \right) \, ,
W_{\rec}( \pX^{\rec},\pY^{\rec} | \pXhat^{\rec},\pYhat^{\rec} ) = & 
  \frac{1}{2\pi \, \sqrt{\vert V \vert}} \, \exp \left( -\frac{1}{2}
  \left( \begin{array}{c} \Delta\pX^{\rec} \\ \Delta\pY^{\rec} \end{array} \right)^{T}
  \cdot V^{-1} \cdot
   \left( \begin{array}{c} \Delta\pX^{\rec} \\ \Delta\pY^{\rec} \end{array} \right)
  \right) \, ,
\label{eq:tf_hadRecoil}
\end{align}
to model the experimental resolution on the momentum of
the hadronic recoil in the transverse plane.
The symbols:
\begin{equation}
\Delta\pX^{\rec} = \pX^{\rec} - \pXhat^{\rec} \quad \mbox{ and } \quad
\Delta\pY^{\rec} = \pY^{\rec} - \pYhat^{\rec} 
\label{eq:tf_hadRecoil_delta}
\end{equation}
refer to the difference between reconstructed and true values of the
momentum components $\pX^{\rec}$ and $\pY^{\rec}$,
and the covariance matrix:
\begin{equation}
V = \left( \begin{array}{cc} \sigma_{x}^{2} & \rho \, \sigma_{x} \, \sigma_{y} \\ \rho \, \sigma_{x} \, \sigma_{y} & \sigma_{y}^{2} \end{array} \right) 
\label{eq:tf_hadRecoil_V}
\end{equation}
represents the experimental resolution on the hadronic recoil,
with $\vert V \vert$ denoting the determinant of $V$.
Two-dimensional normal distributions have been demonstrated to model well the resolution on $\vecMET$ in case of the CMS experiment~\cite{CMS-JME-10-009,CMS-JME-13-003}.
As explained in Section~\ref{sec:mem_hadRecoil}, the resolution on
$\pX^{\rec}$ and $\pY^{\rec}$ is very similar to the resolution on $\METx$ and $\METy$,
motivating the use of TF of the same type for the modelling of the experimental resolution on the hadronic recoil.

For the purpose of this paper,
we assume that the components $\pX^{\rec}$ and $\pY^{\rec}$ are reconstructed with a resolution of $\sigma_{x} = \sigma_{y} = 10$~\GeV each
and that the differences between reconstructed and true values of both
components are independent, \ie $\rho = 0$.
In the real experiment, the covariance matrix $V$ is estimated on an
event-by-event basis,
depending on the hadronic activity in the event,
and using resolution functions that are obtained from the Monte Carlo simulation~\cite{CMS-JME-10-009,CMS-JME-13-003}.

\subsection{Computation of phase space integral}
\label{sec:mem_PSintegration}

The integration over the differential $n$-particle phase space element
$d\Phi_{n}$ in Eq.~(\ref{eq:mem_with_hadRecoil}) needs to be done differently for
events in which both $\Pgt$ leptons decay hadronically (``hadronic'' $\Pgt$ pair decays),
events in which one $\Pgt$ lepton decays hadronically and one $\Pgt$ lepton decays leptonically (``semi-leptonic'' $\Pgt$ pair decays),
and events in which both $\Pgt$ leptons decay leptonically (``leptonic'' $\Pgt$ pair decays).
With the approximation that we treat hadronic $\Pgt$ decays as
two-body decays, the integral
over $d\Phi_{n}$ is of dimension $12$ in case of hadronic $\Pgt$ pair decays,
$15$ in case of semi-leptonic $\Pgt$ pair decays,
and $18$ in case of leptonic $\Pgt$ pair decays.
The differential phase space element reads:
\begin{equation}
d\Phi_{n}= 
 \begin{cases} 
   d\Phi^{(1)}_{\tauhnu} \, d\Phi^{(2)}_{\tauhnu} \, 
 & \mbox{if } \Pgt^{+} \to \textrm{hadrons} + \APnut \mbox{ and } \Pgt^{-} \to \textrm{hadrons} + \Pnut \\
   d\Phi^{(1)}_{\ellnunu} \, d\Phi^{(2)}_{\tauhnu} \, 
 & \mbox{if } \Pgt^{+} \to \ellPlusnunu \mbox{ and } \Pgt^{-} \to \textrm{hadrons} + \Pnut \\
   d\Phi^{(1)}_{\tauhnu} \, d\Phi^{(2)}_{\ellnunu} \, 
 & \mbox{if } \Pgt^{+} \to \textrm{hadrons} + \APnut \mbox{ and } \Pgt^{-} \to \ellMinusnunu \\
   d\Phi^{(1)}_{\ellnunu} \, d\Phi^{(2)}_{\ellnunu} \, 
 & \mbox{if } \Pgt^{+} \to \ellPlusnunu \mbox{ and } \Pgt^{-} \to \ellMinusnunu \, ,
 \end{cases}
\label{eq:PSintegration_taupair}
\end{equation}
where:
\begin{align}
d\Phi^{(i)}_{\tauhnu} = & \frac{d^{3}\bm{\hat{p}}^{\vis(i)}}{(2\pi)^{3} \, 2
  \hat{E}_{\vis(i)}} \, \frac{d^{3}\bm{\hat{p}}^{\Pnu(i)}}{(2\pi)^{3} \, 2 \hat{E}_{\Pnu(i)}} \nonumber \\
d\Phi^{(i)}_{\ellnunu} = &
\frac{d^{3}\bm{\hat{p}}^{\vis(i)}}{(2\pi)^{3} \, 2 \hat{E}_{\vis(i)}} \, 
\frac{d^{3}\bm{\hat{p}}^{\APnu(i)}}{(2\pi)^{3} \, 2 \hat{E}_{\APnu(i)}} \, 
\frac{d^{3}\bm{\hat{p}}^{\Pnu(i)}}{(2\pi)^{3} \, 2 \hat{E}_{\Pnu(i)}} \, .
\label{eq:PSintegration_onetau}
\end{align}

The integrand in Eq.~(\ref{eq:mem_with_hadRecoil}) depends on the
four-momentum of the two $\Pgt$ leptons via the 
product $f(x_{a}) \, f(x_{b})$ of the PDF,
the factor $1/(2 \, x_{a} \, x_{b} \, s)$, referred to as ``flux factor'' in the literature,
and via the squared modulus $\vert \mathcal{M}_{\Pp\Pp \to \PHiggs \to \Pgt\Pgt}(\bm{\tilde{p}},m_{\PHiggs}) \vert^{2}$ 
of the ME for $\PHiggs$ boson production and subsequent decay into a $\Pgt$ lepton pair.
In addition, it depends on the four-momentum of the visible $\Pgt$ decay products via the
squared moduli $\vert \mathcal{M}^{(1)}_{\Pgt\to\cdots}(\bm{\tilde{p}}) \vert^{2}$ and $\vert \mathcal{M}^{(2)}_{\Pgt\to\cdots}(\bm{\tilde{p}}) \vert^{2}$ of the ME for the $\Pgt$ lepton decays
and via the transfer functions $W( \bm{p}^{\vis(1)} | \phat^{\vis(1)} )$ and $W( \bm{p}^{\vis(2)} | \phat^{\vis(2)} )$.
The $\Pgt$ lepton energies and momenta need to be computed as function of the integration variables.

The dimension of the integration over the phase-space elements
$d^{3}\bm{p}^{\Pnu(i)}$ and $d^{3}\bm{p}^{\APnu(i)} \, d^{3}\bm{p}^{\Pnu(i)}$
can be reduced by means of analytic transformations.
Two variables are sufficient to fully parametrize the kinematics of hadronic $\Pgt$ decays.
In case of leptonic $\Pgt$ decays, three variables are sufficient.
We choose to parametrize hadronic $\Pgt$ decays by the variables $z$ and $\phi_{\inv}$.
The variable $z$ represents the fraction of $\Pgt$ lepton energy, in the laboratory frame,
that is carried by the visible $\Pgt$ decay products (\cf Eq.~(\ref{eq:def_z})).
We denote the energy and momentum of the $\Pgt$ neutrino
produced in hadronic $\Pgt$ decays as well as of the neutrino pair produced in leptonic $\Pgt$
decays by the symbols $E_{\inv}$ and $\bm{p}^{\inv}$.
The energy component $E_{\inv}$ is related to the variable $z$ via:
\begin{equation}
E_{\inv} = \frac{1 - z}{z} \, E_{\vis} \, .
\label{eq:E_inv}
\end{equation}
The angle $\theta_{\inv}$ between the $\bm{p}^{\inv}$ vector and the $\bm{p}^{\vis}$ vector
is related to the variable $z$ as well, 
and is given by Eq.~(\ref{eq:hadTauDecaysCosTheta}) in case of hadronic $\Pgt$ decays 
and by Eq.~(\ref{eq:lepTauDecaysCosTheta}) in case of leptonic $\Pgt$ decays.
The variable $\phi_{\inv}$ specifies the orientation of the
$\bm{p}^{\inv}$ vector relative to the direction of the visible $\Pgt$ decay products.
In case of hadronic $\Pgt$ decays, the magnitude of the $\bm{p}^{\inv}$ vector is equal to $E_{\inv}$.
We choose the mass $m_{\inv}$ of the neutrino pair as third variable to parametrize the kinematics of leptonic $\Pgt$ decays,
so that the magnitude of the $\bm{p}^{\inv}$ vector is given by $\sqrt{\left( \frac{1 - z}{z} \right)^{2} \, E_{\vis}^{2} - m^{2}_{\inv}}$.
With the convention that $m_{\inv} = 0$ for hadronic $\Pgt$ decays,
Eqs.~(\ref{eq:hadTauDecaysCosTheta})
and~(\ref{eq:lepTauDecaysCosTheta}) can be expressed by a common form
that is valid for hadronic as well as for leptonic $\Pgt$ decays:
\begin{equation}
\cos\theta_{\inv} = \frac{\frac{1 - z}{z} \, E_{\vis}^{2} - \frac{1}{2}(m^{2}_{\Pgt} - (m^{2}_{\vis} + m^{2}_{\inv}))}{\vert\bm{p}^{\vis}\vert \, 
  \sqrt{\left( \frac{1 - z}{z} \right)^{2} \, E_{\vis}^{2} - m^{2}_{\inv}}} \, .
\label{eq:theta_inv}
\end{equation}
The $\Pgt$ lepton momentum vector is given by the sum of the
$\bm{p}^{\vis}$ and $\bm{p}^{\inv}$ vectors.

The angles $\theta_{\inv}$ and $\phi_{\inv}$ are illustrated in Fig.~\ref{fig:tauDecayParametrization}.
The $\bm{p}^{\inv}$ vector is located on the surface of a cone,
the axis of which is given by the $\bm{p}^{\vis}$ vector and the
opening angle of which is given by Eq.~(\ref{eq:theta_inv}).
The variable $\phi_{\inv}$ represents the angle of rotation, in
counter-clockwise direction, around the
axis of the cone.
The value $\phi_{\inv} = 0$ is chosen to correspond to the case that
the $\bm{p}^{\inv}$ vector is within the plane spanned by the
$\bm{p}^{\vis}$ vector and the beam direction.

\begin{figure}[h]
\begin{center}
\includegraphics*[height=58mm]{figures/tauDecayParametrization.pdf}
\end{center}
\caption{
  Illustration of the variables $\theta_{\inv}$ and $\phi_{\inv}$ that specify the orientation of the $\bm{p}^{\inv}$ vector
  relative to the momentum vector $\bm{p}^{\vis}$ of the visible $\Pgt$ decay products.
}
\label{fig:tauDecayParametrization}
\end{figure} 

The parametrization of the $\Pgt$ decay kinematics by $\bm{p}^{\vis}$
and the variables $z$ and $\phi_{\inv}$, respectively by $z$, $\phi_{\inv}$, and $m_{\inv}$,
allows one to simplify the evaluation of the integral in Eq.~(\ref{eq:mem_with_hadRecoil}) considerably.
Expressions for the product of the differential phase space elements
$d\Phi^{(i)}_{\tauhnu}$  and $d\Phi^{(i)}_{\ellnunu}$ with the squared
modulus of the ME $\vert \BW_{\Pgt} \vert^{2} \cdot \vert \mathcal{M}^{(i)}_{\Pgt\to\cdots}(\bm{p}) \vert^{2}$ are derived in Sections~\ref{sec:appendix_tauToHadDecays} and~\ref{sec:appendix_tauToLepDecays} of the appendix.
The results are:
\begin{align}
\vert \BW_{\Pgt} \vert^{2} \cdot \vert \mathcal{M}^{(i)}_{\Pgt\to\cdots}(\bm{\tilde{p}}) \vert^{2} \, d\Phi^{(i)}_{\tauhnu} 
 = & \, \frac{\pi}{m_{\Pgt}\Gamma_{\Pgt}} \,
 f_{\Phadron}\left(\bm{\hat{p}}^{\vis(i)}, m^{\vis(i)},
   \bm{\hat{p}}^{\inv(i)}\right) \, \frac{d^{3}\bm{\hat{p}}^{\vis}}{2 \hat{E}_{\vis}} \, dz \, d\phi_{\inv} \nonumber \\
\vert \BW_{\Pgt} \vert^{2} \cdot \vert \mathcal{M}^{(i)}_{\Pgt\to\cdots}(\bm{\tilde{p}}) \vert^{2} \, d\Phi^{(i)}_{\ellnunu} 
 = & \, \frac{\pi}{m_{\Pgt}\Gamma_{\Pgt}} \, f_{\ell}\left(\bm{\hat{p}}^{\vis(i)},
 m^{\vis(i)}, \bm{\hat{p}}^{\inv(i)}\right) \, \frac{d^{3}\bm{\hat{p}}^{\vis}}{2 \hat{E}_{\vis}} \, dz \, dm^{2}_{\inv} \, d\phi_{\inv}
 \, .
\label{eq:PSint}
\end{align}
The functions $f_{\Phadron}$ and $f_{\Plepton}$ are given by
Eqs.~(\ref{eq:hadTauDecays_f})
and~(\ref{eq:lepTauDecays_f}).
%The momenta $\bm{\tilde{p}}^{\vis(i)}$ and $\bm{\tilde{p}}^{\inv(i)}$
%in the frame in which the $\PHiggs$ boson has zero $\pT$
%are obtained by a Lorentz boost in the transverse plane.
%The boost vector $(\pThat^{\rec}, \pXhat^{\rec}, \pYhat^{\rec}, 0)$ is given by Eq.~(\ref{eq:xpXhat_and_pYhat}),
%as function of $\bm{\hat{p}}^{\Pgt(1)} = \bm{\hat{p}}^{\vis(1)} + \bm{\hat{p}}^{\inv(1)}$ and $\bm{\hat{p}}^{\Pgt(2)} = \bm{\hat{p}}^{\vis(2)} + \bm{\hat{p}}^{\inv(2)}$
%in the laboratory frame.

Eq.~(\ref{eq:PSint}) represents the quintessence of what is needed 
in order to extend the ME generated by automatized tools such as
CompHEP or MadGraph
by the capability to handle the $\Pgt$ decays.
Instead of performing an integration over $d^{3}\bm{p}^{\Pgt(1)} \,
d^{3}\bm{p}^{\Pgt(2)}$, which treats the $\Pgt$ leptons as stable particles,
one needs to perform the integration over $\vert \BW^{(1)}_{\Pgt} \vert^{2} \cdot \vert
\mathcal{M}^{(1)}_{\Pgt\to\cdots}(\bm{\tilde{p}}) \vert^{2} \, d\Phi^{(1)} \, \vert \BW^{(2)}_{\Pgt} \vert^{2} \cdot \vert
\mathcal{M}^{(2)}_{\Pgt\to\cdots}(\bm{\tilde{p}}) \vert^{2} \, d\Phi^{(2)}$ according to
Eq.~(\ref{eq:PSint}).
The momenta of both $\Pgt$ leptons need to be computed as
function of the integration variables $z_{(1)}$, $\phi_{\inv}^{(1)}$,
$m_{\inv}^{(1)}$ and $z_{(2)}$, $\phi_{\inv}^{(2)}$,
$m_{\inv}^{(2)}$, using Eqs.~(\ref{eq:E_inv})
and~(\ref{eq:theta_inv}),
where $m_{\inv}^{(i)}$ is equal to zero in case the $i$-th $\Pgt$ lepton
decays hadronically.
The $\Pgt$ lepton momenta can then be used to evaluate the product of the PDF, the flux factor, 
and the squared modulus $\vert \mathcal{M}_{\Pp\Pp \to \PHiggs \to \Pgt\Pgt}(\bm{\tilde{p}},m_{\PHiggs}) \vert^{2}$ 
of the ME for $\PHiggs$ boson production and subsequent decay into a $\Pgt$ lepton pair in Eq.~(\ref{eq:mem_with_hadRecoil}).
 
In order to improve the accuracy of the numerical integration,
we perform a further variable transformation, replacing $z_{(2)}$ by the variable $t_{\PHiggs}$, defined below.
The transformation is executed in two steps. 
First, we replace $z_{(2)}$ by:
\begin{equation}
q_{\PHiggs}^{2} = \frac{q^{2}_{\vis}}{z_{(1)} \, z_{(2)}} \, ,
\label{eq:varTransform_z2_to_tHiggs_1}
\end{equation}
with $q_{\vis}$ denoting the mass of the visible decay products of both
$\Pgt$ leptons.
Following Eq.~(8) in Ref.~\cite{Alwall:2010cq}, we then parametrize $q_{\PHiggs}^{2}$ by:
\begin{equation}
q_{\PHiggs}^{2} = m_{\PHiggs}^{2} + m_{\PHiggs} \, \Gamma_{\PHiggs}
\tan t_{\PHiggs} \, .
\label{eq:varTransform_z2_to_tHiggs_2}
\end{equation}
The form of the variable transformation in Eqs.~(\ref{eq:varTransform_z2_to_tHiggs_1}) and~(\ref{eq:varTransform_z2_to_tHiggs_2}) 
is chosen such that the Jacobi factor of the transformation from
$z_{(2)}$ to $t_{\PHiggs}$ is proportional to the inverse of $\vert \BW_{\PHiggs} \vert^{2}$, 
the squared modulus of the Breit-Wigner propagator of the $\PHiggs$ boson in
Eq.~(\ref{eq:meHiggsBreitWigner}).
The Jacobi factor is given by:
\begin{equation}
\left\lvert \frac{\partial z_{(2)}}{\partial q_{\PHiggs}^{2}} \cdot \frac{\partial
  q_{\PHiggs}^{2}}{\partial t_{\PHiggs}} \right\rvert =
\frac{q^{2}_{\vis}}{q_{\PHiggs}^{4} \, z_{(1)}} \cdot \frac{(q_{\PHiggs}^{2}
  - m_{\PHiggs}^{2})^{2} + m_{\PHiggs}^{2} \,
  \Gamma_{\PHiggs}^{2}}{m_{\PHiggs} \, \Gamma_{\PHiggs}} \, .
\label{eq:JacobiFactor_z2_to_tHiggs}  
\end{equation}
Compared to Ref.~\cite{Alwall:2010cq} we differ by a factor $\frac{1}{\pi}$ in the derivative 
$\frac{\partial q_{\PHiggs}^{2}}{\partial t_{\PHiggs}}$. We have verified that Eq.~(\ref{eq:JacobiFactor_z2_to_tHiggs}) is correct.
This transformation improves the numerical precision of evaluating the integral 
in Eq.~(\ref{eq:mem_with_hadRecoil}) by reducing the variance of the integrand.


\subsection{Computation of $\sigma(m_{\PHiggs})$}
\label{sec:mem_xSection}

According to the paradigm of the ME method, the normalization factor
$1/\sigma(m_{\PHiggs})$ in Eq.~(\ref{eq:mem_with_hadRecoil}) is to be computed
by evaluating the integral:
\begin{equation}
\sigma(m_{\PHiggs}) =
\int \, d^{3}\bm{p}^{\vis(1)} \, d^{3}\bm{p}^{\vis(2)} \,
d\pX^{\rec} \, d\pY^{\rec} \, \mathcal{I}(\bm{p}^{\vis(1)},\bm{p}^{\vis(2)};\pX^{\rec},\pY^{\rec}|m_{\PHiggs}) \, ,
\label{eq:xSection}
\end{equation}
with:
\begin{align}
\mathcal{I} = & \frac{32\pi^{4}}{s} \, \int \, d\Phi_{n} \,
\frac{f(x_{a}) f(x_{b})}{2 \, x_{a} \, x_{b} \, s} \, \vert \mathcal{M}(\tilde{\bm{p}},m_{\PHiggs}) \vert^{2} \, \hspace{2cm} \nonumber \\
& \qquad W(\bm{p}^{\vis(1)}|\hat{\bm{p}}^{\vis(1)}) \, W(\bm{p}^{\vis(2)}|\hat{\bm{p}}^{\vis(2)}) \, W_{\rec}( \pX^{\rec},\pY^{\rec} | \pXhat^{\rec},\pYhat^{\rec} ) \, .
\label{eq:I}
\end{align}
The integration over $d\pX^{\rec}$ and $d\pY^{\rec}$ yields unity,
as only one factor in Eq.~(\ref{eq:I}), the TF $W_{\rec}( \pX^{\rec},\pY^{\rec} | \pXhat^{\rec},\pYhat^{\rec} )$ for the hadronic recoil,
depends on $\pX^{\rec}$ and $\pY^{\rec}$,
and the integral $\int \, d\pX^{\rec} \, d\pY^{\rec} \, W_{\rec}(
\pX^{\rec}, \pY^{\rec} | \pXhat^{\rec}, \pYhat^{\rec} )$ yields unity for all values of $\pXhat^{\rec}$ and $\pYhat^{\rec}$
according to the TF normalization condition.

In order to perform the integration over $d^{3}\bm{p}^{\vis(1)}$ and $d^{3}\bm{p}^{\vis(2)}$ we make two variable transformations.
The first transformation replaces $\pZ^{\vis(2)}$ by $m_{\vis}^{2}$,
the mass of the visible $\Pgt$ decay products, which is given by:
\begin{equation}
m_{\vis}^{2} = 
  \left( E_{\vis(1)} + E_{\vis(2)} \right)^{2} 
- \left( (\pX^{\vis(1)} + \pX^{\vis(2)})^{2} + (\pY^{\vis(1)} + \pY^{\vis(2)})^{2} + (\pZ^{\vis(1)} + \pZ^{\vis(2)})^{2} \right) \, .
\label{eq:m_vis}
\end{equation}
To do this, we insert
\begin{align}
1 = & \int \, dm_{\vis}^{2} \, \delta\left(
  \left( E_{\vis(1)} + E_{\vis(2)} \right)^{2} \right. \nonumber \\ 
& \
 - \left. \left( (\pX^{\vis(1)} + \pX^{\vis(2)})^{2} + (\pY^{\vis(1)} + \pY^{\vis(2)})^{2} + (\pZ^{\vis(1)} + \pZ^{\vis(2)})^{2} \right) - m_{\vis}^{2} \right) \nonumber
\end{align} 
into the integrand of Eq.~(\ref{eq:xSection}) and switch the order of the integration 
over $m_{\vis}^{2}$ and $\pZ^{\vis(2)}$. 
Concerning the integration over $\pZ^{\vis(2)}$, the argument of the $\delta$-function is of the form $g(\pZ^{\vis(2)})$.
We apply the $\delta$-function rule:
\begin{equation} 
\delta \left( g(x) \right) = \sum_{k} \frac{\delta \left( x - x_{k}
  \right)}{\vert g'(x_{k}) \vert} \, .
\label{eq:deltaFuncRule}
\end{equation}
The sum over $k$ includes all roots $x_{k}$ of $g(x)$.
The function $g(\pZ^{\vis(2)})$ has two roots:
\begin{align} 
p_{\textrm{z}}^{\vis(2),+} = & \frac{\pZ^{\vis(1)} \left(m_{\vis}^{2} - 2 \,
  \frac{z^{(2)}}{z^{(1)}} \, \left(p_{\textrm{T}}^{\vis(1)}\right)^{2}\right) + E_{\vis(1)} \,
  \sqrt{m_{\vis}^{2} \left(m_{\vis}^{2} - 4 \, \frac{z^{(2)}}{z^{(1)}} \, \left(p_{\textrm{T}}^{\vis(1)}\right)^{2}\right)}}{2 \, \left(p_{\textrm{T}}^{\vis(1)}\right)^{2}} \nonumber \\
p_{\textrm{z}}^{\vis(2),-} = & \frac{\pZ^{\vis(1)} \left(m_{\vis}^{2} - 2 \,
  \frac{z^{(2)}}{z^{(1)}} \, \left(p_{\textrm{T}}^{\vis(1)}\right)^{2}\right) - E_{\vis(1)} \,
  \sqrt{m_{\vis}^{2} \left(m_{\vis}^{2} - 4 \, \frac{z^{(2)}}{z^{(1)}} \, \left(p_{\textrm{T}}^{\vis(1)}\right)^{2}\right)}}{2 \, \left(p_{\textrm{T}}^{\vis(1)}\right)^{2}} \, .
\label{eq:pzMapping}
\end{align}
Its derivative is:
\begin{equation} 
g'(p_{\textrm{z}}^{\vis(2)}) \equiv 
  \frac{\partial g(\pZ^{\vis(2)})}{\partial \pZ^{\vis(2)}} = 
 - 2 \, p_{\textrm{z}}^{\vis(1)} 
 + 2 \, p_{\textrm{z}}^{\vis(2)} \, \sqrt{\frac{E_{\vis(1)}}{\left( \frac{z^{(2)}}{z^{(1)}} \, p_{\textrm{T}}^{\vis(1)} \right)^{2} + \left(\pZ^{\vis(2)}\right)^{2}}} \, ,
\end{equation}
yielding:
\begin{align}
& \sigma(m_{\PHiggs}) =
\int \, d^{3}\bm{p^{\vis(1)}} \, d^{3}\bm{p^{\vis(2)}} \, \mathcal{I} \nonumber \\
& \qquad 
 = \Sigma_{k} \, \int \, d\pX^{\vis(1)} \, d\pY^{\vis(1)} \,
d\pZ^{\vis(1)} \, d\pX^{\vis(2)} \, d\pY^{\vis(2)} \, dm_{\vis}^{2} \,
\mathcal{I} \, \frac{1}{\vert g'(p_{\textrm{z}}^{\vis(2)}) \vert} \, ,
\label{eq:xSection2}
\end{align}
where the sum extends over the two roots $p_{\textrm{z}}^{\vis(2),+}$
and $p_{\textrm{z}}^{\vis(2),-}$ that are given by Eq.~(\ref{eq:pzMapping}).

The second transformation replaces $\pX^{\vis(1)}$, $\pY^{\vis(1)}$, $\pX^{\vis(2)}$, and $\pY^{\vis(2)}$ by:
\begin{align}
\uX = & \frac{\pX^{\vis(1)}}{z^{(1)}} +
\frac{\pX^{\vis(2)}}{z^{(2)}} \, , \qquad
  \uY = \frac{\pY^{\vis(1)}}{z^{(1)}} + \frac{\pX^{\vis(2)}}{z^{(2)}} \nonumber \\
\vX = & \frac{\pX^{\vis(1)}}{z^{(1)}} -
\frac{\pX^{\vis(2)}}{z^{(2)}} \, , \qquad
  \vY = \frac{\pY^{\vis(1)}}{z^{(1)}} - \frac{\pX^{\vis(2)}}{z^{(2)}} \, .
\label{eq:varTransform_uX_uY_vX_vY}
\end{align}
The determinant of the Jacobi matrix for this transformation is given by $\vert J \vert = \left( \frac{z^{(1)} \, z^{(2)}}{2} \right)^{2}$.
The variables $\pX^{\vis(1)}$, $\pY^{\vis(1)}$, $\pX^{\vis(2)}$, and $\pY^{\vis(2)}$ are given 
as function of $\uX$, $\uY$, $\vX$, and $\vY$ by:
\begin{align}
\pX^{\vis(1)} = & \frac{z^{(1)}}{2} \, ( \uX + \vX ) \, , \qquad
  \pY^{\vis(1)} = \frac{z^{(1)}}{2} \, ( \uY + \vY ) \nonumber \\
\pX^{\vis(2)} = & \frac{z^{(2)}}{2} \, ( \uX - \vX ) \, , \qquad
  \pY^{\vis(2)} = \frac{z^{(2)}}{2} \, ( \uY - \vY ) \, . 
\label{eq:varTransform_uX_uY_vX_vY_inverse}
\end{align}
The variables $\uX$ and $\uY$ correspond to the vectorial sum of the $\Pgt^{(1)}$ and $\Pgt^{(2)}$ lepton momenta in the transverse plane,
\ie to the $x$ and $y$ components of the $\PHiggs$ boson momentum,
while the variables $\vX$ and $\vY$ correspond to the difference
between the $\Pgt^{(1)}$ and $\Pgt^{(2)}$ momenta.

The following expression for the cross section $\sigma(m_{\PHiggs})$
is obtained after the two variable transformation:
\begin{align}
& \sigma(m_{\PHiggs}) =
\int \, d^{3}\bm{p}^{\vis(1)} \, d^{3}\bm{p}^{\vis(2)} \,
\mathcal{I}(\bm{p}^{\vis(1)},\bm{p}^{\vis(2)};\pX^{\rec},\pY^{\rec}|m_{\PHiggs})
\nonumber \\
& \qquad
  = \Sigma_{k} \, \int \, d\uX \, d\uY \, d\vX \, d\vY \, d\pZ^{\vis(1)}
\, dm_{\vis}^{2} \, \mathcal{I} \, \frac{1}{\vert
  g'(p_{\textrm{z}}^{\vis(2)}) \vert} \, \left( \frac{z^{(1)} \, z^{(2)}}{2} \right)^{2} \, .
\label{eq:xSection3}
\end{align}
The integral in Eq.~(\ref{eq:xSection3}) is evaluated numerically.

The cross section $\sigma(m_{\PHiggs})$ given by
Eq.~(\ref{eq:xSection3}) is used for the purpose of normalizing
the probability density $P(\bm{p^{\vis(1)}},\bm{p^{\vis(2)}};\pX^{\rec},\pY^{\rec}|m_{\PHiggs})$ in Eq.~(\ref{eq:mem_with_hadRecoil}).
The value of $\sigma(m_{\PHiggs})$ cannot be directly compared to the literature,
due to the fact that we are applying the LO ME to events in which the $\PHiggs$ boson has non-zero $\pT$.
For the purpose of comparing $\sigma(m_{\PHiggs})$ to literature
values for the LO $\Pg\Pg \to \PHiggs$ cross section,
we insert two $\delta$-functions of form $\delta(\uX) \, \delta(\uY)$
into Eq.~(\ref{eq:xSection3}) that
remove the integration over $d\uX$ and $d\uY$ before the integral is
evaluated numerically.
The values of $\sigma(m_{\PHiggs})$ obtained in this way are shown as function of
$m_{\PHiggs}$ in Fig.~\ref{fig:xSection}.
The values agree with the literature values for the LO $\Pg\Pg \to
\PHiggs$ cross section within $10$--$20\%$. 
Differences of this magnitude are expected, as the literature values
have been computed using a different PDF set.

\begin{figure}
\begin{center}
%%\includegraphics*[height=60mm]{figures/makeSVfitMEM_xSectionPlot_log.pdf}
\includegraphics*[height=74mm]{figures/makeSVfitMEM_xSectionPlot_log.pdf}
\end{center}
\caption{
  Cross section $\sigma(m_{\PHiggs})$ 
  that is used in the normalization of the probability density $P(\bm{p}^{\vis(1)},\bm{p}^{\vis(2)};\pX^{\rec},\pY^{\rec}|m_{\PHiggs})$ in Eq.~(\ref{eq:mem_with_hadRecoil})
  as function of mass $m_{\PHiggs}$ of the $\PHiggs$ boson.
}
\label{fig:xSection}
\end{figure}


\subsection{Determination of $\hat{m}_{\Pgt\Pgt}$}
\label{sec:mem_numericalMaximization}

The best estimate, $\hat{m}_{\Pgt\Pgt}$, for the mass of the $\Pgt$ lepton pair in a given event
is obtained by computing the probability density $P(\bm{p}^{\vis(1)},\bm{p}^{\vis(2)};\pX^{\rec},\pY^{\rec}|m_{\Pgt\Pgt}^{\textrm{test}(i)})$ 
for a series of mass hypotheses $m_{\Pgt\Pgt}^{\textrm{test}(i)}$, using Eq.~(\ref{eq:mem_with_hadRecoil}), and determining the value of $m_{\Pgt\Pgt}$ that maximizes this probability density.
The integral in Eq.~(\ref{eq:mem_with_hadRecoil}) is evaluated numerically
using an improved implementation~\cite{VAMP} of the VEGAS algorithm~\cite{VEGAS}.
The series of mass hypotheses is defined by a recursive relation: 
\begin{equation}
m_{\Pgt\Pgt}^{\textrm{test}(i + 1)} = (1 + \delta) \,  m_{\Pgt\Pgt}^{\textrm{test}(i)} \, ,
\label{eq:mTauTau_step_size}
\end{equation}
The step size $\delta$ is chosen such that it is small compared to the expected $m_{\Pgt\Pgt}$ resolution,
typically amounting to $15$--$20\%$ relative to the true mass of the $\Pgt$ lepton pair.
In order to reduce the computing time the series is computed in two passes.
The purpose of the first pass, which uses a step size of $\delta = 0.10$, is to find the region of maximal $P(\bm{p}^{\vis(1)},\bm{p}^{\vis(2)};\pX^{\rec},\pY^{\rec}|m_{\Pgt\Pgt}^{\textrm{test}(i)})$.
The series is started at $m_{\Pgt\Pgt}^{\textrm{test}(0)} = m_{\vis}$,
the mass of the visible $\Pgt$ decay products, 
and stops when $P(\bm{p}^{\vis(1)},\bm{p}^{\vis(2)};\pX^{\rec},\pY^{\rec}|m_{\Pgt\Pgt}^{\textrm{test}(i)})$ falls below one per mille of the maximal $P(\bm{p}^{\vis(1)},\bm{p}^{\vis(2)};\pX^{\rec},\pY^{\rec}|m_{\Pgt\Pgt}^{\textrm{test}(i)})$ value
computed for any $m_{\Pgt\Pgt}^{\textrm{test}(i)}$ so far in a given event.
The the second pass further $P(\bm{p}^{\vis(1)},\bm{p}^{\vis(2)};\pX^{\rec},\pY^{\rec}|m_{\Pgt\Pgt}^{\textrm{test}(i)})$ values are computed for mass hypotheses $m_{\Pgt\Pgt}^{\textrm{test}(i)})$ within a region around the maximum,
using a step size of $\delta = 0.01$.
The graph of $\log \left( P(\bm{p}^{\vis(1)},\bm{p}^{\vis(2)};\pX^{\rec},\pY^{\rec}|m_{\Pgt\Pgt}^{\textrm{test}(i)}) \mbox{~GeV}^{8} \right)$ versus $m_{\Pgt\Pgt}^{\textrm{test}(i)}$ is fitted by a second order polynomial
in the region around the maximum,
and $\hat{m}_{\Pgt\Pgt}$ is taken to be the point at which the polynomial reaches its maximum.

\subsection{Artificial regularization term}
\label{sec:mem_logM}

The $m_{\Pgt\Pgt}$ distribution reconstructed in $\PHiggs \to \Pgt\Pgt$ signal events is expected to peak close to the true value of the $\PHiggs$ boson mass,
while the distribution of $m_{\Pgt\Pgt}$ obtained for the irreducible $\PZ/\Pggx \to \Pgt\Pgt$ background is steeply falling.
The sensitivity to discover a $\PHiggs$ boson signal increases if high mass tails in the $m_{\Pgt\Pgt}$ distribution reconstructed 
for the $\PZ/\Pggx \to \Pgt\Pgt$ background, arising from resolution effects, are avoided.
For this purpose,
we add an artificial regularization term of form 
$\kappa \, \log \left( m_{\PHiggs}^{\textrm{test}(i)}~/\mbox{~GeV} \right)$ 
to the logarithm of the probability density $\mathcal{P}(\bm{p}^{\vis(1)},\bm{p}^{\vis(2)};\pX^{\rec},\pY^{\rec}|m_{\PHiggs}^{\textrm{test}(i)})$
that is computed according to Eq.~(\ref{eq:mem_with_hadRecoil}).
In case of non-zero $\kappa$,
the procedure described in Section~\ref{sec:mem_numericalMaximization} for finding the best estimate $m_{\Pgt\Pgt}$ for the mass of the $\Pgt$ lepton pair is altered.
Instead of fitting 
$\log \left( \mathcal{P}~/\mbox{~GeV}^{8} \right)$ 
versus $m_{\PHiggs}^{\textrm{test}(i)}$ by a second order polynomial,
we fit the sum of $\log \left( \mathcal{P}~/\mbox{~GeV}^{8} \right)$
and $\kappa \, \log \left( m_{\PHiggs}^{\textrm{test}(i)}~/\mbox{~GeV} \right)$.
The parameter $\kappa$ is chosen with the objective of reducing the high mass tail for the $\PZ/\Pggx \to \Pgt\Pgt$ background,
while causing at most a small bias on the $m_{\Pgt\Pgt}$ distribution reconstructed in signal events.
We find that the optimal value of $\kappa$ depends on the experimental resolution
(higher experimental resolutions favor smaller values of $\kappa$) and hence needs to be determined specific to the experimental conditions.


\section{Results}
\label{sec:results}

The performance of the $m_{\Pgt\Pgt}$ reconstruction is studied using
simulated samples of $\PHiggs \to \Pgt\Pgt$ and $\PZ/\Pggx \to
\Pgt\Pgt$ events.
A SM $\PHiggs \to \Pgt\Pgt$ signal sample for a $\PHiggs$ boson of $m_{\PHiggs} = 125$~\GeV is generated with the next-to-leading-order (NLO) program POWHEG v2~\cite{POWHEG1,POWHEG2,POWHEG3}.
We also study the $m_{\Pgt\Pgt}$ reconstruction in events containing pseudoscalar $\PHiggs$ bosons of mass $200$, $300$, $500$, and $700$~\GeV.
The latter samples are generated with the LO generator PYTHIA 8.2~\cite{pythia8}.
In all cases, the $\PHiggs$ bosons are produced by the gluon fusion process.
The $\PZ/\Pggx \to \Pgt\Pgt$ background sample is generated with the LO MadGraph program, in the version MadGraph\_aMCatNLO 2.2.2~\cite{MadGraph_aMCatNLO}.
All events are generated for proton-proton collisions at $\sqrt{s} = 13$~\TeV centre-of-mass energy.
The samples produced by MadGraph and POWHEG are generated with the NNPDF3.0 set of parton distribution functions,
while the samples produced by PYTHIA use the NNPDF2.3LO set~\cite{NNPDF1,NNPDF2,NNPDF3}.
Parton shower and hadronization processes are modelled using the generator PYTHIA with the tune CUETP8M1~\cite{PYTHIA_CUETP8M1tune_CMS}.
The latter is based on the Monash tune~\cite{PYTHIA_MonashTune}.
The decays of $\Pgt$ leptons, including polarization effects, are modelled by PYTHIA.

The experimental resolutions on $\pT^{\vis}$ of $\tauh$ and on $\pXhat^{\rec}$ and $\pYhat^{\rec}$ of the hadronic recoil 
are simulated by sampling from the TF described in
Sections~\ref{sec:mem_TF_tauToHadDecays}
and~\ref{sec:mem_TF_hadRecoil}, respectively.
The $\eta$ and $\phi$ components of the of the $\bm{p}^{\vis}$ vector of electrons, muons and $\tauh$,
as well as the $\pT$ of electrons and muons are assumed to be reconstructed perfectly.

Distributions in $m_{\Pgt\Pgt}$ are computed separately for events in which 
both $\Pgt$ leptons decay hadronically ($\tauh\tauh$), 
events in which one $\Pgt$ lepton decays hadronically and the other into a muon ($\Pgm\tauh$),
and events in which one $\Pgt$ leptons decays into a muon and the other into an electron ($\Pe\Pgm$).
The visible $\Pgt$ decay products are required to pass selection criteria on $\pT$ and $\eta$ 
that are motivated by the SM $\PHiggs \to \Pgt\Pgt$ analysis performed by the CMS collaboration~\cite{HIG-13-004}.
Events in the $\tauh\tauh$ decay channel are required to contain
two $\tauh$ with $\pT > 45$~\GeV and $\vert\eta\vert < 2.1$.
Events in the $\Pgm\tauh$ channel
are required to contain one muon with $\pT > 20$~\GeV and $\vert\eta\vert < 2.1$ plus one $\tauh$ with $\pT > 30$~\GeV and $\vert\eta\vert < 2.3$.
Events selected in the $\Pe\Pgm$ channel are required to contain a muon of $\pT > 10$~\GeV and $\vert\eta\vert < 2.1$ plus an electron of $\pT > 10$~\GeV and $\vert\eta\vert < 2.4$.
At least one lepton is required to satisfy the condition $\pT > 20$~\GeV, while the lepton of lower $\pT$ is required to satisfy $\pT > 10$~\GeV.
Similar selection criteria on $\pT$ and $\eta$ of the visible $\Pgt$ decay products are applied in the $\PHiggs \to \Pgt\Pgt$
analyses performed by the ATLAS
collaboration~\cite{ATLAS_HiggsTauTau_SM,ATLAS_HiggsTauTau_MSSM}.

The $m_{\Pgt\Pgt}$ distributions reconstructed using the 
improved version of the SVfit algorithm described in this paper
are compared to the distributions in $m_{\Pgt\Pgt}$ reconstructed by the previous version of the
SVfit algorithm described in Ref.~\cite{SVfit} and to the distribution of the $\PHiggs$ boson
mass reconstructed by the ``collinear-approximation'' (CA)
method~\cite{massRecoCollinearApprox}.
The improced version of the SVfit algorithm is denoted by SVfitMEM 
and is run with two values of the $\kappa$ parameter described in Section~\ref{sec:mem_logM}, $0$ and $5$.
We refer to the previous version of the SVfit algorithm as ``classic'' SVfit.
The results are shown in Figs.~\ref{fig:massDistributions_tautau} to~\ref{fig:massDistributions_emu}.
The ordinate is drawn in logarithmic scale to better visualize differences in the high mass tails.

\begin{figure}
\setlength{\unitlength}{1mm}
\begin{center}
\ifx\ver\verPAPER
\begin{picture}(160,170)(0,0)
\put(-5.5, 120.0){\mbox{\includegraphics*[height=50mm]
  {plots/makeSVfitMEM_PerformancePlots_DYJets_hadhad_log.pdf}}}
\put(64.0, 120.0){\mbox{\includegraphics*[height=50mm]
  {plots/makeSVfitMEM_PerformancePlots_HiggsSUSYGluGlu125_hadhad_log.pdf}}}
\put(-5.5, 62.0){\mbox{\includegraphics*[height=50mm]
  {plots/makeSVfitMEM_PerformancePlots_HiggsSUSYGluGlu300_hadhad_log.pdf}}}
\put(64.0, 62.0){\mbox{\includegraphics*[height=50mm]
  {plots/makeSVfitMEM_PerformancePlots_HiggsSUSYGluGlu500_hadhad_log.pdf}}}
\put(-5.5, 4.0){\mbox{\includegraphics*[height=50mm]
  {plots/makeSVfitMEM_PerformancePlots_HiggsSUSYGluGlu800_hadhad_log.pdf}}}
\put(64.0, 4.0){\mbox{\includegraphics*[height=50mm]
  {plots/makeSVfitMEM_PerformancePlots_legend_hadhad.pdf}}}
\put(28.0, 116.0){\small (a)}
\put(97.5, 116.0){\small (b)}
\put(28.0, 58.0){\small (c)}
\put(97.5, 58.0){\small (d)}
\put(29.5, 0.0){\small (e)}
\fi
\ifx\ver\verPreprint
\begin{picture}(160,216)(0,0)
\put(-2.5, 152.0){\mbox{\includegraphics*[height=64mm]
  {plots/makeSVfitMEM_PerformancePlots_DYJets_hadhad_log.pdf}}}
\put(79.0, 152.0){\mbox{\includegraphics*[height=64mm]
  {plots/makeSVfitMEM_PerformancePlots_HiggsSUSYGluGlu125_hadhad_log.pdf}}}
\put(-2.5, 77.0){\mbox{\includegraphics*[height=64mm]
  {plots/makeSVfitMEM_PerformancePlots_HiggsSUSYGluGlu300_hadhad_log.pdf}}}
\put(79.0, 77.0){\mbox{\includegraphics*[height=64mm]
  {plots/makeSVfitMEM_PerformancePlots_HiggsSUSYGluGlu500_hadhad_log.pdf}}}
\put(-2.5, 2.0){\mbox{\includegraphics*[height=64mm]
  {plots/makeSVfitMEM_PerformancePlots_HiggsSUSYGluGlu800_hadhad_log.pdf}}}
\put(79.0, 2.0){\mbox{\includegraphics*[height=64mm]
  {plots/makeSVfitMEM_PerformancePlots_legend_hadhad.pdf}}}
\put(35.5, 150.0){\small (a)}
\put(117.0, 150.0){\small (b)}
\put(35.5, 75.0){\small (c)}
\put(117.0, 75.0){\small (d)}
\put(35.5, 0.0){\small (e)}
\fi
\end{picture}
\end{center}
\caption{
  Distribution of alternative mass variables in simulated $\PZ/\Pggx \to \Pgt\Pgt$ (a) 
  and $\PHiggs \to \Pgt\Pgt$ events of different mass:
  $125$~\GeV (b), $300$~\GeV (c), $500$~\GeV (d), and $800$~\GeV (e).
  The events are selected in the $\tauh\tauh$ decay channel.
}
\label{fig:massDistributions_tautau}
\end{figure}

\begin{figure}
\setlength{\unitlength}{1mm}
\begin{center}
\ifx\ver\verPAPER
\begin{picture}(160,170)(0,0)
\put(-5.5, 120.0){\mbox{\includegraphics*[height=50mm]
  {plots/makeSVfitMEM_PerformancePlots_DYJets_muhad_log.pdf}}}
\put(64.0, 120.0){\mbox{\includegraphics*[height=50mm]
  {plots/makeSVfitMEM_PerformancePlots_HiggsSUSYGluGlu125_muhad_log.pdf}}}
\put(-5.5, 62.0){\mbox{\includegraphics*[height=50mm]
  {plots/makeSVfitMEM_PerformancePlots_HiggsSUSYGluGlu300_muhad_log.pdf}}}
\put(64.0, 62.0){\mbox{\includegraphics*[height=50mm]
  {plots/makeSVfitMEM_PerformancePlots_HiggsSUSYGluGlu500_muhad_log.pdf}}}
\put(-5.5, 4.0){\mbox{\includegraphics*[height=50mm]
  {plots/makeSVfitMEM_PerformancePlots_HiggsSUSYGluGlu800_muhad_log.pdf}}}
\put(64.0, 4.0){\mbox{\includegraphics*[height=50mm]
  {plots/makeSVfitMEM_PerformancePlots_legend_muhad.pdf}}}
\put(28.0, 116.0){\small (a)}
\put(97.5, 116.0){\small (b)}
\put(28.0, 58.0){\small (c)}
\put(97.5, 58.0){\small (d)}
\put(29.5, 0.0){\small (e)}
\fi
\ifx\ver\verPreprint
\begin{picture}(160,216)(0,0)
\put(-2.5, 152.0){\mbox{\includegraphics*[height=64mm]
  {plots/makeSVfitMEM_PerformancePlots_DYJets_muhad_log.pdf}}}
\put(79.0, 152.0){\mbox{\includegraphics*[height=64mm]
  {plots/makeSVfitMEM_PerformancePlots_HiggsSUSYGluGlu125_muhad_log.pdf}}}
\put(-2.5, 77.0){\mbox{\includegraphics*[height=64mm]
  {plots/makeSVfitMEM_PerformancePlots_HiggsSUSYGluGlu300_muhad_log.pdf}}}
\put(79.0, 77.0){\mbox{\includegraphics*[height=64mm]
  {plots/makeSVfitMEM_PerformancePlots_HiggsSUSYGluGlu500_muhad_log.pdf}}}
\put(-2.5, 2.0){\mbox{\includegraphics*[height=64mm]
  {plots/makeSVfitMEM_PerformancePlots_HiggsSUSYGluGlu800_muhad_log.pdf}}}
\put(79.0, 2.0){\mbox{\includegraphics*[height=64mm]
  {plots/makeSVfitMEM_PerformancePlots_legend_muhad.pdf}}}
\put(35.5, 150.0){\small (a)}
\put(117.0, 150.0){\small (b)}
\put(35.5, 75.0){\small (c)}
\put(117.0, 75.0){\small (d)}
\put(35.5, 0.0){\small (e)}
\fi
\end{picture}
\end{center}
\caption{
  Distribution of alternative mass variables in simulated $\PZ/\Pggx \to \Pgt\Pgt$ (a) 
  and $\PHiggs \to \Pgt\Pgt$ events of different mass:
  $125$~\GeV (b), $300$~\GeV (c), $500$~\GeV (d), and $800$~\GeV (e).
  The events are selected in the $\Pgm\tauh$ decay channel.
}
\label{fig:massDistributions_mutau}
\end{figure}

\begin{figure}
\setlength{\unitlength}{1mm}
\begin{center}
\ifx\ver\verPAPER
\begin{picture}(160,170)(0,0)
\put(-5.5, 120.0){\mbox{\includegraphics*[height=50mm]
  {plots/makeSVfitMEM_PerformancePlots_DYJets_emu_log.pdf}}}
\put(64.0, 120.0){\mbox{\includegraphics*[height=50mm]
  {plots/makeSVfitMEM_PerformancePlots_HiggsSUSYGluGlu125_emu_log.pdf}}}
\put(-5.5, 62.0){\mbox{\includegraphics*[height=50mm]
  {plots/makeSVfitMEM_PerformancePlots_HiggsSUSYGluGlu300_emu_log.pdf}}}
\put(64.0, 62.0){\mbox{\includegraphics*[height=50mm]
  {plots/makeSVfitMEM_PerformancePlots_HiggsSUSYGluGlu500_emu_log.pdf}}}
\put(-5.5, 4.0){\mbox{\includegraphics*[height=50mm]
  {plots/makeSVfitMEM_PerformancePlots_HiggsSUSYGluGlu800_emu_log.pdf}}}
\put(64.0, 4.0){\mbox{\includegraphics*[height=50mm]
  {plots/makeSVfitMEM_PerformancePlots_legend_emu.pdf}}}
\put(28.0, 116.0){\small (a)}
\put(97.5, 116.0){\small (b)}
\put(28.0, 58.0){\small (c)}
\put(97.5, 58.0){\small (d)}
\put(29.5, 0.0){\small (e)}
\fi
\ifx\ver\verPreprint
\begin{picture}(160,216)(0,0)
\put(-2.5, 152.0){\mbox{\includegraphics*[height=64mm]
  {plots/makeSVfitMEM_PerformancePlots_DYJets_emu_log.pdf}}}
\put(79.0, 152.0){\mbox{\includegraphics*[height=64mm]
  {plots/makeSVfitMEM_PerformancePlots_HiggsSUSYGluGlu125_emu_log.pdf}}}
\put(-2.5, 77.0){\mbox{\includegraphics*[height=64mm]
  {plots/makeSVfitMEM_PerformancePlots_HiggsSUSYGluGlu300_emu_log.pdf}}}
\put(79.0, 77.0){\mbox{\includegraphics*[height=64mm]
  {plots/makeSVfitMEM_PerformancePlots_HiggsSUSYGluGlu500_emu_log.pdf}}}
\put(-2.5, 2.0){\mbox{\includegraphics*[height=64mm]
  {plots/makeSVfitMEM_PerformancePlots_HiggsSUSYGluGlu800_emu_log.pdf}}}
\put(79.0, 2.0){\mbox{\includegraphics*[height=64mm]
  {plots/makeSVfitMEM_PerformancePlots_legend_emu.pdf}}}
\put(35.5, 150.0){\small (a)}
\put(117.0, 150.0){\small (b)}
\put(35.5, 75.0){\small (c)}
\put(117.0, 75.0){\small (d)}
\put(35.5, 0.0){\small (e)}
\fi
\end{picture}
\end{center}
\caption{
  Distribution of alternative mass variables in simulated $\PZ/\Pggx \to \Pgt\Pgt$ (a) 
  and $\PHiggs \to \Pgt\Pgt$ events of different mass:
  $125$~\GeV (b), $300$~\GeV (c), $500$~\GeV (d), and $800$~\GeV (e).
  The events are selected in the $\Pe\Pgm$ decay channel.
}
\label{fig:massDistributions_emu}
\end{figure}

The distributions reconstructed by the SVfitMEM algorithm with $\kappa = 0$ are very similar to the distributions in $m_{\Pgt\Pgt}$ reconstructed by the CA method.
In both cases, pronounced high mass tails reduce the sensitivity of the SM $\PHiggs \to \Pgt\Pgt$ analysis,
because a sizeable fraction of $\PZ/\Pggx \to \Pgt\Pgt$ background events
are reconstructed near $m_{\Pgt\Pgt} \approx 125$~\GeV due to resolution effects.
The advantage of the SVfitMEM algorithm with $\kappa =
0$ is that it provides a physical solution for every event,
while the CA method fails to yield a physical solution for more than $50\%$ of the events.
The large fraction of events for which the CA method fails to find a physical solution is reflected by the normalization of the distribution.
The high mass tail in the $m_{\Pgt\Pgt}$ distribution reconstructed by
the SVfitMEM algorithm is reduced substantially by using
$\kappa = 5$ instead of $\kappa = 0$.
The distributions in $m_{\Pgt\Pgt}$ reconstructed by the SVfitMEM
algorithm with $\kappa = 5$ and by the classic SVfit algorithm are
very similar.
We conclude that the arbitrary normalization used by the classic SVfit
algorithm has an effect that is similar to adding 
an artificial regularization term of form $5 \cdot \log \left( m_{\PHiggs}^{\textrm{test}(i)} \cdot \mbox{GeV}~{-1} \right)$ to the logarithm of the probability
density $P(\bm{p}^{\vis(1)},\bm{p}^{\vis(2)};\pX^{\rec},\pY^{\rec}|m_{\PHiggs}^{\textrm{test}(i)})$.

The distributions in $m_{\Pgt\Pgt}$ reconstructed by the
SVfitMEM algorithm with $\kappa = 5$ exhibit the best
resolution in the $\tauh\tauh$ channel and the worst resolution in the $\Pe\Pgm$ channel.
The difference in resolution between the $\tauh\tauh$, $\Pgm\tauh$,
and $\Pe\Pgm$ channels
is due to the fact that the fraction $z$ of
$\Pgt$ lepton energy that is carried by the visible $\Pgt$ decay
products is typically high in case of hadronic $\Pgt$ decays
and low in case of leptonic $\Pgt$ decays, \cf Fig.~\ref{fig:tauDecay_z}.
The presence of a $\tauh$ of high $\pT$ indicates the decay of a
$\PHiggs$ boson of high mass.
In case the $\pT$ of the visible decay products is high
for one $\Pgt$, the second $\Pgt$ either also has high $\pT$ or the
event is likely to exhibit a significant
imbalance in transverse momentum.
The imbalance in $\pX^{\miss}$ and $\pY^{\miss}$ enters the
SVfitMEM algorithm via Eq.~(\ref{eq:met}).
The presence of high $\pT$ neutrinos also indicates the decay of a $\PHiggs$ boson of high mass.
The low mass tail of the $m_{\Pgt\Pgt}$ distribution in the $\Pe\Pgm$
channel arises from events in which the electron as well as the muon
both have low $\pT$.
The $\Pgt$ leptons in $\PHiggs \to \Pgt\Pgt$ events produced via the
gluon fusion process are typically separated by $\Delta\phi \approx
\pi$ in the transverse plane,
causing the neutrinos produced in the $\Pgt$ decays to be emitted into
opposite hemispheres, with the effect that their contribution to $\pX^{\miss}$ and $\pY^{\miss}$ cancels.
Events featuring a low $\pT$ electron, a low $\pT$ muon and low $\MET$
are indistinguishable from the decays of $\PHiggs$ bosons of low mass
and are hence assigned low $m_{\Pgt\Pgt}$ values by the SVfitMEM algorithm.

In all three decay channels, the SVfitMEM algorithm significantly improves the separation of the $\PHiggs \to \Pgt\Pgt$ signal 
from the irreducible $\PZ/\Pggx \to \Pgt\Pgt$ DY background, yielding a substantial gain in analysis sensitivity
compared to alternative mass observables.
The increase in signal-to-background separation is illustrated in
Fig.~\ref{fig:distributions_mVis_vs_SVfit},
which compares the signal and background distributions for $m_{\Pgt\Pgt}$ and $m_{\vis}$,
the mass of the visible $\Pgt$ decay products.
Numerical values of the mean and root-mean-square (RMS) values
of the mass distributions reconstructed by the different methods are given in
Table~\ref{tab:resolutions_mVis_vs_SVfit}.
Compared to the classic SVfit algorithm, the SVfitMEM algorithm with $\kappa = 5$ improves the relative resolution,
quantified by the ratio of RMS to mean, by $10$--$20\%$.
The majority of the improvement is due to accounting for the experimental resolution on the $\pT$ of $\tauh$ via TF.

\begin{figure}
\setlength{\unitlength}{1mm}
\begin{center}
\begin{picture}(160,214)(0,0)
\put(-2.5, 150.0){\mbox{\includegraphics*[height=70mm]
  {plots/svFitPerformance_hadhad_visMass.pdf}}}
\put(80.0, 150.0){\mbox{\includegraphics*[height=70mm]
  {plots/svFitPerformance_hadhad_svFitMass.pdf}}}
\put(-2.5, 75.0){\mbox{\includegraphics*[height=70mm]
  {plots/svFitPerformance_muhad_visMass.pdf}}}
\put(80.0, 75.0){\mbox{\includegraphics*[height=70mm]
  {plots/svFitPerformance_muhad_svFitMass.pdf}}}
\put(-2.5, 0.0){\mbox{\includegraphics*[height=70mm]
  {plots/svFitPerformance_emu_visMass.pdf}}}
\put(80.0, 0.0){\mbox{\includegraphics*[height=70mm]
  {plots/svFitPerformance_emu_svFitMass.pdf}}}
\end{picture}
\end{center}
\caption{
  Distribution of $m_{\vis}$ (left) and of $m_{\Pgt\Pgt}$ reconstructed by the SVfitMEM algorithm with $\kappa = 5$ (right)
  in simulated $\PZ/\Pggx \to \Pgt\Pgt$ DY background and $\PHiggs \to
  \Pgt\Pgt$ signal events selected in the decay channels $\tauh\tauh$ (top), $\Pgm\tauh$ (center), and $\Pe\Pgm$ (bottom).
  The signal events are generated for $\PHiggs$ boson masses of $m_{\PHiggs} = 125$, $200$, and $300$~\GeV. 
}
\label{fig:distributions_mVis_vs_SVfit}
\end{figure}

\begin{table}
\begin{center}
\begin{tabular}{|l|cc|cc|}
\hline
\multicolumn{5}{|c|}{$\tauh\tauh$ decay channel} \\
\hline
\hline
\multirow{2}{17mm}{Sample} & \multicolumn{2}{c|}{$m_{\vis}$} & \multicolumn{2}{c|}{$m_{\Pgt\Pgt}$ (SVfitMEM, $\kappa = 3$)} \\
\cline{2-5}
 & Mean~[\GeV] & RMS/Mean & Mean~[\GeV] & RMS/Mean \\
\hline
$\PZ/\Pggx \to \Pgt\Pgt$ & $XXX.X$ & $0.XXX$ & $XXX.X$ & $0.XXX$ \\
$\PHiggs \to \Pgt\Pgt$: & & & & \\
 $\quad m_{\PHiggs} = 125$~\GeV & $XXX.X$ & $0.XXX$ & $XXX.X$ & $0.XXX$ \\
 $\quad m_{\PHiggs} = 200$~\GeV & $XXX.X$ & $0.XXX$ & $XXX.X$ & $0.XXX$ \\
 $\quad m_{\PHiggs} = 300$~\GeV & $XXX.X$ & $0.XXX$ & $XXX.X$ & $0.XXX$ \\
 $\quad m_{\PHiggs} = 500$~\GeV & $XXX.X$ & $0.XXX$ & $XXX.X$ & $0.XXX$ \\
 $\quad m_{\PHiggs} = 800$~\GeV & $XXX.X$ & $0.XXX$ & $XXX.X$ & $0.XXX$ \\
 $\quad m_{\PHiggs} = 1200$~\GeV & $XXX.X$ & $0.XXX$ & $XXX.X$ & $0.XXX$ \\ 
 $\quad m_{\PHiggs} = 1800$~\GeV & $XXX.X$ & $0.XXX$ & $XXX.X$ & $0.XXX$ \\ 
 $\quad m_{\PHiggs} = 2600$~\GeV & $XXX.X$ & $0.XXX$ & $XXX.X$ & $0.XXX$ \\ 
\hline
\end{tabular}

\vspace*{0.4 cm}

\begin{tabular}{|l|cc|cc|}
\hline
\multicolumn{5}{|c|}{$\Pgm\tauh$ decay channel} \\
\hline
\hline
\multirow{2}{17mm}{Sample} & \multicolumn{2}{c|}{$m_{\vis}$} & \multicolumn{2}{c|}{$m_{\Pgt\Pgt}$ (SVfitMEM, $\kappa = 3$)} \\
\cline{2-5}
 & Mean~[\GeV] & RMS/Mean & Mean~[\GeV] & RMS/Mean \\
\hline
$\PZ/\Pggx \to \Pgt\Pgt$ & $XXX.X$ & $0.XXX$ & $XXX.X$ & $0.XXX$ \\
$\PHiggs \to \Pgt\Pgt$: & & & & \\
 $\quad m_{\PHiggs} = 125$~\GeV & $XXX.X$ & $0.XXX$ & $XXX.X$ & $0.XXX$ \\
 $\quad m_{\PHiggs} = 200$~\GeV & $XXX.X$ & $0.XXX$ & $XXX.X$ & $0.XXX$ \\
 $\quad m_{\PHiggs} = 300$~\GeV & $XXX.X$ & $0.XXX$ & $XXX.X$ & $0.XXX$ \\
 $\quad m_{\PHiggs} = 500$~\GeV & $XXX.X$ & $0.XXX$ & $XXX.X$ & $0.XXX$ \\
 $\quad m_{\PHiggs} = 800$~\GeV & $XXX.X$ & $0.XXX$ & $XXX.X$ & $0.XXX$ \\
 $\quad m_{\PHiggs} = 1200$~\GeV & $XXX.X$ & $0.XXX$ & $XXX.X$ & $0.XXX$ \\ 
 $\quad m_{\PHiggs} = 1800$~\GeV & $XXX.X$ & $0.XXX$ & $XXX.X$ & $0.XXX$ \\ 
 $\quad m_{\PHiggs} = 2600$~\GeV & $XXX.X$ & $0.XXX$ & $XXX.X$ & $0.XXX$ \\ 
\hline
\end{tabular}

\vspace*{0.4 cm}

\begin{tabular}{|l|cc|cc|}
\hline
\multicolumn{5}{|c|}{$\Pe\Pgm$ decay channel} \\
\hline
\hline
\multirow{2}{17mm}{Sample} & \multicolumn{2}{c|}{$m_{\vis}$} & \multicolumn{2}{c|}{$m_{\Pgt\Pgt}$ (SVfitMEM, $\kappa = 2$)} \\
\cline{2-5}
 & Mean~[\GeV] & RMS/Mean & Mean~[\GeV] & RMS/Mean \\
\hline
$\PZ/\Pggx \to \Pgt\Pgt$ & $XXX.X$ & $0.XXX$ & $XXX.X$ & $0.XXX$ \\
$\PHiggs \to \Pgt\Pgt$: & & & & \\
 $\quad m_{\PHiggs} = 125$~\GeV & $XXX.X$ & $0.XXX$ & $XXX.X$ & $0.XXX$ \\
 $\quad m_{\PHiggs} = 200$~\GeV & $XXX.X$ & $0.XXX$ & $XXX.X$ & $0.XXX$ \\
 $\quad m_{\PHiggs} = 300$~\GeV & $XXX.X$ & $0.XXX$ & $XXX.X$ & $0.XXX$ \\
 $\quad m_{\PHiggs} = 500$~\GeV & $XXX.X$ & $0.XXX$ & $XXX.X$ & $0.XXX$ \\
 $\quad m_{\PHiggs} = 800$~\GeV & $XXX.X$ & $0.XXX$ & $XXX.X$ & $0.XXX$ \\
 $\quad m_{\PHiggs} = 1200$~\GeV & $XXX.X$ & $0.XXX$ & $XXX.X$ & $0.XXX$ \\ 
 $\quad m_{\PHiggs} = 1800$~\GeV & $XXX.X$ & $0.XXX$ & $XXX.X$ & $0.XXX$ \\ 
 $\quad m_{\PHiggs} = 2600$~\GeV & $XXX.X$ & $0.XXX$ & $XXX.X$ & $0.XXX$ \\ 
\hline
\end{tabular}
\end{center}
\caption{
  Mean and root-mean-square (RMS) of the $m_{\vis}$ and $m_{\Pgt\Pgt}$
  distributions
  reconstructed by the SVfitMEM algorithm
  in simulated $\PZ/\Pggx \to \Pgt\Pgt$ DY background and $\PHiggs \to
  \Pgt\Pgt$ signal events selected in the decay channels $\tauh\tauh$
  (top), $\Pgm\tauh$ (centre) and $\Pe\Pgm$ (bottom).
  The signal events are generated for different $\PHiggs$ boson masses $m_{\PHiggs}$.
  {\textbf CV: NUMBERS TO BE UPDATED !!}
}
\label{tab:resolutions_mVis_vs_SVfit}
\end{table}




\section{Summary}
\label{sec:summary}

An algorithm for reconstruction of the $\PHiggs$ boson mass in events
in which the $\PHiggs$ boson decays into a pair of $\Pgt$ leptons has been
presented.
The relative resolution on the $\PHiggs$ boson mass amounts to typically
$15$--$20\%$.
The algorithm has been used in data analyses performed by the CMS
collaboration during LHC run $1$.
It improves the sensitivity of the SM $\PHiggs \to \Pgt\Pgt$ analysisby about $40\%$,
corresponding to a gain in integrated luminosity by about a factor two.

Two improvements to the algorithm have been developed in preparation
for LHC run $2$.
The first concerns the rigorous formulation of the
algorithm in terms of the ME method.
The modelling of the experimental resolution on the $\pT$ of $\tauh$ via TF in the ME
formalism constitutes the second improvement.
Taken together, the two improvements enhance the relative resolution on the $\PHiggs$ boson mass
achieved by the algorithm by $5$--$10\%$.

The development of the formalism to handle $\Pgt$ lepton decays
in the ME method constitutes an important result of this paper.
It allows to extend the matrix elements generated by automatized tools such as
CompHEP or MadGraph by the capability to handle the
$\Pgt$ decays. We expect this to be very useful for future
applications of the ME method to data analyses with $\Pgt$ leptons in
the final state. 


\section*{Acknowledgements}

JC and CV wish to acknowledge the collaboration with our former colleague Evan Friis,
who made vital contributions to the early stage of this project. LM acknowledges the Estonian Research Council for supporting his work with the grant PUTJD110. 


\clearpage

\section{Appendix}
\label{sec:appendix}

We derive here the relations for the product of the squared moduli of the ME and the phase space elements
$d\Phi^{(i)}_{\tauhnu}$ and $d\Phi^{(i)}_{\ellnunu}$ for,
respectively, the decays $\Pgt \to \textrm{hadrons} + \Pnut$ and $\Pgt \to \ellnunu$, given by Eq.~(\ref{eq:PSint}).
We start with the simpler case of hadronic $\Pgt$ decays in
Section~\ref{sec:appendix_tauToHadDecays} and turn to the more complex
case of leptonic $\Pgt$ decays in
Section~\ref{sec:appendix_tauToLepDecays}.
For clarity of notation, we omit the hat symbol in this section and
use the convention that all symbols refer to true values in the laboratory frame, unless indicated
explicitely otherwise.
\subsection{The decay $\Pgt \to \textrm{hadrons} + \Pnut$}
\label{sec:appendix_tauToHadDecays}

We treat hadronic $\Pgt$ decays as a two-body decay into a hadronic
system $\tauh$ plus a tau neutrino,
as explained in Section~\ref{sec:mem_ME},
and take the squared modulus of the ME to be a constant,
which we denote by $\vert\mathcal{M}^{\eff}_{\Pgt \to \tauhnu}\vert^{2}$.
We further denote the momentum of the neutrino produced in the $\Pgt$ decay by
$\bm{p}^{\inv}$ and its energy by $E_{\inv}$ (\cf Section~\ref{sec:mem_PSintegration}).
For reasons that will become clear later, we allow the neutrino to
have non-zero mass $m_{\inv}$.

The product of the squared modulus of the ME and the phase space
element $d\Phi^{(i)}_{\tauhnu}$ reads:
\begin{align}
 & \, \vert \BW_{\Pgt} \vert^{2} \cdot \vert\mathcal{M}_{\textrm{decay}}\vert^{2} \,
 d\Phi_{\tauhnu} = \vert \BW_{\Pgt} \vert^{2} \cdot \vert \mathcal{M}^{(i)}_{\textrm{decay}}
\vert^{2} \, \frac{d^{3}\bm{p}^{\vis}}{(2\pi)^{3} \, 2
   E_{\vis}} \, \frac{d^{3}\bm{p}^{\inv}}{(2\pi)^{3} \, 2 E_{\inv}} \nonumber \\
= & \, (2\pi)^{3} \, \int \, \frac{\pi}{m_{\Pgt} \, \Gamma_{\Pgt}} \,
\delta ( q_{\Pgt}^{2} - m_{\Pgt}^{2} ) \cdot \vert\mathcal{M}^{\eff}_{\Pgt \to
  \tauh\Pnut}\vert^{2} \, \delta \left( E_{\Pgt} - E_{\vis} -
  E_{\inv} \right) \, \delta^{3} \left( \bm{p}^{\Pgt} - \bm{p}^{\vis}
  - \bm{p}^{\inv} \right) \nonumber \\
& \qquad \frac{d^{3}\bm{p}^{\Pgt}}{(2\pi)^{3} \, 2 E_{\Pgt}} \, 
  \frac{d^{3}\bm{p}^{\vis}}{(2\pi)^{3} \, 2E_{\vis}} \, \frac{d^{3}\bm{p}^{\inv}}{(2\pi)^{3} \, 2 E_{\inv}} \, dq^{2}_{\Pgt} \nonumber \\
= & \frac{8\pi^{4}}{m_{\Pgt} \, \Gamma_{\Pgt}} \, \vert\mathcal{M}^{\eff}_{\Pgt \to
  \tauh\Pnut}\vert^{2} \, \delta \left( E_{\Pgt} - E_{\vis} -
  E_{\inv} \right) \, \delta^{3} \left( \bm{p}^{\Pgt} - \bm{p}^{\vis}
  - \bm{p}^{\inv} \right) \nonumber \\
& \qquad \frac{d^{3}\bm{p}^{\Pgt}}{(2\pi)^{3} \, 2 E_{\Pgt}} \, 
  \frac{d^{3}\bm{p}^{\vis}}{2 E_{\vis}} \, \frac{d^{3}\bm{p}^{\inv}}{2
    E_{\inv}} \nonumber \\
= & \, \frac{\pi}{m_{\Pgt} \, \Gamma_{\Pgt}} \, \frac{\vert\mathcal{M}^{\eff}_{\Pgt \to
  \tauh\Pnut}\vert^{2}}{(2\pi)^{6}} 
 \cdot \frac{1}{2 E_{\Pgt}(\bm{p}^{\vis}, \bm{p}^{\inv})} \, \delta
 \left( E_{\Pgt}(\bm{p}^{\vis}, \bm{p}^{\inv}) - E_{\vis} - E_{\inv}
 \right) \nonumber \\
& \qquad
  \frac{d^{3}\bm{p}_{\vis}}{2 E_{\vis}} \, \frac{\vert\bm{p}_{\inv}\vert}{2} \, dE_{\inv} \, d\cos\theta_{\inv} \, d\phi_{\inv} \, ,
\label{eq:hadTauDecaysPSint}
\end{align}
where we have used the formula for recursive phase space generation,
given by Eq.~(43.12) in Ref.~\cite{PDG}, for transforming the first line into the second
and the identity:
\begin{equation} 
d^{3}\bm{p}^{\inv} = \vert\bm{p}^{\inv}\vert^{2} \,
dp^{inv} \, d\cos\theta_{\inv} \, d\phi_{\inv} =
\vert\bm{p}^{\inv}\vert \, E_{\inv} \, dE_{\inv} \, d\cos\theta_{\inv}
\, d\phi_{\inv}
\end{equation} 
for rewriting the third line by the fourth.
The factor $\BW_{\Pgt} \vert^{2} = \frac{\pi}{m_{\Pgt} \,
  \Gamma_{\Pgt}} \, \delta ( q^{2}_{\Pgt} - m^{2}_{\Pgt} )$ removes
the integration over $dq^{2}_{\Pgt}$, enforcing the $\Pgt$ lepton
energy and momentum to be related by $E_{\Pgt} =
\sqrt{\vert\bm{p}^{\Pgt}\vert^{2} + m_{\Pgt}^{2}}$.
The symbol $E_{\Pgt}(\bm{p}^{\vis}, \bm{p}^{\inv})$
indicates that $E_{\Pgt}$ is a function of $\bm{p}^{\vis}$
and $\bm{p}^{\inv}$, as is neccessary to satisfy the $
\delta$-function $\delta^{3} ( \bm{p}^{\Pgt} - \bm{p}^{\vis} - \bm{p}^{\inv} )$.

We define $z = E_{\vis}/E_{\Pgt}$ according to Eq.~(\ref{eq:def_z}) and replace the integration over $dE_{\inv}$ by an integration over $z$.
The Jacobi factor related to this transformation is:
\begin{equation}
E_{\inv} = E_{\Pgt} - E_{\vis} = \left( 1 - z \right)
\, E_{\Pgt} = \frac{1 - z}{z} \, E_{\vis}
  \quad \Longleftrightarrow \quad dE_{\inv} = \frac{\partial E_{\inv}}{\partial z} \, dz = \frac{E_{\vis}}{z^{2}} \, dz \, .
\label{eq:hadTauDecaysJacobi}
\end{equation}

We then perform the integration over $d\cos\theta_{\inv}$.
As in Section~\ref{sec:mem_PSintegration}, we choose the coordinate system such that
$\theta_{\inv}$ is equal to the angle between the $\bm{p}^{\vis}$ and $\bm{p}^{\inv}$ vectors.
The $\delta$-function $\delta \left( E_{\Pgt}(\bm{p}^{\vis}, \bm{p}^{\inv}) - E_{\vis} - E_{\inv} \right)$ depends on $\cos\theta_{\inv}$ via:
\begin{align}
& \, E_{\Pgt}(\bm{p}^{\vis}, \bm{p}^{\inv}) 
= \sqrt{\vert\bm{p}^{\Pgt}\vert^{2} + m^{2}_{\Pgt}} = \sqrt{\left( \bm{p}^{\vis} + \bm{p}^{\inv} \right)^2 + m^{2}_{\Pgt}} \nonumber \\
& \quad = \sqrt{\vert\bm{p}^{\vis}\vert^{2} + \vert\bm{p}^{\inv}\vert^{2} + 2 \bm{p}^{\vis}
  \cdot \bm{p}^{\inv} + m^{2}_{\Pgt}} \nonumber \\
& \quad = \sqrt{\vert\bm{p}^{\vis}\vert^{2} + \vert\bm{p}^{\inv}\vert^{2} + 2 \vert\bm{p}^{\vis}\vert \, \vert\bm{p}^{\inv}\vert \, \cos\theta_{\inv} + m^{2}_{\Pgt}}.
\end{align}
The $\delta$-function argument vanishes if $E_{\Pgt}(\bm{p}^{\vis}, \bm{p}^{\inv}) - E_{\vis} - E_{\inv} = 0$. 
This yields:
\begin{equation}
\cos\theta_{\inv} 
  = \frac{E_{\vis} E_{\inv} - \frac{1}{2} \left(m^{2}_{\Pgt} - \left(
        m^{2}_{\vis} + m^{2}_{\inv} \right)
    \right)}{\vert\bm{p}^{\vis}\vert \, \vert\bm{p}^{\inv}\vert} \, .
\label{eq:hadTauDecaysCosTheta}
\end{equation}

When substituting the expressions of Eqs.~(\ref{eq:hadTauDecaysJacobi}) and~(\ref{eq:hadTauDecaysCosTheta}) into Eq.~(\ref{eq:hadTauDecaysPSint}),
we need to account for the $\delta$-function rule:
\begin{equation} 
\delta \left( g(x) \right) = \sum_{k} \frac{\delta \left( x - x_{k}
  \right)}{\vert g'(x_{k}) \vert} \, .
\label{eq:deltaFuncRule}
\end{equation}
We identify:
\begin{equation} 
g(\cos\theta_{\inv}) = \sqrt{\vert\bm{p}^{\vis}\vert^{2} + \vert\bm{p}^{\inv}\vert^{2}
  + 2 \vert\bm{p}^{\vis}\vert \, \vert\bm{p}^{\inv}\vert \,
  \cos\theta_{\inv} + m^{2}_{\Pgt}} - E_{\vis} - E_{\inv}
\end{equation}
and obtain $\vert g'(x_{0}) \vert = \vert\bm{p}^{\vis}\vert \,
\vert\bm{p}^{\inv}\vert / \left( E_{\vis} + E_{\inv} \right) = \vert\bm{p}^{\vis}\vert \,
\vert\bm{p}^{\inv}\vert / E_{\Pgt}$.

This yields:
\begin{equation}
 \vert\mathcal{M}_{\Pgt}\vert^{2} \,
 d\Phi^{(i)}_{\tauhnu} = \frac{\vert\mathcal{M}^{\eff}_{\Pgt \to
  \tauh\Pnut}\vert^{2}}{256\pi^{5}\, m_{\Pgt} \, \Gamma_{\Pgt}} \cdot 
    \frac{1}{\vert\bm{p}^{\vis}\vert \, z^{2}} \, 
    \frac{d^{3}\bm{p}^{\vis}}{2 E_{\vis}} \, dz \, d\phi_{\inv} \, .
\label{eq:hadTauDecaysResult}
\end{equation}

We define:
\begin{equation}
f_{h}\left(\bm{p}^{\vis}, m_{\vis}, \bm{p}^{\inv}\right) = 
  \frac{\vert\mathcal{M}^{\eff}_{\Pgt \to
  \tauh\Pnut}\vert^{2}}{512\pi^{6} \, \vert\bm{p}^{\vis}\vert \, z^{2}} 
\label{eq:hadTauDecays_f}
\end{equation}
to obtain:
\begin{equation}
\vert\mathcal{M}_{\Pgt}\vert^{2} \,
 d\Phi^{(i)}_{\tauhnu} = \frac{\pi}{m_{\Pgt} \, \Gamma_{\Pgt}} \,
 f_{h}(\bm{p}^{\vis}, m_{\vis}, \bm{p}^{\inv}) \, \frac{d^{3}\bm{p}^{\vis}}{2 E_{\vis}} \, dz \, d\phi_{\inv}
 \, ,
\end{equation}
which is the result that we quote in Eq.~(\ref{eq:PSint}).

\subsection{The decays $\Pgt \to \enunu$ and $\Pgt \to \mununu$}
\label{sec:appendix_tauToLepDecays}

We treat leptonic $\Pgt$ decays as three-body decays and take
into account the ME, assuming the taus to be unpolarized.
The squared modulus of the ME is given by~\cite{Barger:1987nn}:
\begin{equation}
\vert\mathcal{M}_{\Pgt \to \ellnunu} \vert^{2} = 128 \, G^{2}_{F} \,
\left( E_{\Pgt} E_{\APnu} - \bm{p^{\Pgt}} \cdot \bm{p^{\APnu}} \right)
\, \left( E_{\Plepton} E_{\Pnu} - \bm{p^{\Plepton}} \cdot \bm{p^{\Pnu}} \right) \, , 
\label{eq:lepTauDecaysME}
\end{equation}
where $G_{F}$ denotes the Fermi constant, given by Eq.~(\ref{eq:def_G_F}).

The product of the squared modulus of the ME and the phase space
element $d\Phi_{\ellnunu}$ reads:
\begin{align}
& \, \vert\mathcal{M}_{\Pgt}\vert^{2} \,
 d\Phi_{\ellnunu} = (2\pi)^{3} \, \vert \BW_{\Pgt} \vert^{2} \cdot
\vert\mathcal{M}^{\eff}_{\Pgt \to \ellnunu}\vert^{2} \, \delta \left(
  E_{\Pgt} - E_{\Plepton} - E_{\Pnu} - E_{\APnu} \right) \nonumber \\
& \qquad
\delta^{3} \left( \bm{p^{\Pgt}} - \bm{p^{\Plepton}} - \bm{p^{\Pnu}} - \bm{p^{\APnu}} \right) \, \frac{d^{3}\bm{p^{\Pgt}}}{(2\pi)^{3} \, 2 E_{\Pgt}} \,
  \frac{d^{3}\bm{p^{\Plepton}}}{(2\pi)^{3} \, 2 E_{\Plepton}} \, 
  \frac{d^{3}\bm{p^{\Pnu}}}{(2\pi)^{3} \, 2 E_{\Pnu}} \, 
  \frac{d^{3}\bm{p^{\APnu}}}{(2\pi)^{3} \, 2 E_{\APnu}} \, dq^{2}_{\Pgt} \, \vert\mathcal{M}_{\Pgt \to
  \ellnunu}\vert^{2} \nonumber \\
= & \frac{8\pi^{4}}{m_{\Pgt} \, \Gamma_{\Pgt}} \, \vert\mathcal{M}^{\eff}_{\Pgt \to
  \ellnunu}\vert^{2} \, \delta \left( E_{\Pgt} - E_{\Plepton} -
  E_{\Pnu} - E_{\APnu} \right) \, \delta^{3} \left( \bm{p^{\Pgt}} -
  \bm{p^{\Plepton}} - \bm{p^{\Pnu}} - \bm{p^{\APnu}} \right)  \nonumber \\
& \qquad
  \frac{d^{3}\bm{p^{\Pgt}}}{(2\pi)^{3} \, 2 E_{\Pgt}} \,
  \frac{d^{3}\bm{p^{\Plepton}}}{(2\pi)^{3} \, 2 E_{\Plepton}} \,
  \frac{dE_{\Pnu} \, d^{3}\bm{p^{\Pnu}}}{(2\pi)^{3}} \, \theta(E_{\Pnu}) \, \delta \left( \vert\bm{p^{\Pnu}}\vert^{2} \right) \, 
  \frac{dE_{\APnu} \, d^{3}\bm{p^{\APnu}}}{(2\pi)^{3}} \, \theta(E_{\APnu}) \, \delta \left( \vert\bm{p^{\APnu}}\vert^{2} \right) \, . \nonumber 
\end{align}
For expressing the second line by the third,
we have used the identity:
\begin{equation}
\int \, dE \, d^{3}\bm{p} \, \theta(E) \, \delta \left( E^{2} -
  \vert\bm{p}\vert^{2} - m^{2} \right) = \int \, \frac{d^{3}\bm{p}}{2
  \, E} \, ,
\end{equation}
which follows from the delta function rule Eq.~(\ref{eq:deltaFuncRule}).
We assume that the mass of $\Pnu$ as well as the mass of $\APnu$ is zero.

We perform a variable transformation from $(E_{\Pnu}, \bm{p^{\Pnu}})$
and $(E_{\APnu}, \bm{p^{\APnu}})$ to:
\begin{align}
(u_{0}, \bm{u}) = & \, (E_{\Pnu} + E_{\APnu}, \bm{p^{\Pnu}} +
\bm{p^{\APnu}}) \, , \qquad (v_{0}, \bm{v}) = \frac{1}{2} (
  E_{\Pnu} - E_{\APnu}, \bm{p^{\Pnu}} - \bm{p^{\APnu}} ) \, . \nonumber 
\end{align}
The variables $u_{0}$ and $\bm{u}$ represent the energy and momentum of the neutrino pair.
The determinant of the Jacobi matrix for the transformation from
$(E_{\Pnu}, \bm{p^{\Pnu}}; E_{\APnu}, \bm{p^{\APnu}})$
to $(u_{0}, \bm{u}; v_{0}, \bm{v})$ equals unity.
Expressed in the new variables, the energy and momenta of the neutrino and anti-neutrino produced in the tau decay are given by:
\begin{align}
( E_{\Pnu}, \bm{p^{\Pnu}} ) = & \, \frac{1}{2} ( u_{0} + v_{0}, \bm{u}
+ \bm{v} ) \, \mbox{ and } \, ( E_{\APnu}, \bm{p^{\APnu}} ) = \frac{1}{2} (
u_{0} - v_{0}, \bm{u} - \bm{v} ) \, . \nonumber 
\end{align}

The product of the squared modules of the ME and the phase space
element can then be expressed by:
\begin{align}
& \, \vert\mathcal{M}_{\Pgt}\vert^{2} \,
 d\Phi_{\ellnunu} = \frac{8\pi^{4}}{m_{\Pgt} \, \Gamma_{\Pgt}} \, \vert\mathcal{M}^{\eff}_{\Pgt \to
  \ellnunu}\vert^{2} \delta \left( E_{\Pgt} - E_{\Plepton} -
  E_{\Pnu} - E_{\APnu} \right) \, \delta^{3} \left( \bm{p^{\Pgt}} -
  \bm{p^{\Plepton}} - \bm{p^{\Pnu}} - \bm{p^{\APnu}} \right) \nonumber
\\
& \qquad
  \frac{d^{3}\bm{p^{\Pgt}}}{(2\pi)^{3} \, 2 E_{\Pgt}} \,
  \frac{d^{3}\bm{p^{\Plepton}}}{(2\pi)^{3} \, 2 E_{\Plepton}} \,
  \frac{du_{0} \, d^{3}\bm{u}}{(2\pi)^{3}} \nonumber \\
 & \qquad
  \theta(\frac{1}{2} u_{0} + v_{0}) \, \delta \left( \frac{u_{0}^{2} -
      \bm{u}^{2}}{4} + v_{0}^{2} - \bm{v}^{2} + u_{0} \, v_{0} - \bm{u} \cdot \bm{v} \right) \, 
  \frac{dv_{0} \, d^{3}\bm{v}}{(2\pi)^{3}} \nonumber \\
 & \qquad
  \theta(\frac{1}{2} u_{0}
  - v_{0}) \, \delta \left( \frac{u_{0}^{2} - \vert\bm{u}\vert^{2}}{4} +
    v_{0}^{2} - \vert\bm{v}\vert^{2} - u_{0} \, v_{0} + \bm{u}
    \cdot \bm{v} \right) \nonumber \\
= & \frac{8\pi^{4}}{m_{\Pgt} \, \Gamma_{\Pgt}} \, \delta \left( E_{\Pgt} - E_{\Plepton} -
  E_{\Pnu} - E_{\APnu} \right) \, \delta^{3} \left( \bm{p^{\Pgt}} -
  \bm{p^{\Plepton}} - \bm{p^{\Pnu}} - \bm{p^{\APnu}} \right) \, 
\frac{d^{3}\bm{p^{\Pgt}}}{(2\pi)^{3} \, 2 E_{\Pgt}} \,
\frac{d^{3}\bm{p^{\Plepton}}}{(2\pi)^{3} \, 2 E_{\Plepton}} \,
\frac{du_{0} \, d^{3}\bm{u}}{(2\pi)^{3}} \nonumber \\
 & \qquad
  \vert\mathcal{M}^{\eff}_{\Pgt \to
  \ellnunu}\vert^{2} \, \theta(\frac{1}{2} u_{0} + v_{0}) \, \delta
\left( \frac{u_{0}^{2} - \bm{u}^{2}}{4} + v_{0}^{2} - \bm{v}^{2} +
  u_{0} \, v_{0} - \bm{u} \cdot \bm{v} \right) \, 
  \frac{dv_{0} \, d^{3}\bm{v}}{(2\pi)^{3}} \nonumber \\
 & \qquad
  \theta(\frac{1}{2} u_{0}
  - v_{0}) \, \delta \left( \frac{u_{0}^{2} - \vert\bm{u}\vert^{2}}{4} +
    v_{0}^{2} - \vert\bm{v}\vert^{2} - u_{0} \, v_{0} + \bm{u}
    \cdot \bm{v} \right) \, .
\label{eq:lepTauDecaysPSint}
\end{align}
We have used the relation $\delta(a + b) \, \delta(a - b) = \delta(2b) \, \delta(a - b) = \frac{1}{2} \, \delta(b) \, \delta(a)$ 
to express the second line by the third.

We define:
\begin{align}
& \, I_{\inv} = \vert\mathcal{M}^{\eff}_{\Pgt \to
  \ellnunu}\vert^{2} \, \theta(\frac{1}{2} u_{0} + v_{0}) \, \delta
\left( \frac{u_{0}^{2} - \bm{u}^{2}}{4} + v_{0}^{2} - \bm{v}^{2} +
  u_{0} \, v_{0} - \bm{u} \cdot \bm{v} \right) \, 
  \frac{dv_{0} \, d^{3}\bm{v}}{(2\pi)^{3}} \nonumber \\
 & \qquad
  \theta(\frac{1}{2} u_{0}
  - v_{0}) \, \delta \left( \frac{u_{0}^{2} - \vert\bm{u}\vert^{2}}{4} +
    v_{0}^{2} - \vert\bm{v}\vert^{2} - u_{0} \, v_{0} + \bm{u}
    \cdot \bm{v} \right) \, .
\label{eq:def_Iinv}
\end{align}
The quantity $I_{\inv}$ is a Lorentz invariant quantity. 
As such, it can be computed in any frame and will yield the same value as in the laboratory frame.
We choose to evaluate it in the restframe of the di-neutrino system.
In this frame, the energy $u_{0}$ is given by $u_{0} = m_{\inv}$ 
and the momentum $\bm{u}$ by $\bm{u} = ( 0, 0, 0 )$, with $m_{\inv}$ denoting
the mass of the di-neutrino system.
Hence $u_{0} \, v_{0} - \bm{u} \cdot \bm{v} = m_{\inv} \, v_{0} $ in this frame.
Performing the integration over $v_{0}$, we obtain:
\begin{align}
I_{\inv}
= & \, \vert\mathcal{M}^{\eff}_{\Pgt \to
  \ellnunu}\vert^{2} \, \frac{1}{2} \, \frac{dv_{0} \, d^{3}\bm{v}}{(2\pi)^{3}} \, \theta ( \frac{1}{2} \, u_{0} + v_{0} ) \, 
    \delta \left( \underbrace{\frac{u_{0}^{2} - \vert\bm{u}\vert^{2}}{4}}_{=
        \frac{m^{2}_{\inv}}{4}} + \underbrace{v_{0}^{2} -
        \vert\bm{v}\vert^{2}}_{= -\vert\bm{v}\vert^{2}} \right)
    \nonumber \\
& \qquad
    \theta ( \frac{1}{2} u_{0} - v_{0} ) \, \underbrace{\delta \left(
        u_{0} \, v_{0} - \bm{u} \cdot \bm{v} \right)}_{= \frac{1}{m_{\inv}} \, \delta ( v_{0} )} \nonumber \\
= & \, \frac{1}{2 m_{\inv}} \, \underbrace{\theta ( \frac{1}{2} \, u_{0} )}_{= \theta ( u_{0} )} \int \, \frac{d^{3}\bm{v}}{(2\pi)^{3}} \, 
  \vert\mathcal{M}_{\Pgt \to
  \ellnunu}\vert^{2} \, \delta \left( \frac{m^{2}_{\inv}}{4} - \vert\bm{v}\vert^{2} \right) \nonumber \\
= & \, \frac{1}{2 m_{\inv}} \, \theta ( u_{0} ) \, \int \, \frac{\vert\bm{v}\vert^2 d\vert\bm{v}\vert d\Omega_{v}}{(2\pi)^{3}} \, 
  \vert\mathcal{M}_{\Pgt \to
  \ellnunu}\vert^{2} \, \underbrace{\delta \left( \frac{m^{2}_{\inv}}{4} - \vert\bm{v}\vert^{2} \right)}_{
    = \frac{1}{2 \vert\bm{v}\vert} \, \delta \left( \vert\bm{v}\vert - \frac{m_{\inv}}{2} \right)} \nonumber \\
= & \, \frac{1}{2 m_{\inv}} \, \theta ( u_{0} ) \, \int \, \frac{\vert\bm{v}\vert^2 d\vert\bm{v}\vert d\Omega_{v}}{(2\pi)^{3}} \, 
  \vert\mathcal{M}_{\Pgt \to
  \ellnunu}\vert^{2} \, \frac{1}{2 \vert\bm{v}\vert} \, \delta \left( \vert\bm{v}\vert - \frac{m_{\inv}}{2} \right) \nonumber \\
= & \, \frac{1}{8} \, \theta ( u_{0} ) \, \int \, \frac{d\Omega_{v}}{(2\pi)^{3}} \, \vert\mathcal{M}_{\Pgt \to
  \ellnunu}\vert^{2} \, . 
\label{eq:lepTauDecaysI}
\end{align}

In the restframe of the neutrino pair:
\begin{align}
( E_{\Pgt}, \bm{p^{\Pgt}} ) = & \, ( E_{\Pgt}, 0, 0, \vert\bm{p^{\Pgt}}\vert ) \nonumber \\
( E_{\Plepton}, \bm{p^{\Plepton}} ) = & \, ( E_{\Plepton}, 0, 0, \vert\bm{p^{\Plepton}}\vert ) \nonumber \\
( E_{\Pnu}, \bm{p^{\Pnu}} ) = & \, \frac{m_{\inv}}{2} \, ( 1, 0, \sin\theta, \cos\theta ) \nonumber \\
( E_{\APnu}, \bm{p^{\APnu}} ) = & \, \frac{m_{\inv}}{2} \, ( 1, 0, -\sin\theta, -\cos\theta ) \, , \nonumber 
\end{align}
where we have chosen the polar axis such that it is parallel to $\bm{p^{\Plepton}}$.

The ME given by Eq.~(\ref{eq:lepTauDecaysME}) evaluates to:
\begin{align}
\vert\mathcal{M}_{\Pgt \to \ellnunu}\vert^{2} 
 = & \, 128 \, G^{2}_{F} \, 
  \underbrace{\left( E_{\Pgt} \, E_{\Plepton} - \bm{p^{\Pgt}} \cdot \bm{p^{\APnu}} \right)}_{= E_{\Pgt} \, \frac{m_{\inv}}{2} + \vert\bm{p^{\Pgt}}\vert \, \frac{m_{\inv}}{2} \cos\theta \quad} \,
  \underbrace{\left( E_{\Plepton} \, E_{\Pnu} - \bm{p^{\Plepton}} \cdot \bm{p^{\Pnu}} \right)}_{= E_{\Plepton} \, \frac{m_{\inv}}{2} - \vert\bm{p^{\Plepton}}\vert \, \frac{m_{\inv}}{2} \cos\theta} \nonumber \\
 = & \, 32 \, G^{2}_{F} \, m^{2}_{\inv} \, \left( E_{\Pgt} + \vert\bm{p^{\Pgt}}\vert \, \cos\theta \right)  \left( E_{\Plepton} - \vert\bm{p^{\Plepton}}\vert \, \cos\theta \right) 
\label{eq:lepTauDecaysMErf}
\end{align}
in the restframe of the di-neutrino system.

The energy of the $\Pgt$ lepton and of the electron or muon are given
by the relation:
\begin{equation}
m^{2}_{\Plepton} = E_{\Plepton}^{2} - \vert\bm{p^{\Plepton}}\vert^{2} = \left( p_{\Pgt} - u \right)^{2} 
  = m^{2}_{\Pgt} + m^{2}_{\inv} - 2 (E_{\Pgt} \, u_{0} - \bm{p^{\Pgt}} \cdot \bm{u}) 
  = m^{2}_{\Pgt} + m^{2}_{\inv} - 2 m_{\inv} E_{\Pgt}
\end{equation}
from which it follows that:
\begin{align}
E_{\Pgt} = & \, \frac{m^{2}_{\Pgt} + m^{2}_{\inv} - m^{2}_{\Plepton}}{2 m_{\inv}} \nonumber \\
E_{\Plepton} = & \, E_{\Pgt} - m_{\inv} = \frac{m^{2}_{\Pgt} -
  m^{2}_{\inv} - m^{2}_{\Plepton}}{2 m_{\inv}} \, .
\label{eq:lepTauDecaysEn}
\end{align}

Substituting Eq.~(\ref{eq:lepTauDecaysMErf}) into Eq.~(\ref{eq:lepTauDecaysI}) yields:
\begin{align}
I_{\inv} 
= & \, \frac{1}{8} \, \theta ( u_{0} ) \, \int \, \frac{d\Omega_{v}}{(2\pi)^{3}} \, \vert\mathcal{M}_{\Pgt \to
  \ellnunu}\vert^{2} \nonumber \\
= & \, 4 \, G^{2}_{F} \, m^{2}_{\inv} \, \theta ( u_{0} ) \, \int \, \frac{d\cos\theta \, d\phi}{(2\pi)^{3}} \, 
  \left( E_{\Pgt} + \vert\bm{p^{\Pgt}}\vert \, \cos\theta \right)  \left( E_{\Plepton} - \vert\bm{p^{\Plepton}}\vert \, \cos\theta \right) \nonumber \\
= & \, \frac{G^{2}_{F}}{2 \pi^{3}} \, m^{2}_{\inv} \, \theta ( u_{0} ) \, \underbrace{\int_{0}^{2 \pi} \, d\phi}_{= 2 \pi} \, 
  \left( E_{\Pgt} \, E_{\Plepton} \, \underbrace{\int_{-1}^{+1}
      d\cos\theta}_{= 2} \right. \nonumber \\
& \qquad
     \left. + \left( \vert\bm{p^{\Pgt}}\vert \, E_{\Plepton} - E_{\Pgt} \,
       \vert\bm{p^{\Plepton}}\vert \right) \,
     \underbrace{\int_{-1}^{+1} \, d\cos\theta \, \cos\theta}_{= 0} 
  - \vert\bm{p_{\Pgt}}\vert \, \vert\bm{p_{\Plepton}}\vert
       \, \underbrace{\int_{-1}^{+1} \, d\cos\theta \, \cos^{2}\theta}_{= \frac{2}{3}} \right) \nonumber \\
= & \, \frac{G^{2}_{F}}{\pi^{2}} \, m^{2}_{\inv} \, \theta ( u_{0} ) \, 
  \left( 2 \, E_{\Pgt} \, E_{\Plepton} - \frac{2}{3} \, \sqrt{E^{2}_{\Pgt} - m^{2}_{\Pgt}} \, \sqrt{E^{2}_{\Plepton} - m^{2}_{\Plepton}} \right) \, ,
\label{eq:lepTauDecaysIrf}
\end{align}
where $E_{\Pgt}$ and $E_{\Plepton}$ are given by Eq.~(\ref{eq:lepTauDecaysEn}).
Note that the value of $I
_{\inv}$ is a function of $m_{\inv}$ solely,
as $m_{\Pgt}$ and $m_{\Plepton}$ are constants.

Before substituting Eq.~(\ref{eq:lepTauDecaysIrf}) into Eq.~(\ref{eq:lepTauDecaysPSint}),
we perform a variable transformation from $( u_{0}, u_{1}, u_{2}, u_{3} )$ to $( m^{2}_{\inv}, u_{1}, u_{2}, u_{3} ) = ( {u_{0}}^{2} - {u_{1}}^{2} - {u_{2}}^{2} - {u_{3}}^{2}, u_{1}, u_{2}, u_{3} )$.
The determinant of the Jacobi for this transformation is $\vert J
\vert = 2 \, u_{0}$,
from which it follows that:
\begin{equation}
du_{0} \, d^{3}\bm{u} = \frac{1}{\vert J \vert} \, dm^{2}_{\inv} \,
d^{3}\bm{u} = \frac{1}{2 u_{0}} \, dm^{2}_{\inv} \, d^{3}\bm{u} \, .
\label{eq:lepTauDecaysJacobi}
\end{equation}

Substituting Eqs.~(\ref{eq:lepTauDecaysIrf}) and~(\ref{eq:lepTauDecaysJacobi}) into Eq.~(\ref{eq:lepTauDecaysPSint}), we obtain:
\begin{align}
& \, \vert\mathcal{M}_{\Pgt}\vert^{2} \,
 d\Phi_{\ellnunu} = \frac{8\pi^{4}}{m_{\Pgt} \, \Gamma_{\Pgt}} \,
 \delta \left( E_{\Pgt} - E_{\Plepton} - E_{\Pnu} - E_{\APnu} \right)
 \, \delta^{3} \left( \bm{p^{\Pgt}} - \bm{p^{\Plepton}} -
  \bm{p^{\Pnu}} - \bm{p^{\APnu}} \right) \nonumber \\
& \qquad \frac{d^{3}\bm{p^{\Pgt}}}{(2\pi)^{3} \, 2 E_{\Pgt}} \,
  \frac{d^{3}\bm{p^{\Plepton}}}{(2\pi)^{3} \, 2 E_{\Plepton}} \, 
  \frac{d^{3}\bm{u}}{(2\pi)^{3} \, 2 u_{0}} \, I_{\inv} \,
  dm^{2}_{\inv} \, ,
\label{eq:finalLepTauDecaysPSint}
\end{align}
with $u_{0} \equiv E_{\inv}$ and $\bm{u} \equiv \bm{p^{\inv}}$.
The expression in Eq.~(\ref{eq:finalLepTauDecaysPSint}) is identical in structure to the second line of Eq.~(\ref{eq:hadTauDecaysPSint}),
if we identify the integration over the momentum of the neutrino pair,
given by the phase space element $d^{3}\bm{u}$, in Eq.~(\ref{eq:finalLepTauDecaysPSint}) 
with the integration over the neutrino momentum $d^{3}\bm{p^{\inv}}$ in Eq.~(\ref{eq:hadTauDecaysPSint}).
The remaining differences between the formulae for leptonic $\Pgt$ decays compared to hadronic $\Pgt$ decays
are the additional integration over $dm^{2}_{\inv}$ and the
factor $I_{\inv}$ in Eq.~(\ref{eq:finalLepTauDecaysPSint}) that replaces the
factor $\vert\mathcal{M}^{\textrm{eff}}_{\Pgt \to
  \tauh\Pnut}\vert^{2}$ in Eq.~(\ref{eq:hadTauDecaysPSint}).
Note that Eq.~(\ref{eq:finalLepTauDecaysPSint}) as well as Eq.~(\ref{eq:hadTauDecaysPSint}) refer to the laboratory frame.
The restframe of the di-neutrino system was used only for the purpose of evaluating the Lorentz invariant integral $I_{\inv}$.

Since $I_{\inv}$ depends solely on the integration variable $m_{\inv}$, 
$I_{\inv}$ can be considered as constant when performing the integration over $d^{3}\bm{p^{\Pgt}}$ and $d^{3}\bm{u}$.
We can hence use Eq.~(\ref{eq:hadTauDecaysResult}) of Section~\ref{sec:appendix_tauToHadDecays} to express Eq.~(\ref{eq:finalLepTauDecaysPSint}) by:
\begin{align}
\vert\mathcal{M}_{\Pgt}\vert^{2} \,
 d\Phi_{\ellnunu} = \frac{\vert\mathcal{M}^{\eff}_{\Pgt \to
  \ellnunu}\vert^{2}}{512\pi^{5}\, m_{\Pgt} \, \Gamma_{\Pgt}} \cdot 
    \frac{I_{\inv}}{\vert\bm{p^{\vis}}\vert \, z^{2}} \, 
    \frac{d^{3}\bm{p^{\vis}}}{2 E_{\vis}} \, dz \, dm^{2}_{\inv} \,
    d\phi_{\inv} \, .
\label{eq:lepTauDecaysResult}
\end{align}
The symbol $\phi_{\inv}$ specifies the orientation of the neutrino
pair momentum vector $\bm{u}$ with respect to the momentum vector $\bm{p^{\vis}}$
of the electron or muon.

The opening angle between the vector $\bm{p^{\inv}}$ and the direction
of the electron or muon is given in analogy to Eq.~(\ref{eq:hadTauDecaysCosTheta}) by:
\begin{equation}
\cos\theta_{\inv} = \frac{E_{\vis} E_{\inv} - \frac{1}{2} \left(m^{2}_{\Pgt} - \left( m^{2}_{\vis} + m^{2}_{\inv} \right) \right)}{\vert\bm{p^{\vis}}\vert \, 
  \vert\bm{p^{\inv}}\vert}.
\label{eq:lepTauDecaysCosTheta}
\end{equation}

We define:
\begin{equation}
f_{\Plepton}\left(\bm{p^{\vis}}, m_{\vis}, \bm{p^{\inv}}\right) = 
\frac{\vert\mathcal{M}^{\eff}_{\Pgt \to
  \ellnunu}\vert^{2}}{512\pi^{6}} \cdot 
    \frac{I_{\inv}}{\vert\bm{p^{\vis}}\vert \, z^{2}} \, .
\label{eq:lepTauDecays_f}
\end{equation}
to obtain:
\begin{equation}
\vert\mathcal{M}_{\Pgt}\vert^{2} \,
 d\Phi_{\ellnunu} = \frac{\pi}{m_{\Pgt} \, \Gamma_{\Pgt}} \,
 f_{\Plepton}(\bm{p^{\vis}}, m_{\vis}, \bm{p^{\inv}}) \, \frac{d^{3}\bm{p^{\vis}}}{2 E_{\vis}} \, dz \, dm^{2}_{\inv} \, d\phi_{\inv}
 \, .
\end{equation}


\section*{References}

\bibliography{svFitMEM.bib}

\end{document}
