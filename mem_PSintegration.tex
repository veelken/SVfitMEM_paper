\subsection{Computation of phase space integral}
\label{sec:mem_PSintegration}

The integration over the differential $n$-particle phase space element
$d\Phi_{n}$ in Eq.~(\ref{eq:mem_with_hadRecoil}) is to be done differently
for $\Pgt\Pgt \to \textrm{hadrons} + \Pnut \, \textrm{hadrons} +
\Pnut$, $\Pgt\Pgt \to \textrm{hadrons} + \Pnut \, \ellnunu$, 
and $\Pgt\Pgt \to \ellnunu \, \ellnunu$ decays ($\Plepton = \Pe, \Pgm$).
With the approximation that we treat hadronic $\Pgt$ decays as
two-body decays, the integral
over $d\Phi_{n}$ is of dimension $12$ in case of $\Pgt\Pgt \to \textrm{hadrons} + \Pnut \, \textrm{hadrons} +
\Pnut$ decays,
of dimension $15$ in case of $\Pgt\Pgt \to \textrm{hadrons} + \Pnut \, \ellnunu$ decays,
and of dimension $18$ in case of $\Pgt\Pgt \to \ellnunu \, \ellnunu$ decays.
The differential phase space element reads:
\begin{equation}
d\Phi_{n}= 
 \begin{cases} 
   d\Phi^{(1)}_{\tauhnu} \, d\Phi^{(2)}_{\tauhnu} \, 
 & \mbox{if } \Pgt\Pgt \to \textrm{hadrons} + \Pnut \, \textrm{hadrons} +
\Pnut \, , \\
   d\Phi^{(1)}_{\ellnunu} \, d\Phi^{(2)}_{\tauhnu} \, 
 & \mbox{if } \Pgt\Pgt \to \textrm{hadrons} + \Pnut \, \ellnunu \, , \\
   d\Phi^{(1)}_{\ellnunu} \, d\Phi^{(2)}_{\ellnunu} \, 
 & \mbox{if } \Pgt\Pgt \to \ellnunu \, \ellnunu \, ,
 \end{cases}
\label{eq:PSintegration_taupair}
\end{equation}
where
\begin{equation}
d\Phi^{(i)}_{\tauhnu} = \frac{d^{3}\bm{p}^{\vis(i)}}{(2\pi)^{3} \, 2
  E_{\vis(i)}} \, \frac{d^{3}\bm{p}^{\Pnu(i)}}{(2\pi)^{3} \, 2 E_{\Pnu(i)}}
\quad \mbox{ and } \quad 
d\Phi^{(i)}_{\ellnunu} = \frac{d^{3}\bm{p}^{\vis(i)}}{(2\pi)^{3} \, 2
  E_{\vis(i)}} \, \frac{d^{3}\bm{p}^{\Pnu(i)}}{(2\pi)^{3} \, 2
  E_{\Pnu(i)}} \, \frac{d^{3}\bm{p}^{\APnu(i)}}{(2\pi)^{3} \, 2
  E_{\APnu(i)}} \, .
\label{eq:PSintegration_onetau}
\end{equation}

The integrand in Eq.~(\ref{eq:mem_with_hadRecoil}) depends on the
four-momentum of the two $\Pgt$ leptons via the squared modulus of the
ME, $\vert \mathcal{M}_{\Pg\Pg \to \PHiggs \to \Pgt\Pgt}(\bm{p},m_{\PHiggs}) \vert^{2}$,
and on the four-momentum of the visible $\Pgt$ decay products via the
transfer functions $W( \bm{p}^{\vis(1)} | \phat^{\vis(1)} )$ and $W( \bm{p}^{\vis(2)} | \phat^{\vis(2)} )$.
In case of hadronic $\Pgt$ decays, 
we parametrize the $\Pgt$ lepton momentum vector as function of the
momentum of the visible $\Pgt$ decay products $\bm{p}^{\vis(i)}$ and of two kinematic variables,
$z$ and $\phi_{\inv}$.
Leptonic $\Pgt$ decays are parametrized by $\bm{p}^{\vis(i)}$ and
three kinematic variables, $z$, $\phi_{\inv}$ and $m_{\inv}$.
The variable $z$ represents the fraction of $\Pgt$ lepton energy
carried by the visible $\Pgt$ decay products (\cf Eq.~(\ref{eq:def_z})).
We denote the energy and momentum of the $\Pgt$ neutrino
produced in hadronic $\Pgt$ decays as well as of the neutrino pair produced in leptonic $\Pgt$
decays by the symbols $E_{\inv}$ and $\bm{p}^{\inv}$.
The energy component $E_{\inv}$ is related to the variable $z$ via:
\begin{equation}
E_{\inv} = E_{\Pgt} - E_{\vis} = \frac{1 - z}{z} \, E_{\vis} \, .
\label{eq:E_inv}
\end{equation}
In case of leptonic $\Pgt$ decays, the variable $m_{\inv}$ denotes the
mass of the neutrino pair. The mass of single neutrinos is assumed
to be zero.
The energy fraction $z$ is related to the angle $\theta_{\inv}$ between the $\bm{p}^{\inv}$ vector and the direction of the visible $\Pgt$ decay products:
\begin{equation}
\cos\theta_{\inv} = \frac{E_{\vis} \, E_{\inv} - \frac{1}{2}(m^{2}_{\Pgt} - (m^{2}_{\vis} + m^{2}_{\inv}))}{\vert\bm{p}^{\vis}\vert \, \vert\bm{p}^{\inv}\vert} \, ,
\label{eq:theta_inv}
\end{equation}
as derived in Sections~\ref{sec:appendix_tauToHadDecays}
and~\ref{sec:appendix_tauToLepDecays} of the appendix for
hadronic and leptonic $\Pgt$ decays, respectively.
The variable $\phi_{\inv}$ specifies the orientation of the
$\bm{p}^{\inv}$ vector relative to the $\bm{p}^{\vis}$ vector.
As illustrated in Fig.~\ref{fig:tauDecayParametrization}, the
$\bm{p}^{\inv}$ vector is located on the surface of a cone that is centred
on the $\bm{p}^{\vis}$ vector and has an opening angle $\theta_{\inv}$.
The variable $\phi_{\inv}$ represents the angle of rotation around the
axis of the cone.
The value $\phi_{\inv} = 0$ is chosen to correspond to the case that
the $\bm{p}^{\inv}$ vector is within the plane spanned by the
$\bm{p}^{\vis}$ vector and the beam direction.
The $\Pgt$ lepton momentum vector is given by the sum of the
$\bm{p}^{\vis}$ and $\bm{p}^{\inv}$ vectors.

\begin{figure}[h]
\begin{center}
\includegraphics*[height=58mm]{figures/tauDecayParametrization.pdf}
\end{center}
\caption{
  Illustration of the variables $\theta_{\inv}$ and $\phi_{\inv}$ used to parametrize the $\Pgt$ lepton decays.
  The angle $\theta_{\inv}$ between the $\bm{p}^{\vis}$ and $\bm{p}^{\inv}$ vectors is related 
  to the fraction $z$ of $\Pgt$ lepton energy carried by the visible $\Pgt$ decay products
  by, respectively, Eqs.~(\ref{eq:hadTauDecaysCosTheta}) and~(\ref{eq:lepTauDecaysCosTheta}) for hadronic and leptonic $\Pgt$ decays.
}
\label{fig:tauDecayParametrization}
\end{figure} 

The parametrization of the $\Pgt$ decay kinematics by $\bm{p}^{\vis}$
and the variables $z$ and $\phi_{\inv}$, respectively by $z$, $\phi_{\inv}$, and $m_{\inv}$,
allows to simplify the evaluation of the integral in Eq.~(\ref{eq:mem_with_hadRecoil}) by means of analytic transformations.
Expressions for the product of the differential phase space elements
$d\Phi^{(i)}_{\tauhnu}$  and $d\Phi^{(i)}_{\ellnunu}$ with the squared
modulus of the ME $\vert \mathcal{M}^{(i)}_{\Pgt}(\bm{p}) \vert^{2}$ are derived in Sections~\ref{sec:appendix_tauToHadDecays} and~\ref{sec:appendix_tauToLepDecays} of the appendix.
The results are:
\begin{align}
\vert \mathcal{M}^{(i)}_{\Pgt} \vert^{2} \, d\Phi^{(i)}_{\tauhnu} 
 = & \, \frac{\pi}{m_{\Pgt}\Gamma_{\Pgt}} \,
 f_{\Phadron}\left(\bm{p}^{\vis(i)}, m^{\vis(i)},
   \bm{p}^{\inv(i)}\right) \, \frac{d^{3}\bm{p}^{\vis}}{2 E_{\vis}} \, dz \, d\phi_{\inv} \nonumber \\
\vert \mathcal{M}^{(i)}_{\Pgt} \vert^{2} \, d\Phi^{(i)}_{\ellnunu} 
 = & \, \frac{\pi}{m_{\Pgt}\Gamma_{\Pgt}} \, f_{\ell}\left(\bm{p}^{\vis(i)},
 m^{\vis(i)}, \bm{p}^{\inv(i)}\right) \, \frac{d^{3}\bm{p}^{\vis}}{2 E_{\vis}} \, dz \, dm^{2}_{\inv} \, d\phi_{\inv}
 \, .
\label{eq:PSint}
\end{align}
The momentum vector $\bm{p}^{\inv(i)}$ is computed as function of
$\bm{p}^{\vis}$ and of the kinematic variables $z$ and $\phi_{\inv}$
in case of hadronic $\Pgt$ decays and as function of $\bm{p}^{\vis}$
and the kinematic variables $z$, $\phi_{\inv}$, and $m_{\inv}$ in case
of leptonic $\Pgt$ decays.
The functions $f_{\Phadron}$ and $f_{\Plepton}$ are given by
Eqs.~(\ref{eq:hadTauDecays_f})
and~(\ref{eq:lepTauDecays_f}).

Eq.~(\ref{eq:PSint}) represents the quintessence of what is needed 
in order to extend the ME generated by automatized tools such as
CompHEP or MadGraph
by the capability to handle the $\Pgt$ decays.
Instead of performing an integration over $d^{3}\bm{p}^{\Pgt(1)} \,
d^{3}\bm{p}^{\Pgt(2)}$, which treats the $\Pgt$ leptons as stable particles,
one needs to perform the integration over $\vert
\mathcal{M}^{(1)}_{\Pgt} \vert^{2} \, d\Phi^{(1)} \, \vert
\mathcal{M}^{(2)}_{\Pgt} \vert^{2} \, d\Phi^{(2)}$ according to
Eq.~(\ref{eq:PSint}).
The momenta of both $\Pgt$ leptons need to be computed as
function of the integration variables $z^{(1)}$, $\phi_{\inv}^{(1)}$,
$m_{\inv}^{(1)}$ and $z^{(2)}$, $\phi_{\inv}^{(2)}$,
$m_{\inv}^{(2)}$, using Eqs.~(\ref{eq:E_inv})
and~(\ref{eq:theta_inv}),
where $m_{\inv}^{(i)}$ is equal to zero in case the $i$-th $\Pgt$ lepton
decays hadronically.
The $\Pgt$ lepton momenta can then be used to evaluate the other
terms in Eq.~(\ref{eq:mem_with_hadRecoil}).

Note that the dimension of the integration depends on the $\Pgt$ decay
mode.
In case both $\Pgt$ leptons decay to hadrons, the dimension of the
integration is $4$, in case one $\Pgt$ lepton decays hadronically and
the other decays leptonically the dimension is $5$, and in case both
$\Pgt$ leptons decay into electrons or muons the dimension is $6$.
 
In order to improve the accuracy of the numerical integration,
we perform a final variable transformation, replacing $z^{(2)}$ by $t_{\PHiggs}$.
The transformation is executed in two steps. 
In the first step, we replace $z^{(2)}$ by:
\begin{equation}
q_{\PHiggs}^{2} = \frac{m^{2}_{\vis}}{z^{(1)} \, z^{(2)}} \, .
\label{eq:varTransform_z2_to_tHiggs_1}
\end{equation}
Following Eq.~(8) in Ref.~\cite{Alwall:2010cq}, we then parametrize $q_{\PHiggs}^{2}$ by:
\begin{equation}
q_{\PHiggs}^{2} = m_{\PHiggs}^{2} + m_{\PHiggs} \, \Gamma_{\PHiggs}
\tan t_{\PHiggs} 
\label{eq:varTransform_z2_to_tHiggs_2}
\end{equation}
in the second step.
The form of the variable transformation in Eqs.~(\ref{eq:varTransform_z2_to_tHiggs_1}) and~(\ref{eq:varTransform_z2_to_tHiggs_2}) 
is chosen such that the Jacobi factor for the transformation from
$z^{(2)}$ to $t_{\PHiggs}$ equals the inverse of $\vert \BW_{\PHiggs}
\vert^{-2}$, the inverse of the squared modulus
of the Breit-Wigner propagator of the $\PHiggs$ boson in
Eq.~(\ref{eq:meHiggsBreitWigner}),
which, by reducing the variance of the integrand,
improves the numerical precision of evaluating the integral 
in Eq.~(\ref{eq:mem_with_hadRecoil}).
The Jacobi factor is given by:
\begin{equation}
\lvert \frac{\partial z^{(2)}}{\partial q_{\PHiggs}^{2}} \, \frac{\partial
  q_{\PHiggs}^{2}}{\partial t_{\PHiggs}} \rvert =
\frac{m^{2}_{\vis}}{q_{\PHiggs}^{4} \, z^{(1)}} \, \frac{(q_{\PHiggs}^{2}
  - m_{\PHiggs}^{2})^{2} + m_{\PHiggs}^{2} \,
  \Gamma_{\PHiggs}^{2}}{m_{\PHiggs} \, \Gamma_{\PHiggs}} \, .
\label{eq:JacobiFactor_z2_to_tHiggs}  
\end{equation}
Compared to Ref.~\cite{Alwall:2010cq} we differ by a factor $\frac{1}{\pi}$ in the derivative 
$\frac{\partial q_{\PHiggs}^{2}}{\partial t_{\PHiggs}}$. We have verified that Eq.~(\ref{eq:JacobiFactor_z2_to_tHiggs}) is correct.

