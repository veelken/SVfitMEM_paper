\subsection{Computation of phase space integral}
\label{sec:mem_PSintegration}

The integration over the differential $n$-particle phase space element
$d\Phi_{n}$ in Eq.~(\ref{eq:mem_with_hadRecoil}) needs to be done differently for
events in which both $\Pgt$ leptons decay hadronically (``hadronic'' $\Pgt$ pair decays),
events in which one $\Pgt$ lepton decays hadronically and one $\Pgt$ lepton decays leptonically (``semi-leptonic'' $\Pgt$ pair decays),
and events in which both $\Pgt$ leptons decay leptonically (``leptonic'' $\Pgt$ pair decays).
With the approximation that we treat hadronic $\Pgt$ decays as
two-body decays, the integral
over $d\Phi_{n}$ is of dimension $12$ in case of hadronic $\Pgt$ pair decays,
$15$ in case of semi-leptonic $\Pgt$ pair decays,
and $18$ in case of leptonic $\Pgt$ pair decays.
The differential phase space element reads:
\begin{equation}
d\Phi_{n}= 
 \begin{cases} 
   d\Phi^{(1)}_{\tauhnu} \, d\Phi^{(2)}_{\tauhnu} \, 
 & \mbox{if } \Pgt^{+} \to \textrm{hadrons} + \APnut \mbox{ and } \Pgt^{-} \to \textrm{hadrons} + \Pnut \\
   d\Phi^{(1)}_{\ellnunu} \, d\Phi^{(2)}_{\tauhnu} \, 
 & \mbox{if } \Pgt^{+} \to \ellPlusnunu \mbox{ and } \Pgt^{-} \to \textrm{hadrons} + \Pnut \\
   d\Phi^{(1)}_{\tauhnu} \, d\Phi^{(2)}_{\ellnunu} \, 
 & \mbox{if } \Pgt^{+} \to \textrm{hadrons} + \APnut \mbox{ and } \Pgt^{-} \to \ellMinusnunu \\
   d\Phi^{(1)}_{\ellnunu} \, d\Phi^{(2)}_{\ellnunu} \, 
 & \mbox{if } \Pgt^{+} \to \ellPlusnunu \mbox{ and } \Pgt^{-} \to \ellMinusnunu \, ,
 \end{cases}
\label{eq:PSintegration_taupair}
\end{equation}
where:
\begin{align}
d\Phi^{(i)}_{\tauhnu} = & \frac{d^{3}\bm{\hat{p}}^{\vis(i)}}{(2\pi)^{3} \, 2
  \hat{E}_{\vis(i)}} \, \frac{d^{3}\bm{\hat{p}}^{\Pnu(i)}}{(2\pi)^{3} \, 2 \hat{E}_{\Pnu(i)}} \nonumber \\
d\Phi^{(i)}_{\ellnunu} = &
\frac{d^{3}\bm{\hat{p}}^{\vis(i)}}{(2\pi)^{3} \, 2 \hat{E}_{\vis(i)}} \, 
\frac{d^{3}\bm{\hat{p}}^{\APnu(i)}}{(2\pi)^{3} \, 2 \hat{E}_{\APnu(i)}} \, 
\frac{d^{3}\bm{\hat{p}}^{\Pnu(i)}}{(2\pi)^{3} \, 2 \hat{E}_{\Pnu(i)}} \, .
\label{eq:PSintegration_onetau}
\end{align}

The integrand in Eq.~(\ref{eq:mem_with_hadRecoil}) depends on the
four-momentum of the two $\Pgt$ leptons via the 
product $f(x_{a}) \, f(x_{b})$ of the PDF,
the factor $1/(2 \, x_{a} \, x_{b} \, s)$, referred to as ``flux factor'' in the literature,
and via the squared modulus $\vert \mathcal{M}_{\Pp\Pp \to \PHiggs \to \Pgt\Pgt}(\bm{\tilde{p}},m_{\PHiggs}) \vert^{2}$ 
of the ME for $\PHiggs$ boson production and subsequent decay into a $\Pgt$ lepton pair.
In addition, it depends on the four-momentum of the visible $\Pgt$ decay products via the
squared moduli $\vert \mathcal{M}^{(1)}_{\Pgt\to\cdots}(\bm{\tilde{p}}) \vert^{2}$ and $\vert \mathcal{M}^{(2)}_{\Pgt\to\cdots}(\bm{\tilde{p}}) \vert^{2}$ of the ME for the $\Pgt$ lepton decays
and via the transfer functions $W( \bm{p}^{\vis(1)} | \phat^{\vis(1)} )$ and $W( \bm{p}^{\vis(2)} | \phat^{\vis(2)} )$.
The $\Pgt$ lepton energies and momenta need to be computed as function of the integration variables.

The dimension of the integration over the phase-space elements
$d^{3}\bm{p}^{\Pnu(i)}$ and $d^{3}\bm{p}^{\APnu(i)} \, d^{3}\bm{p}^{\Pnu(i)}$
can be reduced by means of analytic transformations.
Two variables are sufficient to fully parametrize the kinematics of hadronic $\Pgt$ decays.
In case of leptonic $\Pgt$ decays, three variables are sufficient.
We choose to parametrize hadronic $\Pgt$ decays by the variables $z$ and $\phi_{\inv}$.
The variable $z$ represents the fraction of $\Pgt$ lepton energy, in the laboratory frame,
that is carried by the visible $\Pgt$ decay products (\cf Eq.~(\ref{eq:def_z})).
We denote the energy and momentum of the $\Pgt$ neutrino
produced in hadronic $\Pgt$ decays as well as of the neutrino pair produced in leptonic $\Pgt$
decays by the symbols $E_{\inv}$ and $\bm{p}^{\inv}$.
The energy component $E_{\inv}$ is related to the variable $z$ via:
\begin{equation}
E_{\inv} = \frac{1 - z}{z} \, E_{\vis} \, .
\label{eq:E_inv}
\end{equation}
The angle $\theta_{\inv}$ between the $\bm{p}^{\inv}$ vector and the $\bm{p}^{\vis}$ vector
is related to the variable $z$ as well, 
and is given by Eq.~(\ref{eq:hadTauDecaysCosTheta}) in case of hadronic $\Pgt$ decays 
and by Eq.~(\ref{eq:lepTauDecaysCosTheta}) in case of leptonic $\Pgt$ decays.
The variable $\phi_{\inv}$ specifies the orientation of the
$\bm{p}^{\inv}$ vector relative to the direction of the visible $\Pgt$ decay products.
In case of hadronic $\Pgt$ decays, the magnitude of the $\bm{p}^{\inv}$ vector is equal to $E_{\inv}$.
We choose the mass $m_{\inv}$ of the neutrino pair as third variable to parametrize the kinematics of leptonic $\Pgt$ decays,
so that the magnitude of the $\bm{p}^{\inv}$ vector is given by $\sqrt{\left( \frac{1 - z}{z} \right)^{2} \, E^{2}_{\vis} - m^{2}_{\inv}}$.
With the convention that $m_{\inv} = 0$ for hadronic $\Pgt$ decays,
Eqs.~(\ref{eq:hadTauDecaysCosTheta})
and~(\ref{eq:lepTauDecaysCosTheta}) can be expressed by a common form
that is valid for hadronic as well as for leptonic $\Pgt$ decays:
\begin{equation}
\cos\theta_{\inv} = \frac{\frac{1 - z}{z} \, E^{2}_{\vis} - \frac{1}{2}(m^{2}_{\Pgt} - (m^{2}_{\vis} + m^{2}_{\inv}))}{\vert\bm{p}^{\vis}\vert \, 
  \sqrt{\left( \frac{1 - z}{z} \right)^{2} \, E^{2}_{\vis} - m^{2}_{\inv}}} \, .
\label{eq:theta_inv}
\end{equation}
The $\Pgt$ lepton momentum vector is given by the sum of the
$\bm{p}^{\vis}$ and $\bm{p}^{\inv}$ vectors.

The angles $\theta_{\inv}$ and $\phi_{\inv}$ are illustrated in Fig.~\ref{fig:tauDecayParametrization}.
The $\bm{p}^{\inv}$ vector is located on the surface of a cone,
the axis of which is given by the $\bm{p}^{\vis}$ vector and the
opening angle of which is given by Eq.~(\ref{eq:theta_inv}).
The variable $\phi_{\inv}$ represents the angle of rotation around the
axis of the cone.
The value $\phi_{\inv} = 0$ is chosen to correspond to the case that
the $\bm{p}^{\inv}$ vector is within the plane spanned by the
$\bm{p}^{\vis}$ vector and the beam direction.

\begin{figure}[h]
\begin{center}
\includegraphics*[height=58mm]{figures/tauDecayParametrization.pdf}
\end{center}
\caption{
  Illustration of the variables $\theta_{\inv}$ and $\phi_{\inv}$ that specify the orientation of the $\bm{p}^{\inv}$ vector
  relative to the momentum vector $\bm{p}^{\vis}$ of the visible $\Pgt$ decay products.
}
\label{fig:tauDecayParametrization}
\end{figure} 

The parametrization of the $\Pgt$ decay kinematics by $\bm{p}^{\vis}$
and the variables $z$ and $\phi_{\inv}$, respectively by $z$, $\phi_{\inv}$, and $m_{\inv}$,
allows one to simplify the evaluation of the integral in Eq.~(\ref{eq:mem_with_hadRecoil}) considerably.
Expressions for the product of the differential phase space elements
$d\Phi^{(i)}_{\tauhnu}$  and $d\Phi^{(i)}_{\ellnunu}$ with the squared
modulus of the ME $\vert \BW_{\Pgt} \vert^{2} \cdot \vert \mathcal{M}^{(i)}_{\Pgt\to\cdots}(\bm{p}) \vert^{2}$ are derived in Sections~\ref{sec:appendix_tauToHadDecays} and~\ref{sec:appendix_tauToLepDecays} of the appendix.
The results are:
\begin{align}
\vert \BW_{\Pgt} \vert^{2} \cdot \vert \mathcal{M}^{(i)}_{\Pgt\to\cdots}(\bm{\tilde{p}}) \vert^{2} \, d\Phi^{(i)}_{\tauhnu} 
 = & \, \frac{\pi}{m_{\Pgt}\Gamma_{\Pgt}} \,
 f_{\Phadron}\left(\bm{\hat{p}}^{\vis(i)}, m^{\vis(i)},
   \bm{\hat{p}}^{\inv(i)}\right) \, \frac{d^{3}\bm{\hat{p}}^{\vis}}{2 \hat{E}_{\vis}} \, dz \, d\phi_{\inv} \nonumber \\
\vert \BW_{\Pgt} \vert^{2} \cdot \vert \mathcal{M}^{(i)}_{\Pgt\to\cdots}(\bm{\tilde{p}}) \vert^{2} \, d\Phi^{(i)}_{\ellnunu} 
 = & \, \frac{\pi}{m_{\Pgt}\Gamma_{\Pgt}} \, f_{\ell}\left(\bm{\hat{p}}^{\vis(i)},
 m^{\vis(i)}, \bm{\hat{p}}^{\inv(i)}\right) \, \frac{d^{3}\bm{\hat{p}}^{\vis}}{2 \hat{E}_{\vis}} \, dz \, dm^{2}_{\inv} \, d\phi_{\inv}
 \, .
\label{eq:PSint}
\end{align}
The functions $f_{\Phadron}$ and $f_{\Plepton}$ are given by
Eqs.~(\ref{eq:hadTauDecays_f})
and~(\ref{eq:lepTauDecays_f}).
%The momenta $\bm{\tilde{p}}^{\vis(i)}$ and $\bm{\tilde{p}}^{\inv(i)}$
%in the frame in which the $\PHiggs$ boson has zero $\pT$
%are obtained by a Lorentz boost in the transverse plane.
%The boost vector $(\pThat^{\rec}, \pXhat^{\rec}, \pYhat^{\rec}, 0)$ is given by Eq.~(\ref{eq:xpXhat_and_pYhat}),
%as function of $\bm{\hat{p}}^{\Pgt(1)} = \bm{\hat{p}}^{\vis(1)} + \bm{\hat{p}}^{\inv(1)}$ and $\bm{\hat{p}}^{\Pgt(2)} = \bm{\hat{p}}^{\vis(2)} + \bm{\hat{p}}^{\inv(2)}$
%in the laboratory frame.

Eq.~(\ref{eq:PSint}) represents the quintessence of what is needed 
in order to extend the ME generated by automatized tools such as
CompHEP or MadGraph
by the capability to handle the $\Pgt$ decays.
Instead of performing an integration over $d^{3}\bm{p}^{\Pgt(1)} \,
d^{3}\bm{p}^{\Pgt(2)}$, which treats the $\Pgt$ leptons as stable particles,
one needs to perform the integration over $\vert \BW^{(1)}_{\Pgt} \vert^{2} \cdot \vert
\mathcal{M}^{(1)}_{\Pgt\to\cdots}(\bm{\tilde{p}}) \vert^{2} \, d\Phi^{(1)} \, \vert \BW^{(2)}_{\Pgt} \vert^{2} \cdot \vert
\mathcal{M}^{(2)}_{\Pgt\to\cdots}(\bm{\tilde{p}}) \vert^{2} \, d\Phi^{(2)}$ according to
Eq.~(\ref{eq:PSint}).
The momenta of both $\Pgt$ leptons need to be computed as
function of the integration variables $z^{(1)}$, $\phi_{\inv}^{(1)}$,
$m_{\inv}^{(1)}$ and $z^{(2)}$, $\phi_{\inv}^{(2)}$,
$m_{\inv}^{(2)}$, using Eqs.~(\ref{eq:E_inv})
and~(\ref{eq:theta_inv}),
where $m_{\inv}^{(i)}$ is equal to zero in case the $i$-th $\Pgt$ lepton
decays hadronically.
The $\Pgt$ lepton momenta can then be used to evaluate the product of the PDF, the flux factor, 
and the squared modulus $\vert \mathcal{M}_{\Pp\Pp \to \PHiggs \to \Pgt\Pgt}(\bm{\tilde{p}},m_{\PHiggs}) \vert^{2}$ 
of the ME for $\PHiggs$ boson production and subsequent decay into a $\Pgt$ lepton pair in Eq.~(\ref{eq:mem_with_hadRecoil}).
 
In order to improve the accuracy of the numerical integration,
we perform a further variable transformation, replacing $z^{(2)}$ by $t_{\PHiggs}$.
The transformation is executed in two steps. 
First, we replace $z^{(2)}$ by:
\begin{equation}
q_{\PHiggs}^{2} = \frac{m^{2}_{\vis}}{z^{(1)} \, z^{(2)}} \, .
\label{eq:varTransform_z2_to_tHiggs_1}
\end{equation}
Following Eq.~(8) in Ref.~\cite{Alwall:2010cq}, we then parametrize $q_{\PHiggs}^{2}$ by:
\begin{equation}
q_{\PHiggs}^{2} = m_{\PHiggs}^{2} + m_{\PHiggs} \, \Gamma_{\PHiggs}
\tan t_{\PHiggs} \, .
\label{eq:varTransform_z2_to_tHiggs_2}
\end{equation}
The form of the variable transformation in Eqs.~(\ref{eq:varTransform_z2_to_tHiggs_1}) and~(\ref{eq:varTransform_z2_to_tHiggs_2}) 
is chosen such that the Jacobi factor of the transformation from
$z^{(2)}$ to $t_{\PHiggs}$ is proportional to the inverse of $\vert \BW_{\PHiggs} \vert^{2}$, 
the squared modulus of the Breit-Wigner propagator of the $\PHiggs$ boson in
Eq.~(\ref{eq:meHiggsBreitWigner}).
The Jacobi factor is given by:
\begin{equation}
\left\lvert \frac{\partial z^{(2)}}{\partial q_{\PHiggs}^{2}} \cdot \frac{\partial
  q_{\PHiggs}^{2}}{\partial t_{\PHiggs}} \right\rvert =
\frac{m^{2}_{\vis}}{q_{\PHiggs}^{4} \, z^{(1)}} \cdot \frac{(q_{\PHiggs}^{2}
  - m_{\PHiggs}^{2})^{2} + m_{\PHiggs}^{2} \,
  \Gamma_{\PHiggs}^{2}}{m_{\PHiggs} \, \Gamma_{\PHiggs}} \, .
\label{eq:JacobiFactor_z2_to_tHiggs}  
\end{equation}
Compared to Ref.~\cite{Alwall:2010cq} we differ by a factor $\frac{1}{\pi}$ in the derivative 
$\frac{\partial q_{\PHiggs}^{2}}{\partial t_{\PHiggs}}$. We have verified that Eq.~(\ref{eq:JacobiFactor_z2_to_tHiggs}) is correct.
This transformation improves the numerical precision of evaluating the integral 
in Eq.~(\ref{eq:mem_with_hadRecoil}) by reducing the variance of the integrand.

