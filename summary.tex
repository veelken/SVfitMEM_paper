\section{Summary}
\label{sec:summary}

An algorithm for reconstruction of the $\PHiggs$ boson mass in events
in which the $\PHiggs$ boson decays into a pair of $\Pgt$ leptons has been
presented.
The relative resolution on the $\PHiggs$ boson mass amounts to typically
$15$--$20\%$.
The algorithm has been used in data analyses performed by the CMS
collaboration during LHC Run $1$.
It improves the sensitivity of the SM $\PHiggs \to \Pgt\Pgt$ analysis by $\approx 40\%$,
corresponding to a gain in integrated luminosity by about a factor of two.

An improved version of the algorithm have been developed in preparation
for LHC Run $2$.
Two variants of the improved algorithm have been implemented.
The first variant, SVfitMEM, has been rigorously developed within the
formalism of the ME method. It is based on proper normalization of the probability density 
$\mathcal{P}(\bm{p}^{\vis(1)},\bm{p}^{\vis(2)};\pX^{\rec},\pY^{\rec}|m_{\PHiggs}^{\textrm{test}(i)})$, given by Eq.~(\ref{eq:mem_with_hadRecoil}).
The second variant of the improved algorithm uses a likelihood
function of arbitrary normalization.
It allows to compute, on an event-by-event basis, any kinematic
function of the two $\Pgt$ leptons and substantially reduces the computing time requirements.
A further improvement concerns the modelling of the experimental
resolution on the $\pT$ of $\tauh$ via TF, described in
Section~\ref{sec:mem_TF_tauToHadDecays}.

The performance of the algorithm has been studied in simulated SM
$\PHiggs \to \Pgt\Pgt$ signal and $\PZ/\Pggx \to \Pgt\Pgt$ background
events, as well as in simulated samples of heavy pseudoscalar Higgs
bosons and heavy spin $1$ resonances.
The SVfit algorithm is found to perform well in all event
categories and over the full range in true mass of the $\Pgt$ lepton
pair that we studied.

The development of the formalism to handle $\Pgt$ lepton decays
in the ME method constitutes an important result of this paper.
The formalism allows one to extend the matrix elements generated by automatized tools such as
CompHEP or MadGraph by the capability to handle hadronic as well as leptonic $\Pgt$ decays.
We expect that the formalism will be very useful for future
applications of the ME method to data analyses with $\Pgt$ leptons in
the final state. 
