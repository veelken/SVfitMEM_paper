\section{Summary}
\label{sec:summary}

An algorithm for reconstruction of the $\PHiggs$ boson mass in events
in which the $\PHiggs$ boson decays into a pair of $\Pgt$ leptons has been
presented.
The relative resolution on the $\PHiggs$ boson mass amounts to typically
$15$--$20\%$.
The algorithm has been used in data analyses performed by the CMS
collaboration during LHC run $1$.
It improves the sensitivity of the SM $\PHiggs \to \Pgt\Pgt$ analysis by $\approx 40\%$,
corresponding to a gain in integrated luminosity by about a factor two.

Two improvements to the algorithm have been developed in preparation
for LHC run $2$.
The first improvement concerns the rigorous formulation of the
algorithm in terms of the ME method
and the proper normalization of the of the probability density 
$\mathcal{P}$, given by Eq.~(\ref{eq:mem_with_hadRecoil}).
The modelling of the experimental resolution on the $\pT$ of $\tauh$ via TF in the ME
formalism, described in Section~\ref{sec:mem_TF_tauToHadDecays}, constitutes the second improvement.
The two improvements enhance the relative resolution on the $\PHiggs$ boson mass
achieved by the algorithm by $5$--$10\%$, 
compared to the previous version of the SVfit algorithm, used during LHC run $1$.

The development of the formalism to handle $\Pgt$ lepton decays
in the ME method constitutes an important result of this paper.
The formalism allows to extend the matrix elements generated by automatized tools such as
CompHEP or MadGraph by the capability to handle hadronic as well as leptonic $\Pgt$ decays.
We expect that the formalism will be very useful for future
applications of the ME method to data analyses with $\Pgt$ leptons in
the final state. 
