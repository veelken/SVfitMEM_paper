\section{Discussion}
\label{sec:discussion}

Considering that the resolution on $m_{\Pgt\Pgt}$,
quantified by $\sigma_{l}/\textrm{M}$ and $\sigma_{h}/\textrm{M}$, achieved
by the SVfitMEM and cSVfit algorithms is almost identical, we find that the cSVfit
algorithm represents the best compromise between physics performance and computing time requirements in practical applications of the SVfit algorithm.

The optimal resolution is achieved in case an artificial regularization term of the type described in Section~\ref{sec:mem_logM}, with small positive $\kappa$,
is added to the likelihood function.
We expect the optimal choice of $\kappa$ to depend on the experimental resolution
as well as on the rates of signal and background processes,
and we recommend to perform a reoptimization of $\kappa$ in practical applications of our algorithm.

The merit of the SVfitMEM algorithm is that the 
formalism to treat $\Pgt$ lepton decays in the ME method, developed
for the SVfitMEM algorithm, can be used
in future applications of the ME method to data analyses with $\Pgt$
leptons in the final state.
An example for such an application is the analysis of SM $\PHiggs$ boson production in association with a pair of top quarks ($\Ptop\APtop\PHiggs$)
in final states with a $\Pgt$ lepton~\cite{HIG-17-003},
in which the existence of neutrinos from $\PW \to \Plepton\Pnu$ decays preclude the reconstruction of the $\PHiggs$ boson mass by the SVfit algorithm.

