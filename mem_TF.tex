\subsection{Transfer functions}
\label{sec:mem_TF}

The TF $W(p^{\vis(i)}|\phat^{\vis(i)})$ relate the reconstructed
$\pT$, $\eta$, and $\phi$ components of the momentum vector of the
visible $\Pgt$ decay products to their respective true values,
given by the momentum vector $\phat^{\vis(i)}$.
The case of hadronic and leptonic $\Pgt$ decays is described in
Sections~\ref{sec:mem_TF_tauToHadDecays}
and~\ref{sec:mem_TF_tauToLepDecays}, respectively.
In both cases the neutrinos produced in the $\Pgt$ decays are not
included in these TF, but are treated separately, as detailed in Sections~\ref{sec:appendix_tauToHadDecays} and~\ref{sec:appendix_tauToLepDecays} of the appendix.
The TF $W_{\rec}( \bm{p}^{\rec}| \phat^{\rec} )$ that we use to model the experimental resolution on the
hadronic recoil is discussed in Section~\ref{sec:mem_TF_hadRecoil}.


\subsubsection{$\Pgt \to \textrm{hadrons} + \Pnut$ decays}
\label{sec:mem_TF_tauToHadDecays}

The energy, or equivalently $\pT$, of $\tauh$ decays
is reconstructed with a
resolution of typically $5$--$25\%$ at the
LHC~\cite{Aad:2014rga,TAU-14-001}.
The resolution may vary as function of $\pT$ and $\eta$ and may depend
also on the multiplicity of charged and neutral hadrons produced that
are produced in the $\Pgt$ decay.
The resolution on the $\tauh$ energy is of similar magnitude as the
resolution on $m_{\Pgt\Pgt}$ that we aim to achieve and needs to be
taken into account by suitable TF when evaluating the integral in
Eq.~\ref{eq:mem_with_hadRecoil}.
We denote the TF used to model the $\pT$ of the $\tauh$ system by
$W_{\Phadron}( \bm{p}^{\vis} | \phat^{\vis} )$.

ATLAS as well as CMS report that the energy response for hadronic
$\Pgt$ decays may be asymmetric~\cite{ATLAS:2011tfa,PRF-14-001}.
For the purpose of this paper, we assume the TF for the $\pT$ of the
$\tauh$ system to have the form:
\begin{equation}
W_{\Phadron}( \pT^{\vis} | \pThat^{\vis} ) = 
 \begin{cases}
   \mathcal{N} \, \zeta_{1} \, \left( \frac{\alpha_{1}}{x_{1}} - x_{1} - \frac{x - \mu}{\sigma} \right)^{-\alpha_{1}} \,  
 & \mbox{if } x < x_{1} \, , \\
   \mathcal{N} \, \exp\left( -\frac{1}{2} \, \left( \frac{x - \mu}{\sigma} \right)^{2} \right) \,
 & \mbox{if } x_{1} < x < x_{2} \, , \\
   \mathcal{N} \, \zeta_{2} \, \left( \frac{\alpha_{2}}{x_{2}} - x_{2} - \frac{x - \mu}{\sigma} \right)^{-\alpha_{2}} \,
 & \mbox{if } x > x_{2} \, ,
 \end{cases}
\label{eq:tf_tauToHadDecays_pT}
\end{equation}
with $\mu = 1.0$, $\sigma = 0.03$, $x_{1} = 0.97$, $\alpha_{1} = 7$,
$x_{2} = 1.03$, and $\alpha_{2} = 3.5$.
The factors $\zeta_{1}$ and $\zeta_{2}$ are chosen such that the
function $W_{\Phadron}( \pT^{\vis} | \pThat^{\vis} )$ is continous at
the points $x = x_{1}$ and $x = x_{2}$. 
Their values are $\zeta_{1} = \left( \frac{\alpha}{x_{1}}
\right)^{\alpha_{1}} \, \exp\left( -\frac{1}{2} \, x_{1}^{2} \right)$ and $\zeta_{2} = \left( \frac{\alpha}{x_{2}}
\right)^{\alpha_{2}} \, \exp\left( -\frac{1}{2} \, x_{2}^{2} \right)$
, respectively.
The factor $\mathcal{N}$ is determined by the condition that the function $W_{\Phadron}( \pT^{\vis} | \pThat^{\vis} )$ 
satisfies the normalization condition $\int \, d\pT^{\vis} \, W_{\Phadron}( \pT^{\vis} | \pThat^{\vis} ) \equiv 1$.

The resolution of the direction of the $\tauh$ system is on the level
of a few milliradians and is negligible in practice.
We hence model the TF for the momentum of the system of hadrons
produced in $\Pgt$ decays by the product of Eq.~\ref{eq:tf_tauToHadDecays} and two
$\delta$-functions:
\begin{equation}
W_{\Phadron}( \bm{p}^{\vis} | \phat^{\vis} ) =
 \frac{\sin^{2}\theta_{\vis}}{\pT^{\vis}} \, 
  W_{\Phadron}( \pT^{\vis} | \pThat^{\vis} ) \,
  \delta( \theta_{\vis} - \thetahat_{\vis} ) \, 
  \delta( \phi_{\vis} - \phihat_{\vis} ) \, .
\label{eq:tf_tauToHadDecays}
\end{equation}
The factor $\frac{\sin^{2}\theta_{\vis}}{\pT^{\vis}}$ is needed
to ensure the correct normalization of the TF:
\begin{align}
& \int \, d^{3}\bm{p}  \, W_{\Phadron}( \bm{p}^{\vis} | \phat^{\vis} ) = \int \, d\pX^{\vis} \, d\pY^{\vis} \, d\pZ^{\vis} \, W_{\Phadron}( \bm{p}^{\vis} | \phat^{\vis} ) \nonumber \\
& \qquad = \int \, d\pT^{\vis} \, d\theta_{\vis} \, d\phi_{\vis} \, W_{\Phadron}( \bm{p}^{\vis} | \phat^{\vis} ) \, 
  \frac{\pT^{\vis}}{\sin^{2} \theta_{\vis}} \equiv 1 \, ,
\end{align}
where the factor $\frac{\pT^{\vis}}{\sin^{2} \theta_{\vis}}$ corresponds to the Jacobian of the variable transformation 
from $(\pX^{\vis} = \pT^{\vis} \, \cos\phi_{\vis}, \pY^{\vis} = \pT^{\vis} \, \sin\phi_{\vis}, \pZ^{\vis} = \frac{\pT^{\vis}}{\tan\theta_{\vis}})$ 
to $(\pT^{\vis}, \theta_{\vis}, \phi_{\vis})$.

\subsubsection{$\Pgt \to \enunu$ and $\Pgt \to \mununu$ decays}
\label{sec:mem_TF_tauToLepDecays}

For leptonic $\Pgt$ decays,
we use the TF:
\begin{equation}
W_{\Plepton}( \bm{p}^{\vis} | \phat^{\vis} ) =  
 \frac{\sin^{2} \theta_{\vis}}{\pT^{\vis}} \, 
  \delta( \pT^{\vis} - \pThat^{\vis} ) \, 
  \delta( \theta_{\vis} - \thetahat_{\vis} ) \, 
  \delta( \phi_{\vis} - \phihat_{\vis} ) \, ,
\label{eq:tf_tauToLepDecays}
\end{equation}
assuming that the experimental resolution on $\pT^{\vis}$,
$\eta_{\vis}$, and $\phi_{\vis}$ is negligible for electrons and
muons.


\subsubsection{Hadronic recoil}
\label{sec:mem_TF_hadRecoil}

We use a two-dimensional normal distribution:
\begin{align}
%%& W_{\rec}( \bm{p}^{\rec} | \phat^{\rec} ) = \hspace{7cm} \nonumber \\
%%& \quad
%%  \frac{1}{2\pi \, \sqrt{\vert V \vert}} \, \exp \left( -\frac{1}{2}
%%  \left( \begin{array}{c} \Delta\pX^{\rec} \\ \Delta\pY^{\rec} \end{array} \right)^{T}
%%  \cdot V^{-1} \cdot
%%   \left( \begin{array}{c} \Delta\pX^{\rec} \\ \Delta\pY^{\rec} \end{array} \right)
%%  \right) \, ,
W_{\rec}( \bm{p}^{\rec} | \phat^{\rec} ) = & 
  \frac{1}{2\pi \, \sqrt{\vert V \vert}} \, \exp \left( -\frac{1}{2}
  \left( \begin{array}{c} \Delta\pX^{\rec} \\ \Delta\pY^{\rec} \end{array} \right)^{T}
  \cdot V^{-1} \cdot
   \left( \begin{array}{c} \Delta\pX^{\rec} \\ \Delta\pY^{\rec} \end{array} \right)
  \right) \, ,
\label{eq:tf_hadRecoil}
\end{align}
to model the experimental resolution on the momentum of
the hadronic recoil in the transverse plane.
The symbols:
\begin{equation}
\Delta\pX^{\rec} = \pX^{\rec} - \pXhat^{\rec} \quad \mbox{ and } \quad
\Delta\pY^{\rec} = \pY^{\rec} - \pYhat^{\rec} 
\label{eq:tf_hadRecoil_delta}
\end{equation}
refer to the difference between reconstructed and true values of the
momentum components $\pX^{\rec}$ and $\pY^{\rec}$,
and the covariance matrix
\begin{equation}
V = \left( \begin{array}{cc} \sigma_{x}^{2} & \rho \, \sigma_{x} \, \sigma_{y} \\ \rho \, \sigma_{x} \, \sigma_{y} & \sigma_{y}^{2} \end{array} \right) 
\label{eq:tf_hadRecoil_V}
\end{equation}
represents the experimental resolution,
with $\vert V \vert$ denoting the determinant of $V$.
Two-dimensional normal distributions have been demonstrated to model well the resolution on $\vecMET$ in case of the CMS experiment~\cite{CMS-JME-10-009,CMS-JME-13-003}.
As explained in Section~\ref{sec:mem_hadRecoil}, the resolutions on $\pX^{\rec}$ and $\pY^{\rec}$ and on $\METx$ and $\METy$ are very similar,
motivating the use of TF of the same type for the modelling of the experimental resolution on the hadronic recoil.

For the purpose of this paper,
we assume that the components $\pX^{\rec}$ and $\pY^{\rec}$ are reconstructed with a resolution of $\sigma_{x} = \sigma_{y} = 10$~\GeV each
and that the differences between reconstructed and true values of both
components are independents, \ie $\rho = 0$.
In the real experiment, the covariance matrix $V$ is estimated on an
event-by-event basis,
depending on the hadronic activity in the event,
and using resolution functions that are obtained from the Monte Carlo simulation~\cite{CMS-JME-10-009,CMS-JME-13-003}.
