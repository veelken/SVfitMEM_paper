\subsection{The decays $\Pgt \to \enunu$ and $\Pgt \to \mununu$}
\label{sec:appendix_tauToLepDecays}

We treat leptonic $\Pgt$ decays as three-body decays and do
account for the ME. Assuming the taus to be unpolarized,
the squared modulus of the ME is given by~\cite{Barger:1987nn}:
\begin{equation}
\vert\mathcal{M}_{\Pgt \to \ellnunu} \vert^{2} = 64 \, G^{2}_{F} \,
\left( E_{\Pgt} E_{\APnu} - \bm{p}^{\Pgt} \cdot \bm{p}^{\APnu} \right)
\, \left( E_{\Plepton} E_{\Pnu} - \bm{p}^{\Plepton} \cdot \bm{p}^{\Pnu} \right) \, , 
\label{eq:lepTauDecaysME}
\end{equation}
where $G_{F}$ denotes the Fermi constant, given by Eq.~(\ref{eq:def_G_F}).

The product of the squared modulus of the ME and the phase space
element $d\Phi_{\ellnunu}$ reads:
\begin{align}
& \, \vert \BW_{\Pgt} \vert^{2} \cdot \vert\mathcal{M}_{\textrm{decay}}\vert^{2} \,
 d\Phi_{\ellnunu} = \vert \BW_{\Pgt} \vert^{2} \cdot \vert \mathcal{M}^{(i)}_{\textrm{decay}}
\vert^{2} \, 
  \frac{d^{3}\bm{p}^{\vis}}{(2\pi)^{3} \, 2 E_{\vis}} \, 
  \frac{d^{3}\bm{p}^{\Pnu}}{(2\pi)^{3} \, 2 E_{\Pnu}} \, 
  \frac{d^{3}\bm{p}^{\APnu}}{(2\pi)^{3} \, 2 E_{\APnu}} \nonumber \\
= & (2\pi)^{3} \, \int \, \frac{\pi}{m_{\Pgt} \, \Gamma_{\Pgt}} \,
\delta ( q_{\Pgt}^{2} - m_{\Pgt}^{2} ) \cdot
\vert\mathcal{M}_{\Pgt \to \ellnunu}\vert^{2} \, \delta \left(
  E_{\Pgt} - E_{\vis} - E_{\Pnu} - E_{\APnu} \right) \nonumber \\
& \qquad
\delta^{3} \left( \bm{p}^{\Pgt} - \bm{p}^{\vis} - \bm{p}^{\Pnu} - \bm{p}^{\APnu} \right) \, \frac{d^{3}\bm{p}^{\Pgt}}{(2\pi)^{3} \, 2 E_{\Pgt}} \,
  \frac{d^{3}\bm{p}^{\vis}}{(2\pi)^{3} \, 2 E_{\vis}} \, 
  \frac{d^{3}\bm{p}^{\Pnu}}{(2\pi)^{3} \, 2 E_{\Pnu}} \, 
  \frac{d^{3}\bm{p}^{\APnu}}{(2\pi)^{3} \, 2 E_{\APnu}} \, dq^{2}_{\Pgt} \nonumber \\
= & \frac{8\pi^{4}}{m_{\Pgt} \, \Gamma_{\Pgt}} \, \vert\mathcal{M}_{\Pgt \to
  \ellnunu}\vert^{2} \, \delta \left( E_{\Pgt} - E_{\vis} -
  E_{\Pnu} - E_{\APnu} \right) \, \delta^{3} \left( \bm{p}^{\Pgt} -
  \bm{p}^{\vis} - \bm{p}^{\Pnu} - \bm{p}^{\APnu} \right)  \nonumber \\
& \qquad
  \frac{d^{3}\bm{p}^{\Pgt}}{(2\pi)^{3} \, 2 E_{\Pgt}} \,
  \frac{d^{3}\bm{p}^{\vis}}{(2\pi)^{3} \, 2 E_{\vis}} \,
  \frac{dE_{\Pnu} \, d^{3}\bm{p}^{\Pnu}}{(2\pi)^{3}} \,
  \theta(E_{\Pnu}) \, \delta \left( E_{\Pnu}^{2} - \vert\bm{p}^{\Pnu}\vert^{2} \right) \, 
  \frac{dE_{\APnu} \, d^{3}\bm{p}^{\APnu}}{(2\pi)^{3}} \,
  \theta(E_{\APnu}) \, \delta \left( E_{\APnu}^{2} - \vert\bm{p}^{\APnu}\vert^{2} \right) \, , 
\end{align}
where we have used the identity:
\begin{equation}
\int \, dE \, d^{3}\bm{p} \, \theta(E) \, \delta \left( E^{2} -
  \vert\bm{p}\vert^{2} - m^{2} \right) = \int \, \frac{d^{3}\bm{p}}{2
  \, E} \, ,
\label{eq:PSintFourDim}
\end{equation}
for expressing the second line by the third.
Eq.~(\ref{eq:PSintFourDim}) follows from the delta function rule Eq.~(\ref{eq:deltaFuncRule}).
We assume that the mass of the $\Pnu$ as well as the mass of the $\APnu$ is zero.

We perform a variable transformation from $(E_{\Pnu}, \bm{p}^{\Pnu})$
and $(E_{\APnu}, \bm{p}^{\APnu})$ to:
\begin{align}
(u_{0}, \bm{u}) = & \frac{1}{\sqrt{2}} \, (E_{\Pnu} + E_{\APnu}, \bm{p}^{\Pnu} +
\bm{p}^{\APnu}) \, , \qquad (v_{0}, \bm{v}) = \frac{1}{\sqrt{2}} (
  E_{\Pnu} - E_{\APnu}, \bm{p}^{\Pnu} - \bm{p}^{\APnu} ) \, . \nonumber 
\end{align}
The variables $u_{0}$ and $\bm{u}$ represent the energy and momentum of the neutrino pair.
The determinant of the Jacobi matrix for the transformation from
$(E_{\Pnu}, \bm{p}^{\Pnu}; E_{\APnu}, \bm{p}^{\APnu})$
to $(u_{0}, \bm{u}; v_{0}, \bm{v})$ equals unity.
Expressed in the new variables, the energy and momentum of the
$\Pnu$ and of the $\APnu$ produced in the tau decay are given by:
\begin{align}
( E_{\Pnu}, \bm{p}^{\Pnu} ) = & \, \frac{1}{\sqrt{2}} ( u_{0} + v_{0}, \bm{u}
+ \bm{v} ) \, \mbox{ and } \, ( E_{\APnu}, \bm{p}^{\APnu} ) = \frac{1}{\sqrt{2}} (
u_{0} - v_{0}, \bm{u} - \bm{v} ) \, . \nonumber 
\end{align}

The product of the squared modules of the ME and the phase space
element can then be expressed by:
\begin{align}
& \, \vert\mathcal{M}_{\Pgt}\vert^{2} \,
 d\Phi_{\ellnunu} = \frac{8\pi^{4}}{m_{\Pgt} \, \Gamma_{\Pgt}} \, \delta \left( E_{\Pgt} - E_{\vis} -
  E_{\Pnu} - E_{\APnu} \right) \, \delta^{3} \left( \bm{p}^{\Pgt} -
  \bm{p}^{\vis} - \bm{p}^{\Pnu} - \bm{p}^{\APnu} \right) \nonumber \\
 & \qquad
\frac{d^{3}\bm{p}^{\Pgt}}{(2\pi)^{3} \, 2 E_{\Pgt}} \,
\frac{d^{3}\bm{p}^{\vis}}{(2\pi)^{3} \, 2 E_{\vis}} \,
\frac{du_{0} \, d^{3}\bm{u}}{(2\pi)^{3}} \nonumber \\
 & \qquad
  \vert\mathcal{M}_{\Pgt \to
  \ellnunu}\vert^{2} \, \theta(u_{0} + v_{0}) \, \delta
\left( \frac{u_{0}^{2} - \vert\bm{u}\vert^{2}}{2} + \frac{v_{0}^{2} - \vert\bm{v}\vert^{2}}{2} +
  u_{0} \, v_{0} - \bm{u} \cdot \bm{v} \right) \, 
  \frac{dv_{0} \, d^{3}\bm{v}}{(2\pi)^{3}} \nonumber \\
 & \qquad
  \theta(u_{0}
  - v_{0}) \, \delta \left( \frac{u_{0}^{2} - \vert\bm{u}\vert^{2}}{2} +
    \frac{v_{0}^{2} - \vert\bm{v}\vert^{2}}{2} - u_{0} \, v_{0} + \bm{u}
    \cdot \bm{v} \right) \, .
\label{eq:lepTauDecaysPSint}
\end{align}

We define:
\begin{align}
& \, I_{\inv} = \vert\mathcal{M}_{\Pgt \to
  \ellnunu}\vert^{2} \, \theta(u_{0} + v_{0}) \, \delta
\left( \frac{u_{0}^{2} - \vert\bm{u}\vert^{2}}{2} + \frac{v_{0}^{2} - \vert\bm{v}\vert^{2}}{2} +
  u_{0} \, v_{0} - \bm{u} \cdot \bm{v} \right) \, 
  \frac{dv_{0} \, d^{3}\bm{v}}{(2\pi)^{3}} \nonumber \\
 & \qquad
  \theta(u_{0}
  - v_{0}) \, \delta \left( \frac{u_{0}^{2} - \vert\bm{u}\vert^{2}}{2} +
    \frac{v_{0}^{2} - \vert\bm{v}\vert^{2}}{2} - u_{0} \, v_{0} + \bm{u}
    \cdot \bm{v} \right) \, .
\label{eq:def_Iinv}
\end{align}
The quantity $I_{\inv}$ is a Lorentz invariant quantity. 
As such, it can be computed in any frame and will yield the same value as in the laboratory frame.
We choose to evaluate it in the restframe of the neutrino pair.
In this frame, the energy is given by $u_{0} = m_{\inv}$ 
and the momentum by $\bm{u} = ( 0, 0, 0 )$, with $m_{\inv}$ denoting
the mass of the neutrino pair.
Hence $u_{0} \, v_{0} - \bm{u} \cdot \bm{v} = m_{\inv} \, v_{0} $ in this frame.
Performing the integration over $v_{0}$, we obtain:
\begin{align}
I_{\inv}
= & \, \vert\mathcal{M}_{\Pgt \to
  \ellnunu}\vert^{2} \, \frac{1}{2} \, \frac{dv_{0} \, d^{3}\bm{v}}{(2\pi)^{3}} \, \theta ( u_{0} + v_{0} ) \, 
    \delta \left( \underbrace{\frac{u_{0}^{2} - \vert\bm{u}\vert^{2}}{2}}_{=
        \frac{m^{2}_{\inv}}{2}} + \underbrace{\frac{v_{0}^{2} -
        \vert\bm{v}\vert^{2}}{2}}_{= -\frac{1}{2} \, \vert\bm{v}\vert^{2}} \right)
    \nonumber \\
& \qquad
    \theta ( u_{0} - v_{0} ) \, \underbrace{\delta \left(
        u_{0} \, v_{0} - \bm{u} \cdot \bm{v} \right)}_{= \frac{1}{m_{\inv}} \, \delta ( v_{0} )} \nonumber \\
= & \, \frac{1}{2 m_{\inv}} \, \theta ( u_{0} ) \, \int \, \frac{d^{3}\bm{v}}{(2\pi)^{3}} \, 
  \vert\mathcal{M}_{\Pgt \to
  \ellnunu}\vert^{2} \, \underbrace{\delta \left( \frac{m^{2}_{\inv}}{2} - \frac{\vert\bm{v}\vert^{2}}{2} \right)}_{
    = \frac{1}{\vert\bm{v}\vert} \, \delta \left( \vert\bm{v}\vert - m_{\inv} \right)} \nonumber \\
= & \, \frac{1}{2 m_{\inv}} \, \theta ( u_{0} ) \, \int \, \frac{\vert\bm{v}\vert^2 d\vert\bm{v}\vert d\Omega_{v}}{(2\pi)^{3}} \, 
  \vert\mathcal{M}_{\Pgt \to
  \ellnunu}\vert^{2} \, \frac{1}{\vert\bm{v}\vert} \, \delta \left( \vert\bm{v}\vert - m_{\inv} \right) \nonumber \\
= & \, \frac{1}{2} \, \theta ( u_{0} ) \, \int \, \frac{d\Omega_{v}}{(2\pi)^{3}} \, \vert\mathcal{M}_{\Pgt \to
  \ellnunu}\vert^{2} \, . 
\label{eq:lepTauDecaysI}
\end{align}
We have used the relation $\delta(a + b) \, \delta(a - b) = \delta(2b) \, \delta(a - b) = \frac{1}{2} \, \delta(b) \, \delta(a)$ 
to express Eq.~(\ref{eq:def_Iinv}) by the first line in
Eq.~(\ref{eq:lepTauDecaysI}).
The identify $\frac{v_{0}^{2} - \vert\bm{v}\vert^{2}}{2} =
-\frac{1}{2} \, \vert\bm{v}\vert^{2}$ follows from the
$\delta$-function $\delta ( v_{0} )$ in the first line.

In the restframe of the neutrino pair:
\begin{align}
( E_{\Pgt}, \bm{p}^{\Pgt} ) = & \, ( E_{\Pgt}, 0, 0, \vert\bm{p}^{\Pgt}\vert ) \nonumber \\
( E_{\vis}, \bm{p}^{\vis} ) = & \, ( E_{\vis}, 0, 0, \vert\bm{p}^{\vis}\vert ) \nonumber \\
( E_{\Pnu}, \bm{p}^{\Pnu} ) = & \, \frac{m_{\inv}}{2} \, ( 1, 0, \sin\theta, \cos\theta ) \nonumber \\
( E_{\APnu}, \bm{p}^{\APnu} ) = & \, \frac{m_{\inv}}{2} \, ( 1, 0, \mbox{-}\sin\theta, \mbox{-}\cos\theta ) \, , \nonumber 
\end{align}
where we have chosen the polar axis such that it is parallel to $\bm{p}^{\vis}$.

The ME given by Eq.~(\ref{eq:lepTauDecaysME}) evaluates to:
\begin{align}
\vert\mathcal{M}_{\Pgt \to \ellnunu}\vert^{2} 
 = & \, 64 \, G^{2}_{F} \, 
  \underbrace{\left( E_{\Pgt} \, E_{\vis} - \bm{p}^{\Pgt} \cdot \bm{p}^{\APnu} \right)}_{= E_{\Pgt} \, \frac{m_{\inv}}{2} + \vert\bm{p}^{\Pgt}\vert \, \frac{m_{\inv}}{2} \cos\theta \quad} \,
  \underbrace{\left( E_{\vis} \, E_{\Pnu} - \bm{p}^{\vis} \cdot \bm{p}^{\Pnu} \right)}_{= E_{\vis} \, \frac{m_{\inv}}{2} - \vert\bm{p}^{\vis}\vert \, \frac{m_{\inv}}{2} \cos\theta} \nonumber \\
 = & \, 16 \, G^{2}_{F} \, m^{2}_{\inv} \, \left( E_{\Pgt} + \vert\bm{p}^{\Pgt}\vert \, \cos\theta \right)  \left( E_{\vis} - \vert\bm{p}^{\vis}\vert \, \cos\theta \right) 
\label{eq:lepTauDecaysMErf}
\end{align}
in the restframe of the neutrino pair.

The energy of the $\Pgt$ lepton and of the electron or muon are
related by:
\begin{align}
m^{2}_{\vis} = & E_{\vis}^{2} - \vert\bm{p}^{\vis}\vert^{2} 
 = \left( E_{\Pgt} - u_{0} \right)^{2} - \left( \bm{p}^{\Pgt} - \bm{u} \right)^{2} \nonumber \\
& \qquad
 = m^{2}_{\Pgt} + m^{2}_{\inv} - 2 (E_{\Pgt} \, u_{0} - \bm{p}^{\Pgt} \cdot \bm{u}) 
 = m^{2}_{\Pgt} + m^{2}_{\inv} - 2 m_{\inv} E_{\Pgt} \, ,
\end{align}
from which it follows that:
%\begin{align}
\begin{equation}
%E_{\Pgt} = & \, \frac{m^{2}_{\Pgt} + m^{2}_{\inv} - m^{2}_{\vis}}{2 m_{\inv}} \nonumber \\
%E_{\vis} = & \, E_{\Pgt} - m_{\inv} = \frac{m^{2}_{\Pgt} -
%  m^{2}_{\inv} - m^{2}_{\vis}}{2 m_{\inv}} \, .
E_{\Pgt} = \frac{m^{2}_{\Pgt} + m^{2}_{\inv} - m^{2}_{\vis}}{2
  m_{\inv}} \quad \mbox{ and } \quad E_{\vis} = E_{\Pgt} - m_{\inv} =
\frac{m^{2}_{\Pgt} - m^{2}_{\inv} - m^{2}_{\vis}}{2 m_{\inv}} \, .
\label{eq:lepTauDecaysEn}
%\end{align}
\end{equation}

Substituting Eq.~(\ref{eq:lepTauDecaysMErf}) into Eq.~(\ref{eq:lepTauDecaysI}) yields:
\begin{align}
I_{\inv} 
= & \, \frac{1}{2} \, \theta ( u_{0} ) \, \int \, \frac{d\Omega_{v}}{(2\pi)^{3}} \, \vert\mathcal{M}_{\Pgt \to
  \ellnunu}\vert^{2} \nonumber \\
= & \, 8 \, G^{2}_{F} \, m^{2}_{\inv} \, \theta ( u_{0} ) \, \int \, \frac{d\cos\theta \, d\phi}{(2\pi)^{3}} \, 
  \left( E_{\Pgt} + \vert\bm{p}^{\Pgt}\vert \, \cos\theta \right)  \left( E_{\vis} - \vert\bm{p}^{\vis}\vert \, \cos\theta \right) \nonumber \\
= & \, \frac{G^{2}_{F}}{\pi^{3}} \, m^{2}_{\inv} \, \theta ( u_{0} ) \, \underbrace{\int_{0}^{2 \pi} \, d\phi}_{= 2 \pi} \, 
  \left( E_{\Pgt} \, E_{\vis} \, \underbrace{\int_{-1}^{+1}
      d\cos\theta}_{= 2} \right. \nonumber \\
& \qquad
     \left. + \left( \vert\bm{p}^{\Pgt}\vert \, E_{\vis} - E_{\Pgt} \,
       \vert\bm{p}^{\vis}\vert \right) \,
     \underbrace{\int_{-1}^{+1} \, d\cos\theta \, \cos\theta}_{= 0} 
  - \vert\bm{p}_{\Pgt}\vert \, \vert\bm{p}_{\vis}\vert
       \, \underbrace{\int_{-1}^{+1} \, d\cos\theta \, \cos^{2}\theta}_{= \frac{2}{3}} \right) \nonumber \\
= & \, \frac{2 \, G^{2}_{F}}{\pi^{2}} \, m^{2}_{\inv} \, \theta ( u_{0} ) \, 
  \left( 2 \, E_{\Pgt} \, E_{\vis} - \frac{2}{3} \, \sqrt{E^{2}_{\Pgt} - m^{2}_{\Pgt}} \, \sqrt{E^{2}_{\vis} - m^{2}_{\vis}} \right) \, ,
\label{eq:lepTauDecaysIrf}
\end{align}
with $E_{\Pgt}$ and $E_{\vis}$ given by Eq.~(\ref{eq:lepTauDecaysEn}).
Note that $I
_{\inv}$ depends on a single kinematic variable, $m_{\inv}$, only,
as $m_{\Pgt}$ and $m_{\vis}$ are constants.

Before substituting Eq.~(\ref{eq:lepTauDecaysIrf}) into Eq.~(\ref{eq:lepTauDecaysPSint}),
we perform a variable transformation from $( u_{0}, u_{1}, u_{2}, u_{3} )$ to $( m^{2}_{\inv}, u_{1}, u_{2}, u_{3} ) = ( 2 \, ( {u_{0}}^{2} - {u_{1}}^{2} - {u_{2}}^{2} - {u_{3}}^{2}), u_{1}, u_{2}, u_{3} )$.
The determinant of the Jacobi matrix for this transformation is $\vert J
\vert = 4 \, u_{0}$,
from which it follows that:
\begin{equation}
du_{0} \, d^{3}\bm{u} = \frac{1}{\vert J \vert} \, dm^{2}_{\inv} \,
d^{3}\bm{u} = \frac{1}{4 u_{0}} \, dm^{2}_{\inv} \, d^{3}\bm{u} \, .
\label{eq:lepTauDecaysJacobi}
\end{equation}

Substituting Eqs.~(\ref{eq:lepTauDecaysIrf})
and~(\ref{eq:lepTauDecaysJacobi}) into
Eq.~(\ref{eq:lepTauDecaysPSint}), we then obtain:
\begin{equation}
\begin{aligned}
& \, \vert\mathcal{M}_{\Pgt}\vert^{2} \,
 d\Phi_{\ellnunu} = \frac{8\pi^{4}}{m_{\Pgt} \, \Gamma_{\Pgt}} \,
 \delta \left( E_{\Pgt} - E_{\vis} - u_{0} \right)
 \, \delta^{3} \left( \bm{p}^{\Pgt} - \bm{p}^{\vis} -
  \bm{u} \right) \\
& \qquad \frac{d^{3}\bm{p}^{\Pgt}}{(2\pi)^{3} \, 2 E_{\Pgt}} \,
  \frac{d^{3}\bm{p}^{\vis}}{(2\pi)^{3} \, 2 E_{\vis}} \, 
  \frac{d^{3}\bm{u}}{(2\pi)^{3} \, 2 u_{0}} \, \frac{I_{\inv}}{2} \,
  dm^{2}_{\inv} \, ,
\end{aligned}
\label{eq:finalLepTauDecaysPSint}
\end{equation}
with $u_{0} \equiv E_{\inv}$ and $\bm{u} \equiv \bm{p}^{\inv}$.
The expression in Eq.~(\ref{eq:finalLepTauDecaysPSint}) is very similar in structure to the third line of Eq.~(\ref{eq:hadTauDecaysPSint}),
if we identify the integration over the momentum of the neutrino pair,
given by the phase space element $d^{3}\bm{u}$ in Eq.~(\ref{eq:finalLepTauDecaysPSint}), 
with the integration over the neutrino momentum $d^{3}\bm{p}^{\inv}$ in Eq.~(\ref{eq:hadTauDecaysPSint}).
The differences between the formulae for leptonic and hadronic $\Pgt$ decays
are the additional integration over $dm^{2}_{\inv}$ and the
factor $I_{\inv}/2$ in Eq.~(\ref{eq:finalLepTauDecaysPSint}), which replaces the
factor $\vert\mathcal{M}_{\Pgt \to
  \tauh\Pnut}\vert^{2}$ in Eq.~(\ref{eq:hadTauDecaysPSint}).
Note that Eq.~(\ref{eq:finalLepTauDecaysPSint}) as well as Eq.~(\ref{eq:hadTauDecaysPSint}) refer to the laboratory frame.
The restframe of the neutrino pair was used only for the purpose of evaluating the Lorentz invariant integral $I_{\inv}$.

Since $I_{\inv}$ depends only on the integration variable $m_{\inv}$, 
$I_{\inv}$ can be considered as constant when performing the integration over $d^{3}\bm{p}^{\Pgt}$ and $d^{3}\bm{u}$.
We can hence use Eq.~(\ref{eq:hadTauDecaysResult}) of Section~\ref{sec:appendix_tauToHadDecays} to express Eq.~(\ref{eq:finalLepTauDecaysPSint}) by:
\begin{align}
\vert\mathcal{M}_{\Pgt}\vert^{2} \,
 d\Phi_{\ellnunu} = \frac{1}{256\pi^{5}\, m_{\Pgt} \, \Gamma_{\Pgt}} \cdot 
    \frac{I_{\inv}}{2\vert\bm{p}^{\vis}\vert \, z^{2}} \, 
    \frac{d^{3}\bm{p}^{\vis}}{2 E_{\vis}} \, dz \, dm^{2}_{\inv} \,
    d\phi_{\inv} \, .
\label{eq:lepTauDecaysResult}
\end{align}
The symbol $\phi_{\inv}$ specifies the orientation of the neutrino
pair momentum vector $\bm{u}$ with respect to the momentum vector $\bm{p}^{\vis}$
of the electron or muon.

The opening angle between the vector $\bm{p}^{\inv}$ and the direction
of the electron respectively muon is given in analogy to Eq.~(\ref{eq:hadTauDecaysCosTheta}) by:
\begin{equation}
\cos\theta_{\inv} = \frac{E_{\vis} E_{\inv} - \frac{1}{2} \left(m^{2}_{\Pgt} - \left( m^{2}_{\vis} + m^{2}_{\inv} \right) \right)}{\vert\bm{p}^{\vis}\vert \, 
  \vert\bm{p}^{\inv}\vert}.
\label{eq:lepTauDecaysCosTheta}
\end{equation}

We define:
\begin{equation}
f_{\Plepton}\left(\bm{p}^{\vis}, m_{\vis}, \bm{p}^{\inv}\right) = 
\frac{I_{\inv}}{512\pi^{6} \, \vert\bm{p}^{\vis}\vert \, z^{2}}
\label{eq:lepTauDecays_f}
\end{equation}
to obtain:
\begin{equation}
\vert\mathcal{M}_{\Pgt}\vert^{2} \,
 d\Phi_{\ellnunu} = \frac{\pi}{m_{\Pgt} \, \Gamma_{\Pgt}} \,
 f_{\Plepton}(\bm{p}^{\vis}, m_{\vis}, \bm{p}^{\inv}) \, \frac{d^{3}\bm{p}^{\vis}}{2 E_{\vis}} \, dz \, dm^{2}_{\inv} \, d\phi_{\inv}
 \, ,
\end{equation}
the result that we quote in Eq.~(\ref{eq:PSint}).
