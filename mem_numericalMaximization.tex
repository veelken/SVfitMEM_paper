\subsection{Determination of $\hat{m}_{\Pgt\Pgt}$}
\label{sec:mem_numericalMaximization}

The best estimate, $m_{\Pgt\Pgt}$, for the mass of the $\Pgt$ lepton pair in a given event
is obtained by computing the probability density $\mathcal{P}(\bm{p}^{\vis(1)},\bm{p}^{\vis(2)};\pX^{\rec},\pY^{\rec}|m_{\PHiggs}^{\textrm{test}(i)})$ 
for a series of mass hypotheses $m_{\PHiggs}^{\textrm{test}(i)}$, using Eq.~(\ref{eq:mem_with_hadRecoil}), and determining the value of $m_{\PHiggs}$ that maximizes this probability density.

Graphs of the probability density $\mathcal{P}(\bm{p}^{\vis(1)},\bm{p}^{\vis(2)};\pX^{\rec},\pY^{\rec}|m_{\PHiggs}^{\textrm{test}(i)})$ 
as function of $m_{\PHiggs}^{\textrm{test}(i)}$ are shown in Fig.~\ref{fig:likelihoodGraphs} for two exemplary events.
The events are drawn from a simulated sample of $\PZ/\Pggx \to \Pgt\Pgt$ background events, produced as described in Section~\ref{sec:performance}.
In the first event (``$0$-jet'') the $\PZ$ boson has little $\pT$, while in the second event (``$1$-jet boosted'') the $\PZ$ boson recoils against a high $\pT$ jet.
The scale of the ordinate is adjusted such that the probability density is equal to one
for the value of $m_{\PHiggs}^{\textrm{test}(i)}$ that maximises $\mathcal{P}(\bm{p}^{\vis(1)},\bm{p}^{\vis(2)};\pX^{\rec},\pY^{\rec}|m_{\PHiggs}^{\textrm{test}(i)})$.
The width of the graph reflects the experimental resolution on $m_{\Pgt\Pgt}$.
As will be explained in more detail in Section~\ref{sec:performance}, the resolution on $m_{\Pgt\Pgt}$ improves considerably in case the $\PZ$ boson has high $\pT$.
The graphs of the probability density are superimposed for two cases:
for the case that the artificial regularization term described in Section~\ref{sec:mem_logM} is used respectively not used.
In the $\PZ/\Pggx \to \Pgt\Pgt$ background event in which the $\PZ$ boson has little $\pT$
the probability density decreases slowly as function of the mass hypothesis in case no artificial regularization is used,
with the effect that such events have a sizeable probability for the reconstructed $m_{\Pgt\Pgt}$ to populate, due to resolution effects, 
the region in which the $\PHiggs$ boson signal is searched for.

\begin{figure}
\setlength{\unitlength}{1mm}
\begin{center}
\begin{picture}(160,68)(0,0)
\put(-2.5, 0.0){\mbox{\includegraphics*[height=68mm]{plots_apr07_2017/makeSVfitMEM_likelihoodGraphs_DYJetsToLLM50_0Jets_r1_ls618805_ev123760983.pdf}}}
\put(81.0, 0.0){\mbox{\includegraphics*[height=68mm]{plots_apr07_2017/makeSVfitMEM_likelihoodGraphs_DYJetsToLLM50_1JetBoosted_r1_ls781771_ev156354150.pdf}}}
\end{picture}
\end{center}
\caption{
  Graphs of the probability density $\mathcal{P}(\bm{p}^{\vis(1)},\bm{p}^{\vis(2)};\pX^{\rec},\pY^{\rec}|m_{\PHiggs}^{\textrm{test}(i)})$ 
  as function of the mass hypothesis $m_{\PHiggs}^{\textrm{test}(i)}$ in two exemplary simulated $\PZ/\Pggx \to \Pgt\Pgt$ background events.
  In the event shown on the left (``$0$-jet'') the $\PZ$ boson has little $\pT$, while in the event shown on the right (``$1$-jet boosted'') the $\PZ$ boson recoils against a high $\pT$ jet.
  The scale of the ordinate is adjusted such that the probability density is equal to one
  for the value of $m_{\PHiggs}^{\textrm{test}(i)}$ that maximises $\mathcal{P}(\bm{p}^{\vis(1)},\bm{p}^{\vis(2)};\pX^{\rec},\pY^{\rec}|m_{\PHiggs}^{\textrm{test}(i)})$.
}
\label{fig:likelihoodGraphs}
\end{figure}

The integral in Eq.~(\ref{eq:mem_with_hadRecoil}) is evaluated numerically using the VAMP algorithm.
For each mass hypotheses $m_{\PHiggs}^{\textrm{test}(i)}$ the integrand is evaluated $20,000$ times.
The series of mass hypotheses is defined by a recursive relation: 
\begin{equation}
m_{\PHiggs}^{\textrm{test}(i + 1)} = (1 + \delta) \,  m_{\PHiggs}^{\textrm{test}(i)} \, \mbox{ with } \, m_{\PHiggs}^{\textrm{test}(0)} = m_{\vis} \, ,
\label{eq:mTauTau_step_size}
\end{equation}
where $m_{\vis}$ denotes the mass of the visible $\Pgt$ decay products.
The step size $\delta$ is chosen such that it is small compared to the
expected resolution on $m_{\PHiggs}$,
typically amounting to $15$--$20\%$ relative to the true mass of the $\Pgt$ lepton pair.

In order to reduce the computing time, the series is computed in two passes.
The purpose of the first pass, which uses a step size $\delta = 0.10$, is to find an approximate value of $m_{\Pgt\Pgt}$
that maximizes the probability density $\mathcal{P}$.
The series of mass hypotheses that is used in the first pass stops when $\mathcal{P}$ falls below one per mille 
of the maximal probability density $\mathcal{P}^{\textrm{max}}$
computed for any $m_{\PHiggs}^{\textrm{test}(i)}$ so far in a given event.
In the second pass, which uses a step size of $\delta = 0.01$,
further $\mathcal{P}$ values 
are computed for mass hypotheses $m_{\PHiggs}^{\textrm{test}(i)}$ that
are within a region around the maximum
for which the probability density computed in the first pass exceeds
$0.10 \cdot \mathcal{P}^{\textrm{max}}$.

Finally, the graph of $\log(\mathcal{P} \, \cdot \, \mbox{\GeV}^{8})$ 
versus $m_{\PHiggs}^{\textrm{test}(i)}$ is fitted by a second order polynomial
in the region around the maximum,
and $m_{\Pgt\Pgt}$ is taken to be the point at which the polynomial reaches its maximum.
