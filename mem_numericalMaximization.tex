\subsection{Determination of $\hat{m}_{\Pgt\Pgt}$}
\label{sec:mem_numericalMaximization}

The best estimate, $\hat{m}_{\Pgt\Pgt}$, for the mass of the $\Pgt$ pair in a given event
is obtained by computing the probability density $P(\bm{p^{\vis(1)}},\bm{p^{\vis(2)}},\bm{p^{\rec}}|m_{\Pgt\Pgt}^{\textrm{test}(i)})$ according to Eq.~(\ref{eq:mem_with_hadRecoil}) 
for a series of mass hypotheses $m_{\Pgt\Pgt}^{\textrm{test}(i)}$ and determining the value of $m_{\Pgt\Pgt}$ that maximizes this probability density.
For each mass hypothesis, the integral in Eq.~(\ref{eq:mem_with_hadRecoil}) is evaluated numerically,
using an improved implementation~\cite{VAMP} of the \textsc{VEGAS} algorithm~\cite{VEGAS}.
The series of mass hypotheses is defined by a recursive relation: 
\begin{equation}
m_{\Pgt\Pgt}^{\textrm{test}(i + 1)} = (1 + \delta) \,  m_{\Pgt\Pgt}^{\textrm{test}(i)} \, ,
\label{eq:mTauTau_step_size}
\end{equation}
The step size $\delta$ is chosen such that it is small compared to the expected $m_{\Pgt\Pgt}$ resolution,
which typically amounts to $15$--$20\%$, relative to the true mass of the $\Pgt$ pair.
In order to reduce the computing time the series is computed in two passes.
The purpose of the initial pass, which uses a step size of $\delta = 0.10$, is to find the region of maximal $P(\bm{p^{\vis(1)}},\bm{p^{\vis(2)}},\bm{p^{\rec}}|m_{\Pgt\Pgt}^{\textrm{test}(i)})$.
The series is started at $m_{\Pgt\Pgt}^{\textrm{test}(0)} = m_{\vis}$,
the mass of the visible $\Pgt$ decay products, 
and stops when $P(\bm{p^{\vis(1)}},\bm{p^{\vis(2)}},\bm{p^{\rec}}|m_{\Pgt\Pgt}^{\textrm{test}(i)})$ falls below one per mille of the maximal value of the $P(\bm{p^{\vis(1)}},\bm{p^{\vis(2)}},\bm{p^{\rec}}|m_{\Pgt\Pgt}^{\textrm{test}(i)})$ 
computed for any $m_{\Pgt\Pgt}^{\textrm{test}(i)}$ so far in a given event.
Further $P(\bm{p^{\vis(1)}},\bm{p^{\vis(2)}},\bm{p^{\rec}}|m_{\Pgt\Pgt}^{\textrm{test}(i)})$ values are computed in the second pass,
restricted to the region around the maximum.
A step size of $\delta = 0.01$ is used for the second pass.
The graph of $\log P(\bm{p^{\vis(1)}},\bm{p^{\vis(2)}},\bm{p^{\rec}}|m_{\Pgt\Pgt}^{\textrm{test}(i)})$ versus $m_{\Pgt\Pgt}^{\textrm{test}(i)}$ is fitted by a second order polynomial
in the region around the maximum,
and $\hat{m}_{\Pgt\Pgt}$ is taken to be the point at which the polynomial reaches its maximum.
