\subsection{Artificial regularization term}
\label{sec:mem_logM}

The SM $\PHiggs \to \Pgt\Pgt$ signal is produced at a rate about three orders of magnitude smaller 
than the irreducible $\PZ/\Pggx \to \Pgt\Pgt$ background.
As illustrated in Fig.~\ref{fig:xSection},
the production rate for hypothetical heavy resonances, such as heavy pseudoscalar Higgs bosons or heavy spin $1$ resonances,
typically decreases steeply with mass.
Physics analyses at the LHC will soon start to probe signal cross sections of order $1$~fb.
In order to maintain high sensitivity for the SM $\PHiggs \to \Pgt\Pgt$ signal as well as for hypothetical heavy resonances
it is imperative to reduce, as much as possible, high mass tails in the $m_{\Pgt\Pgt}$ distribution reconstructed in $\PZ/\Pggx \to \Pgt\Pgt$ background events,
as tails that are on or below the per mille level may yet compete with or even dwarf potential signals.

As exemplified in Fig.~\ref{fig:likelihoodGraphs},
high mass tails in the $m_{\Pgt\Pgt}$ distribution for the $\PZ/\Pggx \to \Pgt\Pgt$ background predominantly arise from events in which the $\PZ$ boson has little $\pT$
and the probability density $\mathcal{P}$ 
decreases slowly as function of $m_{\PHiggs}^{\textrm{test}(i)}$.

Penalized maximum likelihood (ML) estimation is an established method for circumventing problems 
that arise in the stability of parameter estimates in case the likelihood function is relatively flat~\cite{penalizedMaximumLikelihood1}.
Instead of maximising the log-likelihood function $\log \mathcal{L}(\Theta|\bm{y})$,
the penalized ML method finds the best estimate $\hat{\Theta}$ for the unknown model parameter $\Theta$ by maximising the sum $\log \mathcal{L}(\Theta|\bm{y}) + r(\Theta)$.
The penalty function $r(\Theta)$ is added to the log-likelihood function in order to pull the best estimate $\hat{\Theta}$ for the parameter $\Theta$
towards a value that has some rationale as good guess for $\Theta$ in information outside of the likelihood function.
The penalized ML approach can be viewed as a method for introducing some tolerable degree of bias 
in exchange for a reduction in the sampling variability of parameter estimates~\cite{penalizedMaximumLikelihood2}.

From a Bayesian perspective, the penalty function $r(\Theta)$ can be interpreted as prior distribution that describes the information that one has on the parameter $\Theta$
outside of any information conveyed by the measured observables $\bm{y}$.

In the context of the SVfit algorithm, we choose a penalty function $r(m_{\PHiggs}^{\textrm{test}(i)})$
which gives preference to low values of $m_{\Pgt\Pgt}$ 
in case the probability density $\mathcal{P}$ 
is relatively flat.
Recall that the cross section for producing resonances decaying into $\Pgt$ lepton pairs 
decreases steeply with the mass of the resonance (\cf Fig.~\ref{fig:xSection}).
The information that large mass values are less likely is not contained within the probability density defined by Eq.~(\ref{eq:mem_with_hadRecoil})
and motivates the use of a penalized ML approach with a penalty function $r(m_{\PHiggs}^{\textrm{test}(i)})$ 
that increases (decreases) for small (large) $m_{\PHiggs}^{\textrm{test}(i)}$.

In previous applications of the SVfit algorithm for data analyses performed by the CMS collaboration,
regularization functions of the type 
$r(m_{\PHiggs}^{\textrm{test}(i)}) = -\kappa \cdot \log(m_{\PHiggs}^{\textrm{test}(i)} \cdot \mbox{~\GeV}^{-1})$ were considered.
We will focus on these type of regularization functions in this paper,
but remark that from a Bayesian perspective a well motivated alternative choice 
would be to use as penalty function the logarithm of the cross section as function of mass.

Our choice of the parameter $\kappa$ is performed with the objective of achieving an
optimal compromise between reducing the high mass tail in the $m_{\Pgt\Pgt}$ distribution reconstructed for 
$\PZ/\Pggx \to \Pgt\Pgt$ background events on the one hand and 
causing no or at most a small bias on the $m_{\Pgt\Pgt}$ distribution
reconstructed in signal events on the other hand.
We find that the optimal value of $\kappa$ depends on the experimental
resolution on $\vecMET$ and on the $\pT$ of $\tauh$ and hence needs to be adjusted to the experimental conditions.
Higher (lower) experimental resolution favors a small (large) value of $\kappa$. 
The optimal value of $\kappa$ may furthermore differ 
for events in which both $\Pgt$ leptons decay hadronically,
events in which one $\Pgt$ lepton decays into hadrons and the other into an electron or muon,
and events in which both $\Pgt$ leptons decay into electrons or muons,
with larger (smaller) values of $\kappa$ being favored in case both (none) of the $\Pgt$ leptons decay hadronically.

The effect of adding a regularization term of the kind 
$r(m_{\PHiggs}^{\textrm{test}(i)}) = -\kappa \cdot \log(m_{\PHiggs}^{\textrm{test}(i)} \cdot \mbox{~\GeV}^{-1})$ 
to the probability density $\mathcal{P}$ given by Eq.~(\ref{eq:mem_with_hadRecoil})
on the distribution in $m_{\Pgt\Pgt}$ reconstructed in $\PZ/\Pggx \to \Pgt\Pgt$ background events
is visualized in Fig.~\ref{fig:logMscatterPlots}.
The effect on events in which the $\PZ$ boson recoils against a high $\pT$ jet, for which the probability density typically exhibits a narrow maximum, is small, 
while the addition of the regularization term effectively reduces the high mass tail in the $m_{\Pgt\Pgt}$ distribution reconstructed in
events in which the $\PZ$ boson has little $\pT$,
for which the probability density decreases more slowly as function of $m_{\PHiggs}^{\textrm{test}(i)}$ (\cf Fig.~\ref{fig:likelihoodGraphs}).

\begin{figure}
\setlength{\unitlength}{1mm}
\begin{center}
\begin{picture}(160,72)(0,0)
\put(-3.5, 0.0){\mbox{\includegraphics*[height=72mm]{plots_apr07_2017/makeSVfitMEM_logMscatterPlots_DYJetsToLLM50_0Jets.pdf}}}
\put(82.0, 0.0){\mbox{\includegraphics*[height=72mm]{plots_apr07_2017/makeSVfitMEM_logMscatterPlots_DYJetsToLLM50_1JetBoosted.pdf}}}
\end{picture}
\end{center}
\caption{
  Correlation between $m_{\Pgt\Pgt}$ values reconstructed in simulated $\PZ/\Pggx \to \Pgt\Pgt$ background events
  in case the artificial regularization term defined by Eq.~(\ref{eq:logM}) is used (ordinate) respectively not used (axis of abscissae).
  The correlation is shown separately for events in which the the $\PZ$ boson has little $\pT$ (left) and for events in which the $\PZ$ boson recoils against a high $\pT$ jet (right).
}
\label{fig:logMscatterPlots}
\end{figure}


