\subsection{Artificial regularization term}
\label{sec:mem_logM}

The $m_{\Pgt\Pgt}$ distribution reconstructed in $\PHiggs \to \Pgt\Pgt$ signal events is expected to peak close to the true value of the $\PHiggs$ boson mass,
while the distribution of $m_{\Pgt\Pgt}$ obtained for the irreducible $\PZ/\Pggx \to \Pgt\Pgt$ background is steeply falling.
The sensitivity to discover a $\PHiggs$ boson signal increases if high mass tails in the $m_{\Pgt\Pgt}$ distribution reconstructed 
for the $\PZ/\Pggx \to \Pgt\Pgt$ background, arising from resolution effects, are avoided.
For this purpose,
we add an artificial regularization term of form 
$\kappa \, \log \left( m_{\PHiggs}^{\textrm{test}(i)} \cdot \mbox{GeV}~{-1} \right)$ 
to the logarithm of the probability density $P(\bm{p}^{\vis(1)},\bm{p}^{\vis(2)};\pX^{\rec},\pY^{\rec}|m_{\PHiggs}^{\textrm{test}(i)})$
computed according to Eq.~(\ref{eq:mem_with_hadRecoil}).
In case of non-zero $\kappa$,
the procedure described in Section~\ref{sec:mem_numericalMaximization} for finding the best estimate $m_{\Pgt\Pgt}$ for the mass of the $\Pgt$ lepton pair is altered.
Instead of fitting 
$\log \left( P(\bm{p}^{\vis(1)},\bm{p}^{\vis(2)};\pX^{\rec},\pY^{\rec}|m_{\PHiggs}^{\textrm{test}(i)}) \cdot \mbox{~GeV}^{8} \right)$ 
versus $m_{\PHiggs}^{\textrm{test}(i)}$ by a second order polynomial,
we fit the sum of $\log \left( P(\bm{p}^{\vis(1)},\bm{p}^{\vis(2)};\pX^{\rec},\pY^{\rec}|m_{\PHiggs}^{\textrm{test}(i)}) \cdot \mbox{~GeV}^{8} \right)$
and $\kappa \, \log \left( m_{\PHiggs}^{\textrm{test}(i)} \cdot \mbox{GeV}~{-1} \right)$.
The parameter $\kappa$ is chosen with the objective of reducing the high mass tail for the $\PZ/\Pggx \to \Pgt\Pgt$ background,
while causing at most a small bias on the $m_{\Pgt\Pgt}$ distribution reconstructed in signal events.
We find values $\kappa \approx 5$ to perform optimal for this purpose.
