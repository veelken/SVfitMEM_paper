\subsection{Artificial regularization term}
\label{sec:mem_logM}

The $m_{\Pgt\Pgt}$ distribution reconstructed in $\PHiggs \to \Pgt\Pgt$ signal events is expected to peak close to the true value of the $\PHiggs$ boson mass,
while the distribution of $m_{\Pgt\Pgt}$ obtained for the irreducible
$\PZ/\Pggx \to \Pgt\Pgt$ background is steeply falling in the region
$m_{\Pgt\Pgt}$ greater than $m_{\PZ}$.
The sensitivity to discover a $\PHiggs$ boson signal increases if high mass tails in the $m_{\Pgt\Pgt}$ distribution reconstructed 
for the $\PZ/\Pggx \to \Pgt\Pgt$ background, arising from resolution effects, are avoided.
For this purpose,
we add an artificial regularization term of form 
$\kappa \cdot \log \left( m_{\PHiggs}^{\textrm{test}(i)} / \mbox{~\GeV} \right)$ 
to the logarithm of the probability density $\mathcal{P}$
computed according to Eq.~(\ref{eq:mem_with_hadRecoil}).
In case of non-zero $\kappa$,
the procedure described in Section~\ref{sec:mem_numericalMaximization} for finding the best estimate $m_{\Pgt\Pgt}$ for the mass of the $\Pgt$ lepton pair is altered.
Instead of fitting 
$\log \left( \mathcal{P} \cdot \mbox{~\GeV}^{8} \right)$ 
versus $m_{\PHiggs}^{\textrm{test}(i)}$ by a second order polynomial,
we fit the sum of $\log \left( \mathcal{P} \cdot \mbox{~\GeV}^{8} \right)$
and $\kappa \cdot \log \left( m_{\PHiggs}^{\textrm{test}(i)} / \mbox{~\GeV} \right)$.
The parameter $\kappa$ is chosen with the objective of achieving an
optimal compromise between reducing the high mass tail for the
$\PZ/\Pggx \to \Pgt\Pgt$ background on the one hand and 
causing no or at most a small bias on the $m_{\Pgt\Pgt}$ distribution
reconstructed in signal events on the other hand.
We find that the optimal value of $\kappa$ depends on the experimental
resolution on $\vecMET$ and on the $\pT$ of $\tauh$ and hence needs to be adjusted to the experimental conditions.
Higher (lower) experimental resolution favors a small (large) value of $\kappa$. 
