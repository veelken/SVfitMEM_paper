\subsection{Computation of $\sigma'(m_{\PHiggs})$}
\label{sec:mem_xSection}

According to the paradigm of the ME method, the normalization factor
$1/\sigma'(m_{\PHiggs})$ in Eq.~(\ref{eq:mem_with_hadRecoil}) is to be computed
by evaluating the integral:
\begin{align}
\sigma'(m_{\PHiggs}) = &
  \frac{32\pi^{4}}{s} \, \int \, d^{3}\bm{p}^{\vis(1)} \, d^{3}\bm{p}^{\vis(2)} \, d\pX^{\rec} \, d\pY^{\rec} \, 
  d\Phi_{n} \, \frac{f(x_{a}) f(x_{b})}{2 \, x_{a} \, x_{b} \, s} \cdot \hspace{2cm} \nonumber \\
& \qquad \vert \mathcal{M}_{\Pp\Pp \to \PHiggs \to \Pgt\Pgt}(\bm{\tilde{p}},m_{\PHiggs}) \vert^{2} 
  \cdot \vert \BW^{(1)}_{\Pgt} \vert^{2} \cdot \vert \mathcal{M}^{(1)}_{\Pgt\to\cdots}(\bm{\tilde{p}}) \vert^{2} 
  \cdot \vert \BW^{(2)}_{\Pgt} \vert^{2} \cdot \vert
  \mathcal{M}^{(2)}_{\Pgt\to\cdots}(\bm{\tilde{p}}) \vert^{2} \cdot \nonumber \\
& \qquad W(\bm{p}^{\vis(1)}|\bm{\hat{p}}^{\vis(1)}) \, W(\bm{p}^{\vis(2)}|\bm{\hat{p}}^{\vis(2)}) \, W_{\rec}( \pX^{\rec},\pY^{\rec} | \pXhat^{\rec},\pYhat^{\rec} ) \, .
\label{eq:xSection}
\end{align}
Of the factors in the integrand of Eq.~(\ref{eq:xSection}),
only the TF depend on the measured momenta $\bm{p}^{\vis(1)}$ and $\bm{p}^{\vis(2)}$ of the visible decay products of the two $\Pgt$ leptons
and on the measured momentum components $\pX^{\rec}$ and $\pY^{\rec}$ of the hadronic recoil.
All other factors depend solely on the true values of the momenta.
According to the TF normalization conditions
\begin{align}
& \int \, d^{3}\bm{p}^{\vis(1)} \, W(\bm{p}^{\vis(1)}|\bm{\hat{p}}^{\vis(1)}) = 1 \nonumber \\
& \int \, d^{3}\bm{p}^{\vis(2)} \, W(\bm{p}^{\vis(2)}|\bm{\hat{p}}^{\vis(2)}) = 1 \nonumber \\
& \int \, d\pX^{\rec} \, d\pY^{\rec} \, W_{\rec}( \pX^{\rec},\pY^{\rec} | \pXhat^{\rec},\pYhat^{\rec} ) = 1 \, ,
\end{align}
so that Eq.~(\ref{eq:xSection}) becomes:
\begin{align}
\sigma'(m_{\PHiggs}) = &
  \frac{32\pi^{4}}{s} \, \int \, 
  d\Phi_{n} \, \frac{f(x_{a}) f(x_{b})}{2 \, x_{a} \, x_{b} \, s} \,
  \hspace{2cm} \nonumber \\
& \qquad \vert \mathcal{M}_{\Pp\Pp \to \PHiggs \to
    \Pgt\Pgt}(\bm{\tilde{p}},m_{\PHiggs}) \vert^{2} 
  \cdot \vert \BW^{(1)}_{\Pgt} \vert^{2} \cdot \vert \mathcal{M}^{(1)}_{\Pgt\to\cdots}(\bm{\tilde{p}}) \vert^{2} 
  \cdot \vert \BW^{(1)}_{\Pgt} \vert^{2} \cdot \vert \mathcal{M}^{(2)}_{\Pgt\to\cdots}(\bm{\tilde{p}}) \vert^{2} \, .
\label{eq:xSection2}
\end{align}

The integral in Eq.~(\ref{eq:xSection2}) is computed numerically,
for $\PHiggs$ boson masses $m_{\PHiggs}$ ranging from $50$ to $5000$~\GeV in steps of $1$~\GeV.
The numeric integration is performed using the VAMP algorithm~\cite{VAMP},
an improved implementation of the VEGAS algorithm~\cite{VEGAS}.
The result is used for the purpose of normalizing the probability density $\mathcal{P}(\bm{p}^{\vis(1)},\bm{p}^{\vis(2)};\pX^{\rec},\pY^{\rec}|m_{\PHiggs})$
in Eq.~(\ref{eq:mem_with_hadRecoil}).

The cross sections $\sigma'(m_{\PHiggs})$ computed in this way cannot 
be directly compared to literature values,
due to the fact that we are applying the LO ME to events in which the $\PHiggs$ boson has non-zero $\pT$.
For the purpose of comparing $\sigma(m_{\PHiggs})$ to literature
values for the LO $\Pg\Pg \to \PHiggs$ cross section,
we compute the integral in Eq.~(\ref{eq:xSection2}) for the case that the $\PHiggs$ boson has zero $\pT$,
by inserting two $\delta$-functions, $\delta(\pX^{\Pgt(1)} + \pX^{\Pgt(2)})$ and $\delta(\pY^{\Pgt(1)} + \pY^{\Pgt(2)})$, into the integrand.
We then use the $\delta$-functions to remove two integration variables
analytically before evaluating the integral numerically.
The values of $\sigma(m_{\PHiggs})$ obtained in this way are shown as function of
$m_{\PHiggs}$ in Fig.~\ref{fig:xSection}.
The values agree with the literature values for the LO $\Pg\Pg \to
\PHiggs$ cross section within $\approx 10\%$. 
The level of agreement is sufficient for our purposes.

\begin{figure}
\begin{center}
%%\includegraphics*[height=60mm]{figures/makeSVfitMEM_xSectionPlot_log.pdf}
\includegraphics*[height=74mm]{plots/makeSVfitMEM_xSectionPlot_log.pdf}
\end{center}
\caption{
  Cross section $\sigma(m_{\PHiggs})$ as function of the $\PHiggs$ boson mass $m_{\PHiggs}$,
  computed for proton-proton collisions at $\sqrt{s} = 13$~\TeV centre-of-mass energy.
}
\label{fig:xSection}
\end{figure}

